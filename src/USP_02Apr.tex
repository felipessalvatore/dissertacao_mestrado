\documentclass{beamer}


\mode<presentation>
{
	\usetheme{Darmstadt}
	\setbeamercovered{transparent}
}

\setbeamertemplate{headline}{}

\usepackage[english]{babel}
\usepackage[utf8]{inputenc}
\usepackage{times}
\usepackage[T1]{fontenc}

\usepackage{calc}

\usepackage{tikz}

\usetikzlibrary{decorations.markings}

\tikzstyle{vertex}=[circle, draw, inner sep=0pt, minimum size=6pt]

\newcommand{\vertex}{\node[vertex]}

\newcounter{Angle}

\setbeamertemplate{bibliography item}[text]


\theoremstyle{definition}
\newtheorem{teor}{Theorem} 
\newtheorem{pro}{Proposition} 
\newtheorem{lema}{Lemma} 
\newtheorem{coro}{Corollary} 
\newtheorem{defn}{Definition}



\newcommand{\Li}{\mathcal{L}}
\newcommand{\C}{\mathcal{C}}
\newcommand{\D}{\mathcal{D}}
\newcommand{\W}{\mathcal{W}}
\newcommand{\R}{\mathcal{R}}
\newcommand{\M}{\mathcal{M}}
\newcommand{\N}{\mathcal{N}}
\newcommand{\V}{\mathcal{V}}
\newcommand{\E}{\mathcal{E}}
\newcommand{\J}{\mathcal{J}}
\newcommand{\Pa}{\mathcal{P}}
\newcommand{\I}{\mathcal{I}}
\newcommand{\up}{\upsilon} 
\newcommand{\Kmodel}{\bl\W,\R,\D,\I,\br}
\newcommand{\Fmodel}{\bl\W,\R,\D,\I,\E \br}
\newcommand{\p}{^{\prime}} 
\newcommand{\pp}{^{\prime\prime}}   
\newcommand{\nmodels}{\not\models}
\newcommand{\nao}{\neg}
\newcommand{\bige}{\bigwedge}
\newcommand{\e}{\wedge}
\newcommand{\see}{\longleftrightarrow}
\newcommand{\ou}{\vee}
\newcommand{\impli}{\rightarrow}
\newcommand{\bigou}{\bigvee}
\newcommand{\todo}{\forall} 
\newcommand{\ex}{\exists} 
\newcommand{\teo}{\vdash}
\newcommand{\vazio}{\emptyset}
\newcommand{\bl}{\langle}
\newcommand{\br}{\rangle}

%	\begin{frame}{Make Titles Informative.}
%		
%		You can create overlays\dots
%		\begin{itemize}
%			\item using the \texttt{pause} command:
%			\begin{itemize}
%				\item
%				First item.
%				\pause
%				\item    
%				Second item.
%			\end{itemize}
%			\item
%			using overlay specifications:
%			\begin{itemize}
%				\item<3->
%				First item.
%				\item<4->
%				Second item.
%			\end{itemize}
%			\item
%			using the general \texttt{uncover} command:
%			\begin{itemize}
%				\uncover<5->{\item
%					First item.}
%				\uncover<6->{\item
%					Second item.}
%			\end{itemize}
%		\end{itemize}
%	\end{frame}
%	


\setbeamertemplate{navigation symbols}{}%remove navigation symbols


\title[First-order justification logic JT45] 
{First-order justification logic JT45}


\author[F.Salvatore] 
{Felipe~Salvatore\inst{1}}

\institute[Universities of Somewhere and Elsewhere] 
{
	\inst{1}%
	graduate student in philosophy\\
	University of São Paulo}


\date[CUNY 2015] 
{USP, April 02, 2015}

\begin{document}
	
\begin{frame}
\titlepage 
\end{frame}


\begin{frame}
	\begin{itemize}
	\item[] {\color{blue}Motivations}
	\vspace{5mm}		
	\item[]Justification logic: a brief introduction
	\vspace{5mm}
	\item[] First-order justification logic
	\vspace{5mm}
	\item[] First-order JT45
	\vspace{5mm}
	\item[] Discussion: Realization
	\vspace{5mm}
	\item[] Discussion: Interpolation
	\end{itemize} 
\end{frame}

\begin{frame}{Motivations}

	
\begin{itemize}
\item Introduce justification terms into epistemic first-order logic.
\vspace{5mm}
\item Investigate the connection between justification logic and modal logic; specifically the role of the \textit{Interpolation Theorem}. 
\begin{itemize}
\item The Interpolation Theorem fails for first-order S5 (FOS5) \cite{Fine79}.
\item The Interpolation Theorem can be restored:


\begin{enumerate} [i)]
\item Restoration through the mechanism of hybrid logic \cite{Areces01}.
\item Restoration through propositional quantification \cite{Fitting02}.
\end{enumerate}

\end{itemize}

\end{itemize}
\end{frame}



\begin{frame}
	\begin{itemize}
	\item[] {Motivations}
	\vspace{5mm}		
	\item[]{\color{blue}Justification logic: a brief introduction}
	\vspace{5mm}
	\item[] First-order justification logic
	\vspace{5mm}
	\item[] First-order JT45
	\vspace{5mm}
	\item[] Discussion: Realization
	\vspace{5mm}
	\item[] Discussion: Interpolation
	\end{itemize} 
\end{frame}


\begin{frame} {Intuiitionistic logic}

\qquad In the debate around foundations of mathematics one of the philosophical positions that arose was Brouwer's intuitionism. 

\vspace{5mm}

\qquad Briefly, intuitionism says that the truth of a mathematical statement should be identified with the proof of that statement. Summarizing the core idea of this position in a slogan: 

\begin{center}
{\color{blue} truth means provability}
\end{center}

\end{frame}


\begin{frame} {BHK semantics}

\qquad The Brouwer–Heyting–Kolmogorov (BHK) semantics gives an informal meaning to the logical connectives $\bot, \e, \ou, \impli, \nao$ in the following way:

\begin{itemize}
	\item $\bot$ is a proposition which has no proof (a absurdity, e.g. $0=1$).
	\item A proof of $\varphi \e \psi$ consist of a proof of $\varphi$ and a proof of $\psi$.
	\item A proof of $\varphi \ou \psi$ is given by exhibiting either a proof of $\varphi$ or a proof of $\psi$.
	\item A proof of $\varphi \impli \psi$ is a construction $f$ transforming any proof $t$ of $\varphi$ into a proof $f(t)$ of $\psi$.
	\item A proof of $\nao \varphi$ is a construction which transforms any proof of $\varphi$ into a proof of a contradiction. 
\end{itemize}

\end{frame}

\begin{frame} {Gödel 1933}
\qquad In \cite{Goedel33} Gödel introduced a new unary operator $B$ to classical logic; $B\varphi$ should be read as `$\varphi$ is provable'. To describe the behavior of this operator Gödel constructed the following calculus (S4):

\vspace{5mm}

\begin{itemize}
 \item[] All tautologies
 \item[] $B\varphi \impli \varphi$
 \item[] $B(\varphi \impli \psi) \impli$ $(B\varphi \impli B\psi)$
 \item[] $B\varphi \impli BB\varphi$
 \item[] (\textit{Modus Ponens}) $\teo \varphi$, $\teo \varphi\impli\psi$ $\Rightarrow$ $\teo \psi$
 \item[] (\textit{Internalization})  $\teo \varphi$ $\Rightarrow$ $\teo B\varphi$
\end{itemize}
\end{frame}

\begin{frame} {Gödel 1933}
\qquad Based on the intuitionistic notion of truth as provability, Gödel defined the following translation:

\begin{itemize}
	\item $p^{B} = Bp$;
	\item $\bot^{B} = \bot$;	
	\item $(\varphi \e \psi)^{B} = (\varphi^{B} \e \psi^{B})$;
	\item $(\varphi \ou \psi)^{B} = (\varphi^{B} \ou \psi^{B})$;	
	\item $(\varphi \impli \psi)^{B} = B(\varphi^{B} \impli \psi^{B})$.
\end{itemize}

\qquad It was shown that this translation `makes sense', i.e., that the following theorem holds:

\begin{center}
	For every formula $\varphi$, Int $\teo \varphi$ iff S4 $\teo \varphi^{B}$.
\end{center}
\end{frame}




\begin{frame} {Gödel 1933}
 \qquad In \cite{Goedel33} Gödel pointed out that S4 does not correspond to the calculus of the predicate $Prov(x)$ -- $\ex y Proof (y,x)$ -- in \textbf{PA}. Simply because S4 proves the formula: 
 
 \begin{center}
 $B(B(\bot) \impli \bot)$
 \end{center}
 
 
\qquad If we translate this formula in the language of \textbf{PA}:
 
 
 
  \begin{center}
$Prov(\ulcorner Prov(\ulcorner\bot\urcorner) \impli \bot\urcorner)$
  \end{center}
 
\qquad Since the following sentences are equivalent in \textbf{PA}:

\begin{center}
	$Prov(\ulcorner\bot\urcorner) \impli \bot$\\
	$\nao Prov(\ulcorner\bot\urcorner)$\\
	$Consist(\textbf{PA})$
\end{center}

\qquad $Prov(\ulcorner Prov(\ulcorner\bot\urcorner) \impli \bot\urcorner)$ means that the consistency of \textbf{PA} is internally provable in \textbf{PA}, which contradicts Gödel's Second Incompleteness Theorem.
\end{frame}


\begin{frame} {Gödel 1938}
\qquad In a lecture in 1938 \cite{Goedel38} Gödel suggested a way to remedy this problem. Instead of using the implicit representation of proofs by the existential quantifier in the formula $\ex y Proof (y,x)$ one can use explicit variables for proofs (like $t$) in the formula $Proof (t,x)$. In these lines, Gödel proposed the following ternary operator

\begin{center}
$tB(\varphi,\psi)$
\end{center}


\qquad which should be read as
\begin{center}
`$t$ is a derivation of $\psi$ from $\varphi$'
\end{center}
\end{frame}


\begin{frame} {Gödel 1938}
\qquad Using $tB(\varphi)$ as an abbreviation of $tB(\top,\varphi)$, Gödel formulate the following axiom system:
\vspace{5mm}
\begin{itemize}
	\item[] All tautologies
	\item[] $tB(\varphi)\impli \varphi$
	\item[] $tB(\varphi,\psi) \impli (sB(\psi,\theta) \impli f(t,s)B(\varphi,\theta))$
	\item[] $tB(\varphi) \impli t\p B(tB(\varphi))$
	\item[] (\textit{Modus Ponens}) $\teo \varphi$, $\teo \varphi\impli\psi$ $\Rightarrow$ $\teo \psi$;
	\item[] (\textit{Internalization})  $\teo \varphi$ $\Rightarrow$ $\teo tB(\varphi)$ (where $t$ is an derivation of $\varphi$).
\end{itemize}
\end{frame}



\begin{frame} {Logic of Proofs (LP)}
\qquad Independently of Gödel's system presented in \cite{Goedel38} (the lecture was published only in 1998), Artemov (in \cite{Artemov01}) propose one new logic called {\color{blue}Logic of Proofs (LP)} which is axiomatized by the following system:     

\begin{itemize}
	\item[] All tautologies
	\item[] $t$$:$$\varphi \impli \varphi$
	\item[] $t$$:$$(\varphi \impli \psi) \impli$ $(s$$:$$\varphi \impli$ $[t\cdot s]$$:$$\psi)$
	\item[] $t$$:$$\varphi \impli$ $!t$$:$$t$$:$$\varphi$
	\item[] $t$$:$$\varphi \impli$ $[t+s]$$:$$\varphi$ 
	\item[] $s$$:$$\varphi \impli$ $[t+s]$$:$$\varphi$
	\item[] (\textit{Modus Ponens}) $\teo \varphi$, $\teo \varphi\impli\psi$ $\Rightarrow$ $\teo \psi$;
	\item[] (\textit{axiom necessitation})  $\teo c$$:$$\varphi$, where $\varphi$ is an axiom and $c$ is a justification constant.
\end{itemize}
\end{frame}


\begin{frame} {Logic of Proofs (LP)}
\qquad LP can mirror derivations in S4. For example:

\vspace{5mm}
In S4:

\begin{enumerate}[1.]

\item $p \impli (p\ou q)$ (tautology)

\item $\Box (p \impli (p\ou q))$ (necessitation)

\item $\Box p \impli \Box(p\ou q)$ (distribution)

\item $q \impli (p\ou q)$(tautology)

\item $\Box (q \impli (p\ou q))$ (necessitation)

\item $\Box q \impli \Box(p\ou q)$ (distribution)

\item $(\Box p \ou \Box q ) \impli \Box(p\ou q)$ (classical reasoning)
\end{enumerate}



\end{frame}



\begin{frame} {Logic of Proofs (LP)}

\vspace{5mm}
In LP:

\vspace{5mm}


\begin{enumerate}[1.]

\item $p \impli (p\ou q)$ (tautology)

\item $c_{1}$$:$$(p \impli (p\ou q))$ (axiom necessitation)

\item $c_{1}$$:$$(p \impli (p\ou q)) \impli (x$$:$$p \impli  [c_{1}\cdot x]$$:$$  
(p\ou q))$ 

\item $x$$:$$p \impli  [c_{1}\cdot x]$$:$$  
(p\ou q)$ (modus ponens) 


\item $q \impli (p\ou q)$(tautology)

\item $c_{2}$$:$$(q \impli (p\ou q))$ (axiom necessitation)

\item $c_{2}$$:$$(q \impli (p\ou q)) \impli (y$$:$$q \impli  [c_{2}\cdot y]$$:$$  
(p\ou q))$ 

\item $y$$:$$q \impli  [c_{2}\cdot y]$$:$$  
(p\ou q)$ (modus ponens) 


\item $[c_{1}\cdot x]$$:$$  
(p\ou q) \impli [[c_{1}\cdot x] + [c_{2}\cdot y]]$$:$$  
(p\ou q)$ 

\item $[c_{2}\cdot y]$$:$$  
(p\ou q) \impli [[c_{1}\cdot x] + [c_{2}\cdot y]]$$:$$  
(p\ou q)$ 


\item $(x$$:$$ p \ou y$$:$$ q ) \impli [[c_{1}\cdot x] + [c_{2}\cdot y]]$$:$$  
(p\ou q)$ (classical reasoning)
\end{enumerate}
\end{frame}




\begin{frame} {Logic of Proofs (LP)}

\qquad If $\varphi$ is a S4 formula, there is a mapping $r$ (called a \textit{realization}) from the occurrences of $B$'s (or boxes) into terms. The result of this mapping on $\varphi$ is denoted $\varphi^{r}$. For example:


\begin{center}
$((\Box p \ou \Box q ) \impli \Box(p\ou q))^{r}$\\ $=$\\ $(x$$:$$ p \ou y$$:$$ q ) \impli [[c_{1}\cdot x] + [c_{2}\cdot y]]$$:$$
(p\ou q)$
\end{center}




\textbf{(Realization Theorem between S4 and LP)} For every $\varphi$ in the language of S4, there is a realization $r$ such that

\begin{center}
S4 $\teo \varphi$ iff LP $\teo \varphi^{r}$ 
\end{center}



\end{frame}


\begin{frame}{Logic of Proofs (LP)}
\qquad There is a way to define an interpretation $*$ of the LP formulas into the sentences of \textbf{PA} (for details see \cite{Artemov01}). And with all this machinery Artemov was able to prove the following result:

\vspace{5mm}


\textbf{(Provability Completeness of Intuitionistic Logic)} For every $\varphi$, for every interpretation $*$, there is a realization $r$ such that

\begin{center}
Int $\teo \varphi$ iff S4 $\teo \varphi^{B}$ iff LP $\teo (\varphi^{B})^{r}$ iff \textbf{PA} $\teo ((\varphi^{B})^{r})^{*}$
\end{center}
\end{frame}


\begin{frame} {JT45}
\qquad LP is just one example of \textit{Justification Logic}. Another example, that is interesting to us, is the one called {\color{blue}JT45}, it extends the language of LP with the unary justification operator $?$ and has the following additional axiom scheme:\\

\begin{center}
$\nao t$$:$$\varphi \impli$ $?t$$:$$\nao t$$:$$\varphi$
\end{center}
 
 \qquad We can prove the realization theorem for this logic too!  
 
 \vspace{5mm}
 
 \textbf{(Realization Theorem between S5 and JT45)} For every $\varphi$ in the language of S5, there is a realization $r$ such that
 
 \begin{center}
 	S5 $\teo \varphi$ iff JT45 $\teo \varphi^{r}$ 
 \end{center}
 
 
 
 
 
\end{frame}




\begin{frame}
	\begin{itemize}
		
		\item[] Motivations
		\vspace{5mm}
		\item[] Justification logic: a brief introduction
		\vspace{5mm}	
		\item[] {\color{blue}First-order justification logic}
		\vspace{5mm}
		\item[] First-order JT45
		\vspace{5mm}
		\item[] Discussion: Realization
		\vspace{5mm}
		\item[] Discussion: Interpolation
	\end{itemize} 
\end{frame}


	
\begin{frame}{From propositional justification logic to first-order}

\qquad Let $\varphi(x)$ be any tautology, and let $t$ be the following derivation:

\begin{enumerate}[1.]
\item $\varphi(x)$ 
\item $\todo x \varphi(x)$                 (generalization)
\item $\todo x\varphi(x) \impli (Q(x) \impli \todo x\varphi(x))$ (tautology)
\item $Q(x) \impli \todo x\varphi(x)$ (Modus Ponens)
\end{enumerate}

\qquad Although $x$ is free in the formula $Q(x) \impli \todo x\varphi(x)$, if $c$ is a term we can not substitute $c$ for $x$ in $t$ in order to obtain a derivation $t(c)$ of $Q(c) \impli \todo x\varphi(x)$ (if we do that we ruin the derivation at 2.).
\end{frame}



\begin{frame}{From propositional justification logic to first-order}
	
\qquad Now, let $s$ be the following derivation:

\begin{enumerate}[1.]
\item $\varphi(x)$ 
\item $\todo x \varphi(x)$                 (generalization)
\item $\todo x\varphi(x) \impli (Q(y) \impli \todo x\varphi(x))$ (tautology)
\item $Q(y) \impli \todo x\varphi(x)$ (Modus Ponens)
\end{enumerate}
	
\qquad $y$ is free in the formula $Q(y) \impli \todo x\varphi(x)$ and moreover for every term $c$ the result of substituting $c$ for $y$ in $s$, $s(c/y)$, is the  derivation of $Q(c) \impli \todo x\varphi(x)$.
\end{frame}




\begin{frame}{From propositional justification logic to first-order}
\qquad These examples show us that there are two different roles of variables in a derivation: a variable can be a \textit{formal symbol} that can be subjected to generalization or a \textit{place-holder} that can be substituted for.

\qquad In $t$, $x$ is both a formal symbol and a place-holder. And in $s$, $x$ is a formal symbol and $y$ is a place-holder.

\qquad This consideration motivates the following definition:

\begin{center}
$x$ \textit{is free in the derivation} $t$ of the formula $\varphi$ iff for every term $c$, $t(c/x)$ is the derivation of $\varphi(c/x)$.
\end{center}

\end{frame}

\begin{frame}{From propositional justification logic to first-order}

\qquad In propositional justification logic we write $t$$: $$\varphi$ to express that $t$ is a derivation of $\varphi$. In order to represent the distinct roles of variables in the first-order justification logic, we are going to write formulas of the form:
\begin{center}
\vspace{5 mm}
$t$$:$$Q(x) \impli \todo x\varphi(x)$\\
\vspace{5 mm}
$s$$:_{\{y\}}$$Q(y) \impli \todo x\varphi(x)$
\end{center}


\qquad The role of $\{y\}$ in $s$$:_{\{y\}}$$Q(y) \impli \todo x\varphi(x)$ is to point out that $y$ is free in the derivation $s$ of $Q(y) \impli \todo x\varphi(x)$. 
\end{frame}

\begin{frame}
	\begin{itemize}
		\item[] Motivations
		\vspace{5mm}
		\item[] Justification logic: a brief introduction
		\vspace{5mm}
		\item[] First-order justification logic
		\vspace{5mm}
		\item[] {\color{blue}First-order JT45}
		\vspace{5mm}
		\item[] Discussion: Realization
		\vspace{5mm}
		\item[] Discussion: Interpolation
	\end{itemize} 
\end{frame}



\begin{frame}{Language of first-order JT45}

\qquad The basic definitions that we present here are taken from the technical report of Artemov and Yarvorskaya \cite{Artemov11}.\\
\vspace{5mm}
Justification Terms
	\begin{center}
		$ t : = p_{i}$   $|$ $c_{i}$ $|$  $(t_{1} \cdot t_{2})$ $|$ $(t_{1} + t_{2})$ $|$  $!t$ $|$ $?t$ $|$ $gen_{x}(t)$
	\end{center}
\vspace{5mm}	
Formulas
	\begin{center}
		$ \varphi : = Q(x_1, \dots, x_n)$   $|$ $\bot$ $|$  $\varphi \impli \psi$ $|$ $\todo x \varphi$ $|$  $t$$:_{X}$$\varphi$
	\end{center}


\end{frame}
	

\begin{frame}{Language of first-order JT45}
\qquad Where $X, Y, \dots$ are variables for finite set of individual variables.  We write $Xy$ instead of $X \cup \{y\}$, in this case it is assumed that $y \notin X$. We use $t$$:$$\varphi$ as an abbreviation for $t$$:_{\vazio}$$\varphi$. And we write L to denote the set of formulas.\\
\vspace{5mm}
\qquad We define the notion of free variables of $\varphi$, $fv(\varphi)$, by induction similarly as in the classical case, the new clause is
\begin{itemize} 
\item If $\varphi$ is $t$$:_{X}$$\psi$, then  $fv(\varphi)$ is $X$.
\end{itemize}
 
\end{frame}	
		
		

	
\begin{frame}{First-order JT45: axiom system}

\qquad First-order JT45 (FOJT45) is axiomatized by the following schemes and inference rules:\\
 \vspace{5 mm}	
		\textbf{A1} classical axioms of first-order logic\\
		\textbf{A2} $t$$:_{Xy}$$\varphi \impli$ $t$$:_{X}$$\varphi$, provided $y$ does not occur free in $\varphi$\\
		
		\textbf{A3} $t$$:_{X}$$\varphi \impli$ $t$$:_{Xy}$$\varphi$ \\
		
		\textbf{B1} $t$$:_{X}$$\varphi \impli \varphi$\\
		
		\textbf{B2} $s$$:_{X}$$(\varphi \impli \psi) \impli$ $(t$$:_{X}$$\varphi \impli$ $[t\cdot s]$$:_{X}$$\psi)$\\
		
		\textbf{B3} $t$$:_{X}$$\varphi \impli$ $[t+s]$$:_{X}$$\varphi$, $s$$:_{X}$$\varphi \impli$ $[t+s]$$:_{X}$$\varphi$\\ 
		
		\textbf{B4} $t$$:_{X}$$\varphi \impli$ $!t$$:_{X}$$t$$:_{X}$$\varphi$\\
		
		
	{\color{blue} 	\textbf{B5} $\nao t$$:_{X}$$\varphi \impli$ $?t$$:_{X}$$\nao t$$:_{X}$$\varphi$}\\
		
		
		\textbf{B6} $t$$:_{X}$$\varphi \impli$ $gen_{x}(t)$$:_{X}$$ \todo x \varphi$, provided $x \notin X$\\
		
		
		\textbf{R1} (\textit{Modus Ponens}) $\teo \varphi$, $\teo \varphi\impli\psi$ $\Rightarrow$ $\teo \psi$ \\
		
		\textbf{R2} (\textit{generalization})  $\teo \varphi$ $\Rightarrow$ $\teo \todo x \varphi$ \\
		
		\textbf{R3} (\textit{axiom necessitation})  $\teo c$$:$$\varphi$, where $\varphi$ is an axiom and $c$ is a justification constant.\\	
\end{frame}	
	

\begin{frame}{Basic Results}


\begin{teor}
	(\textit{Internalization}) Let $p_{0}, \dots, p_{k}$ be justification variables; $X_{0}, \dots, X_{k}$ be finite sets of individual variables, and $X =X_{0} \cup \dots \cup X_{k}$. In these conditions, if  $p_{0}$$:_{X_{0}}$$\varphi_{0}, \dots, p_{k}$$:_{X_{k}}$$\varphi_{k} \teo \psi$, then there is a justification term $t(p_{0}, \dots, p_{k})$ such that 

	\begin{center}
	$p_{0}$$:_{X_{0}}$$\varphi_{0}, \dots, 
	p_{k}$$:_{X_{k}}$$\varphi_{k} \teo t$$:_{X}$$\psi$.
	\end{center}

\end{teor}
\end{frame}




	
	
	
\begin{frame}{Basic Results}
	
\begin{pro}
	(\textit{Explicit counterpart of the Barcan Formula and its converse}) For every formula $\varphi(x)$ and every justification term $t$, there are justification terms $CB(t)$ and $B(t)$ such that: 
	\begin{center}
		$\teo t$$:$$\todo x \varphi(x) \impli \todo x CB(t)$$:_{\{x\}}$$\varphi(x)$\\
		\vspace{2 mm}
		$\teo \todo x t$$:_{\{x\}}$$\varphi(x) \impli B(t)$$:$$\todo x \varphi(x)$
	\end{center}
\end{pro}	
\vspace{2 mm}
\begin{itemize}
\item<2-> $t:\todo x \varphi (x) \impli \todo x [c \cdot t]$$:_{\{x\}} \varphi(x)$
\vspace{2 mm}
\item<3-> $\todo x t$$:_{\{x\}}$$\varphi(x) \impli[r \cdot ?[[c_{3} \cdot [c_{2} \cdot gen_{x}(c_{1})]]\cdot ?t]]$$:\todo x \varphi(x)$   		
\end{itemize}


\end{frame}	
	

	

\begin{frame}{Semantics}
\qquad A possible world semantics for first-order LP is presented in Fitting \cite{Fitting14}. We have adopted the definitions of this paper for FOJT45 and with this definitions we were able to prove a completeness theorem for this logic. 

\vspace{5mm}

\qquad But we leave the semantical part for a future talk...

\end{frame}




\begin{frame}
	\begin{itemize}
		\item[] Motivations
		\vspace{5mm}
		\item[] Justification logic: a brief introduction
		\vspace{5mm}
		\item[] First-order justification logic
		\vspace{5mm}
		\item[] First-order JT45
		\vspace{5mm}
		\item[] {\color{blue}Discussion: Realization}
		\vspace{5mm}
		\item[] Discussion: Interpolation
	\end{itemize} 
\end{frame}


\begin{frame}{Realization}
\qquad Let $\varphi$ be a formula of FOS5. We define the {\color{blue}realization} of $\varphi$ in the language of FOJT45, $\varphi^{r}$, as follows:


\begin{itemize}
\item If $\varphi$ is atomic, then $\varphi^{r} = \varphi$.
\item If $\varphi = \psi\impli\theta$, then $\varphi^{r} = \psi^{r}\impli\theta^{r}$
\item If $\varphi = \todo x \psi$, then $\varphi^{r} = \todo x \psi^{r}$
\item If $\varphi = \Box \psi (x_{1}, \dots, x_{n})$, then $\varphi^{r} =t$$:_{\{x_{1}, \dots, x_{n}\}}$$ \psi^{r}$
\end{itemize}

\qquad A realization is normal if all negative occurrences of $\Box$ are assigned justification variables. It can easily be checked that

\begin{center}
For every $\varphi$, $fv(\varphi) = fv(\varphi^{r})$
\end{center}
\end{frame}



\begin{frame}{Realization}
\qquad Let $\varphi$ be a formula of FOJT45. {\color{blue}The forgetful projection} of $\varphi$, $\varphi^{\circ}$, is defined as follows:


\begin{itemize}
\item If $\varphi$ is atomic, then $\varphi^{\circ} = \varphi$.
\item If $\varphi = \psi\impli\theta$, then $\varphi^{\circ} = \psi^{\circ}\impli\theta^{\circ}$
\item If $\varphi = \todo x \psi$, then $\varphi^{\circ} = \todo x \psi^{\circ}$
\item If $\varphi = t$$:_{X}$$ \psi$, then $\varphi^{\circ} = \Box \todo \vec{y}\psi^{\circ}$\\ where $\vec{y} \in fv(\psi)\backslash X$.
\end{itemize}

\qquad As before, it can easily be checked that

\begin{center}
	For every $\varphi$, $fv(\varphi) = fv(\varphi^{\circ})$
\end{center}	
\end{frame}




\begin{frame}{Realization}

\begin{pro}
For every justification formula $\varphi$,
\begin{center}
If FOJT45 $\teo \varphi$, then FOS5 $\teo \varphi^{\circ}$.
\end{center}
\end{pro}	

\begin{teor}
(Realization) If FOS5 $\teo \varphi$, then FOJT45 $\teo \varphi^{r}$ for a normal realization $r$.
\end{teor}	
\end{frame}



\begin{frame}
	\begin{itemize}
		\item[] Motivations
		\vspace{5mm}
		\item[] Justification logic: a brief introduction
		\vspace{5mm}
		\item[] First-order justification logic
		\vspace{5mm}
		\item[] First-order JT45
		\vspace{5mm}
		\item[] Discussion: Realization
		\vspace{5mm}
		\item[] {\color{blue}Discussion: Interpolation}
	\end{itemize} 
\end{frame}




\begin{frame}{Interpolation}


\begin{itemize}

\item \textit{The Interpolation Theorem} holds for FOS5 iff for every sentences $\varphi$ and $\psi$  if $\teo \varphi \impli\psi$, then there is a formula $\theta$ such that $\teo \varphi \impli \theta$, $\teo \theta \impli \psi$ and the non-logical symbols that occur in $\theta$ occur both in $\varphi$ and $\psi$.
\vspace{5mm}
\item \textit{The Interpolation Theorem} holds for FOJT45 iff for sentences $\varphi$ and $\psi$ if $\teo \varphi \impli\psi$, then there is a formula $\theta$ such that $\teo \varphi \impli \theta$, $\teo \theta \impli \psi$ and the non-logical symbols and the justification terms that occur in $\theta$ occur both in $\varphi$ and $\psi$.
\end{itemize}
	
\end{frame}



\begin{frame}{Interpolation}

\begin{pro}
If the Realization Theorem holds between FOS5 and FOJT45, then the Interpolation Theorem fails for FOJT45. 
\end{pro}

{\color{blue} Proof}\\ 
\qquad Suppose that the Interpolation Theorem holds for FOJT45. By \cite{Fine79}, let $\varphi$ and $\psi$ be sentences such that FOS5 $\teo \varphi \impli \psi$ and there is no interpolant between them.\\
\vspace{5mm}
\qquad By the Realization Theorem, there is a normal realization $r$ such that\\
\vspace{5mm}
FOJT45 $\teo \varphi^{r} \impli \psi^{r}$\\
\end{frame}



\begin{frame}{Interpolation}
\qquad By hypothesis, there is a formula $\theta$ such that the non-logical symbols and the justification terms that occur in $\theta$ occur both in $\varphi^{r}$ and $\psi^{r}$. \\
\vspace{5mm}
FOJT45 $\teo \varphi^{r} \impli \theta$\\
FOJT45 $\teo \theta \impli \psi^{r}$\\	
\vspace{5mm}
\qquad By the forgetful projection: 



\vspace{5mm}
FOS5 $\teo (\varphi^{r} \impli \theta)^{\circ}$\\
FOS5 $\teo (\theta \impli \psi^{r})^{\circ}$\\	
\vspace{5mm}
i.e.,

\vspace{5mm}
FOS5 $\teo \varphi \impli \theta^{\circ}$\\
FOS5 $\teo \theta^{\circ} \impli \psi$\\	

\end{frame}


\begin{frame}{Interpolation}
\qquad Now, since there is no interpolant between $\varphi$ and $\psi$, then there is no relation symbol occurring in $\theta^{\circ}$. Hence, $\theta^{\circ}$ is a formula such that $\bot$ is the only atomic formula that occur in $\theta^{\circ}$. Thus, either $\theta^{\circ}$ is valid or $\theta^{\circ}$ is unsatisfiable.
\vspace{5mm}

\qquad If $\theta^{\circ}$ is valid, then, since $\models \theta^{\circ}\impli \psi$, $\psi$ is valid. And so, $\varphi \impli \psi$ has an interpolant, a contradiction.

\vspace{5mm}

\qquad If $\theta^{\circ}$ is unsatisfiable, then, since $\models \varphi \impli \theta^{\circ}$, $\varphi$ is unsatisfiable. And so, $\varphi \impli \psi$ has an interpolant, a contradiction.


\end{frame}





\begin{frame}

\begin{center}
{\color{blue} Thank you\\ for your attention.}
\end{center} 
	
	
\end{frame}












\begin{frame}[allowframebreaks]
\frametitle{References}
\bibliographystyle{amsplain}
\bibliography{ref}
\end{frame}


\end{document}


