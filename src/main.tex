\documentclass[12pt]{report}
\usepackage[utf8]{inputenc}
\usepackage{graphicx}
\graphicspath{ {images/} } 
\usepackage{color}
\usepackage{enumerate}
\usepackage{cite}
\usepackage{qtree}
\usepackage[utf8]{inputenc}
\usepackage{amssymb}
\usepackage{amsthm}
\usepackage{setspace}
\usepackage{amsmath}
\usepackage{caption}
\usepackage[margin=3.3cm]{geometry}


\title{
	{\textbf{Topics in Modal Quantification Theory}}\\
	{\large University of São Paulo. Faculty of Philosophy, Languages and Literature, and Human Sciences (FFLCH). Department of Philosophy}}
\author{Felipe de Souza Salvatore}
\date{03 June 2015}
 
 

\theoremstyle{definition}
\newtheorem{teor}{Theorem} 
\newtheorem{pro}{Proposition} 
\newtheorem{lema}{Lemma} 
\newtheorem{coro}{Corollary} 
\newtheorem{defn}{Definition}



\newcommand{\A}{\mathfrak{A}}   
\newcommand{\B}{\mathcal{B}}  
\newcommand{\C}{\mathcal{C}}
\newcommand{\Li}{\mathcal{L}}
\newcommand{\D}{\mathcal{D}}
\newcommand{\W}{\mathcal{W}}
\newcommand{\R}{\mathcal{R}}
\newcommand{\M}{\mathcal{M}}
\newcommand{\N}{\mathcal{N}}
\newcommand{\V}{\mathcal{V}}
\newcommand{\E}{\mathcal{E}}
\newcommand{\J}{\mathcal{J}}
\newcommand{\Pa}{\mathcal{P}}
\newcommand{\I}{\mathcal{I}}
\newcommand{\Z}{\mathbb{Z}}
\newcommand{\up}{\upsilon}
\newcommand{\barA}{\bar{A}}
\newcommand{\barD}{\bar{\D}}  
\newcommand{\bara}{\bar{a}} 
\newcommand{\barB}{\bar{\B}}
\newcommand{\barE}{\bar{\E}}
\newcommand{\barDp}{\barD\p}
\newcommand{\Frame}{\bl\W,\R, \D, \barD\br}
\newcommand{\strucA}{\bl\W,\R, \D, \barD,\I\br}
\newcommand{\strucAS}{\bl\W,\D, \barD,\I\br}
\newcommand{\strucASB}{\bl\W,\D,\I\br}
\newcommand{\strucB}{\bl\V,\R\p, \D\p, \barD\p,\I\p\br}
\newcommand{\strucBS}{\bl\V,\B,\barB,\J\br}
\newcommand{\strucBSB}{\bl\V,\B,\J\br}
\newcommand{\p}{^{\prime}} 
\newcommand{\pp}{^{\prime\prime}}   
\newcommand{\iso}{\cong}    
\newcommand{\FLi}{Fml(\Li)}
\newcommand{\Fj}{Fml_{J}}
\newcommand{\Fjv}{Fml_{J}(\textbf{V})}
\newcommand{\FjD}{\D--Fml_{J}}
\newcommand{\Cv}{\C(\textbf{V})}
\newcommand{\nmodels}{\not\models}
\newcommand{\vS}{\models_{FOS5V}}
\newcommand{\vSB}{\models_{FOS5}}
\newcommand{\vSL}{\models_{L}}
\newcommand{\vSs}{\models_{h}}
\newcommand{\nvSs}{\nmodels_{h}} 
\newcommand{\vSp}{\models_{h\p}} 
\newcommand{\F}{\mathcal{F}}
\newcommand{\nao}{\neg}
\newcommand{\bige}{\bigwedge}
\newcommand{\e}{\wedge}
\newcommand{\see}{\longleftrightarrow}
\newcommand{\ou}{\vee}
\newcommand{\impli}{\rightarrow}
\newcommand{\bigou}{\bigvee}
\newcommand{\todo}{\forall} 
\newcommand{\ex}{\exists} 
\newcommand{\teo}{\vdash}
\newcommand{\teoc}{\vdash_{\C}}
\newcommand{\teocv}{\vdash_{\Cv}}
\newcommand{\tp}{\textbf{p}}
\newcommand{\tq}{\textbf{q}}
\newcommand{\tr}{\textbf{r}}
\newcommand{\tvp}{\vec{\textbf{p}}}
\newcommand{\vvarphi}{\vec{\varphi}}
\newcommand{\vazio}{\emptyset}
\newcommand{\bl}{\langle}
\newcommand{\br}{\rangle}
\newcommand{\model}{\bl\W,\R,\D,\I,\E \br}


\begin{document}

\onehalfspacing

\chapter*{Acknowledgements}


\qquad I would first like to acknowledge the academic and personal support of my supervisor Rodrigo Bacellar (a.k.a. Roderick Batchelor). He introduced me to modal logic, and (more important) his enthusiasm in this area influenced me completely. Also, his comments and suggestions were very significant for the final version of this text. 

\qquad Professor Rodrigo Freire is responsible for a great part of my education in mathematical logic. Not only his classes, but also the conversations that I have had with him and the books that he recommended, all of this has profoundly molded my technical knowledge.

\qquad I would like to thank professor Melvin Fitting for receiving me as a visiting research scholar at the Graduate Center at the City University of New York (CUNY). He not only made me feel very welcome at CUNY, but he also had the patience to hear all my comments on first-order justification logic, answer all my questions on different logic related topics; and sent me material when I needed it. I was very inspired by his writings before I went to New York; after meeting him personally I discovered that he is the model researcher of logic that I myself would one day like to be.

\qquad I am grateful to Sergei Artemov in allowing me to present what is now Chapter 5 of this dissertation in his seminar, and his comments were very important to mature my understanding of justification logic. 

\qquad I would also like to thank all my colleagues and friends from USP, UNICAMP and CUNY; especially: Alfredo Roque, Bruno Ramos, Edgar Almeida, Henrique Meretti, Julio de Rizzo and Konstantinos Pouliasis; thank you for all the discussion on logic and for all the laughs.

\qquad A special thanks also goes to David Gilbert for
all the advice that he gave me. I would like to thank the remainder of my thesis committee, Ed\'elcio de Souza and Marcelo Finger. 

\qquad I am grateful to Mike Knight for helping me to correct the English in this thesis.

\qquad I owe a heartfelt thanks to Mariana Bardelli for her unconditional support and kindness. Her presence was fundamental to the realization of this project. 

\qquad Eu gostaria de agraceder meus pais e meu irm\~ao por todo o amor, apoio e compreens\~ao na minha trajet\'oria acad\^emica. 

\qquad Finally, I would like to thank the S\~ao Paulo Research Foundation (Funda\c{c}\~ao de Amparo \`a Pesquisa do Estado de S\~ao Paulo, FAPESP) for the financial support both in Brazil and abroad.



\chapter*{Resumo}

SALVATORE, F. S. T\'opicos em Teoria da Quantifica\c{c}\~ao Modal. 2015. 94 f. Disserta\c{c}\~ao (Mestrado) -- Faculdade de Filosofia, Letras e Ci\^encias Humanas. Departamento de Filosofia, Universidade de S\~ao Paulo, S\~ao Paulo, 2015.  \\


A l\'ogica modal S5 nos oferece um ferramental t\'ecnico para analizar algumas no\c{c}\~oes filos\'oficas centrais (por exemplo,  necesidade metaf\'isica e certos conceitos epistemol\'ogicos como conhecimento e cren\c{c}a). Apesar de ser axiomatizada por princ\'ipios simples, esta l\'ogica apresenta algumas propriedades peculiares. Uma das mais not\'orias \'e a seguinte: podemos provar o Teorema da Interpola\c{c}\~ao para a vers\~ao proposicional, mas esse mesmo teorema n\~ao pode ser provado quando adicionamos quantificadores de primeira ordem a essa l\'ogica. Nesta disserta\c{c}\~ao vamos estudar a falha dos Teoremas da Definibilidade e da Interpola\c{c}\~ao para a vers\~ao quantificada de S5. Ao mesmo tempo, vamos combinar os resultados da \textit{l\'ogica da justifica\c{c}\~ao} e investigar a contraparte da vers\~ao quantificada de S5 na l\'ogica da justifica\c{c}\~ao (a l\'ogica chamada JT45 de primeira ordem). Desse modo, vamos explorar a rela\c{c}\~ao entre l\'ogica modal e l\'ogica da justifica\c{c}\~ao para ver se a l\'ogica da justifica\c{c}\~ao pode contribuir para a restaura\c{c}\~ao do Teorema da Interpola\c{c}\~ao.\\

Palavras-chave: l\'ogica, l\'ogica modal de primeira ordem, l\'ogica da justifica\c{c}\~ao, interpola\c{c}\~ao.  
 

\chapter*{Abstract}

SALVATORE, F. S. Topics in Modal Quantification Theory. 2015. 94 f. Thesis (Master Degree) -- Faculty of Philosophy, Languages and Literature, and Human Sciences. Department of Philosophy, University of S\~ao Paulo, S\~ao Paulo, 2015.\\

The modal logic S5 gives us a simple technical tool to analyze some main notions from philosophy (e.g. metaphysical necessity and epistemological concepts such as knowledge and belief). Although S5 can be axiomatized by some simple rules, this logic shows some puzzling properties. For example, an interpolation result holds for the propositional version, but this same result fails when we add first-order quantifiers to this logic. In this dissertation, we study the failure of the Definability and Interpolation Theorems for first-order S5. At the same time, we combine the results of justification logic and we investigate the quantified justification counterpart of S5 (first-order JT45). In this way we explore the relationship between justification logic and modal logic to see if justification logic can contribute to the literature concerning `the restoration of the Interpolation Theorem'.\\

Keywords: logic, first-order modal logic, justification logic, interpolation.




\tableofcontents{}

\chapter{Introduction}
\section{Modal quantification theory: adding first-order quantifiers to modal logic}
\qquad The aim of the present thesis is to present a study of some relevant topics concerning first-order modal logic. To make our intentions precise, some restrictions must be made.

\qquad It is ambiguous to write `first-order modal logic', because unlike in classical logic we can use alternative propositional logics to be the background logic, and even when we choose one specific propositional logic, there are different choices that can be made to construct the first-order version of modal logic. Among the variety of propositional logics, in this thesis we are only concerned with S5. This choice is not arbitrary, because among the other propositional modal logics \textit{S5 is more interesting for the researcher in philosophy}. To be more precise, S5 gives us a plausible way to deal with main philosophical concepts such as the concept of metaphysical necessity and the epistemological notions of knowledge and belief.

\qquad Although use of first-order S5 is made mainly by philosophers, from the technical point of view, this logic is a very intriguing one. That is the richness of first-order S5 as a research subject: by studying this logic, \textit{we can study a complex technical subject and at the same time we can stay connected with deep philosophical problems}.


\qquad The failure of the Interpolation Theorem is a good example. Also known as \textit{Craig's interpolation theorem}, this theorem was first proved for classical logic. It states that if a formula $\varphi$ implies a formula $\psi$ then there is a formula  $\theta$, referred to as \textit{the interpolant between $\varphi$ and $\psi$}, such that every nonlogical symbol in $\theta$ occurs both in $\varphi$ and $\psi$, $\varphi$ implies $\theta$, and $\theta$ implies $\psi$.   

\qquad Investigating if this theorem holds in other logics (like modal logic) is, from the technical point of view, interesting in itself. The richness that we mentioned is that we can show that this theorem fails for first-order S5 and moreover from this failure we can conclude some philosophical implications.

\qquad In the metaphysical debate on necessity and existence there are two main positions: one claims that necessarily everything is necessarily something, i.e. \textit{existence is necessary}; the other claims that possibly something is possibly nothing, i.e. \textit{existence is contingent}. Following Timothy Williamson \cite{Willi}, we call the first position \textit{Necessitism} and the second \textit{Contingentism}. 

\qquad Suppose we assume the Contigentism thesis. Then, sometimes different possible worlds have different inhabitants. In this setting, it makes sense to define two kinds of quantifiers: the \textit{inner quantifiers} $\ex$ and $\todo$; and the \textit{outer quantifiers} $\Sigma$ and $\Pi$. Without entering into the technicalities, we can say, very informally, that the formula $\ex x \varphi$ is true at the world $w$, if $\varphi$ is true at $w$ for a specific interpretation that interprets $x$ into an inhabitant of $w$. On the other hand, $\Sigma x \varphi$ is true at the world $w$, if $\varphi$ is true at $w$ for a specific interpretation that interprets $x$ into an inhabitant of any possible world $w\p$. As usual, we define $\todo$ as the dual of $\ex$ and $\Pi$ as the dual of $\Sigma$.
 
\qquad Saul Kripke pointed out in \cite{Kripke83} that from the failure of the Interpolation Theorem for first-order S5, it follows that, for this modal logic, the outer quantifiers are not definable in the usual modal language with the inner quantifiers. And so a philosophical result follows: some metaphysical notions that can be expressed using quantification over possible entities cannot be emulated by a restricted language which has only quantification over actual entities. Putting it another way: \textit{if we assume the Contigentism thesis and if we assume that the only meaningful discourse is the one that only speaks about actual entities, then there are some metaphysical notions that we are not going to be able to express}.

\qquad Going beyond this problem of metaphysical modality, the reason for the failure of the Interpolation Theorem is understood as a lack of expressiveness of the quantified version of S5. This lack of expressive power had left a natural question open: if we add more machinery to first-order S5 are we able to restore the Interpolation Theorem? The answer is not an easy one. There are many ways to extend the expressiveness of modal logic. In the literature around `the restoration of the Interpolation Theorem' we find examples where the restoration of this theorem can be obtained (e.g., when we use \textit{hybrid logics} \cite{Areces01}, or when we use propositional quantifiers \cite{Fitting02}), but there is not a general argument to show how to extend the expressive power of modal logic in order to guarantee the Interpolation Theorem.

\qquad That is why, from a theoretical point of view, it is significant to investigate how the Interpolation Theorem behaves in different extensions of modal logic.

\qquad Justification logic is a term used to classify a relatively new kind of modal-like logics. The first justification logic, LP (Logic of Proofs), was originated from a question in provability logic (the logic that arises when we interpret the modal formulas with arithmetical semantics). Nowadays we work with an extensive family of propositional justification logics. And, for the philosophical discussion, the interest in justification logic lies in the connection between this logic and some epistemic notions: as the name indicates, justification logic enables us to introduce the notion of \textit{justification} into the setting of epistemic logic. 

\qquad Although justification logic is now a well-studied subject, the main focus is on the propositional case. There are only a few papers in quantified justification logic, and the majority of those papers investigate the justification counterpart of first-order S4.


\qquad In this thesis we present the failure of the Definability and Interpolation Theorems for first-order S5. We establish the basic setting for the justification counterpart of the first-order version of S5. And we indicate how we can relate the failure of the Interpolation Theorem to the research agenda of justification logic.

\qquad In Chapter 2 we give an introduction to the basic subjects that are present when modal operators and quantifiers come to the discussion. In Chapter 3 we present the now classical proofs of the failure of Interpolation and Beth's Definability theorems.  In Chapter 4 we give a brief presentation of justification logic. In Chapter 5 we present the justification counterpart of quantified S5 (called first-order JT45). And in Chapter 6 we comment on how all the topics presented in this thesis can be combined in order to advance the research on modal logic. 



\section{Notation}

\qquad  In this text we abbreviate `if and only if' with `iff', and we use the following set-theoretical notation:

\begin{itemize}
\item $\Pa(A)$ denotes the power-set of $A$.
\item $A \backslash B$ denotes the set $\{x$ $|$ $ x \in A \e x \notin B\}$.
%\item We use $A\approx B$ to say that there is a function $f$ such that $f$ is an one-one function from $A$ onto $B$. To point out that $f$ is a function with those properties we write $f: A\approx B$.
\item We write $A \subseteq_{fin} B$ to say that $A \subseteq B$ and $A$ is a finite set.
\item If $f$ is a function, we write $Dom(f)$, $Rng(f)$ and $Field(f)$ to denote the sets $\{x$ $|$ $ \ex y ((x,y) \in f)\}$, $\{y$ $|$ $ \ex x ((x,y) \in f)\}$  and $Dom(f)\cup Rng(f)$, respectively. We also write $f \upharpoonleft A$ and $f[A]$ to denote the sets $\{(x,y)$ $|$ $ x \in A \e (x,y) \in f\}$
and $\{f(x)$ $|$ $ x \in Dom(f) \cap A\}$, respectively.
\item If $f$ and $g$ are functions, $f\circ g$ denotes the composition of functions, i.e., $f\circ g$ denotes the set $\{(x,z)$ $|$ $ \ex y ((x,y) \in g \e (y,z) \in f)\}$. And if $f$ is an injective function, we write $f^{-1}$ to denote the function $\{(y,x)$ $|$ $ (x,y) \in f \}$.

\end{itemize}

\chapter{Preliminaries}
\section{Syntactical considerations}

\begin{defn}
A language $\Li$ is a set of symbols. Throughout this dissertation we are going to work only with relational laguages; in some specific moments we will add constants to the language, but we will be explicit when we are doing so. We use $P, Q, P\p, Q\p, \dots$ to denote relation symbols. It is assumed that each relation symbol $P$ of $\Li$ is an $n$-ary relation symbol for $n \in \omega$. We call a $0$-ary relation symbol a \textit{propositional letter}, and we use $p,q,p\p,q\p, \dots$ to denote propositional letters (also called \textit{propositional variables}). 

\qquad We use $\Li, \Li^{\prime}, \Li^{\prime \prime}, \dots$ as variables for languages. If $\Li \subseteq \Li^{\prime}$, we say that $\Li^{\prime}$ is an \textit{expansion} of $\Li$, and that $\Li$ is a \textit{reduction} of $\Li^{\prime}$.
\end{defn}

\begin{defn}
Together with $\Li$ we define the following \textit{logical symbols}:

\begin{itemize} 
\item $x_{0}, x_{1}, x_{2}, \dots$ (\textit{variables});
\item $\nao, \ou$ (\textit{not, or});
\item $\ex$ (\textit{there exists});
\item $\Diamond$ (\textit{possibility symbol});
\item $=$ (\textit{equality symbol});
\item $),($ (\textit{parentheses}).
\end{itemize}
\qquad We use $x, y, z, \dots$ as syntactical variables for variables.
\end{defn}

\begin{defn}
The set $\FLi$ of formulas of $\Li$ is defined by the following rules:  
\begin{itemize} 
\item If $x, y$ are variables, then $x =y$ is a formula of $\Li$.
\item If $x_1, \dots, x_n$ $(n \geq 0)$  are variables and $P$ is an $n$-ary relation symbol of $\Li$, then $Px_1 \dots x_n$ is a formula of $\Li$.
\item If $\varphi$ is a formula of $\Li$, then $\nao \varphi$ is a formula of $\Li$.
\item If $\varphi$ and $\psi$ are formulas of $\Li$, then $(\varphi \ou \psi)$ is a formula of $\Li$.
\item If $\varphi$ is a formula of $\Li$, then $\Diamond \varphi$ is a formula of $\Li$.
\item If $\varphi$ is a formula of $\Li$ and $x$ a variable, then $\ex x \varphi$ is a formula of $\Li$.
\end{itemize}
\qquad We assume the standard syntactical notions of \textit{atomic formula}, \textit{free variable}, \textit{bound variable}, \textit{sentence}, \textit{formula complexity} and \textit{proof (definition) by induction on formulas}. We are going to employ the usual abbreviations: 

\begin{center}	
$(\varphi \e \psi) := \nao (\nao \varphi \ou \nao \psi)$\\
$(\varphi \impli \psi) := (\nao \varphi \ou \psi)$\\
$(\varphi \see \psi) := (\varphi \impli \psi)\e (\psi \impli \varphi)$\\
$\todo x \varphi := \nao\ex x \nao \varphi$\\
$\Box \varphi := \nao\Diamond \nao \varphi$\\
\end{center}

\qquad We write $\varphi(x_{1}, \dots, x_{n})$ to denote that the free variables of $\varphi$ are among $\{x_{1}, \dots, x_{n}\}$. Where $y_{1}, \dots, y_{n}$ are variables, we write $\varphi(y_{1}/x_{1}, \dots, y_{n}/x_{n})$ to denote the formula obtained by substitution of $y_{1}, \dots, y_{n}$ for all the free occurrences of $x_{1}, \dots, x_{n}$ in $\varphi$,  respectively. When it is clear from the context which variables are free in $\varphi$ we simply write $\varphi(y_{1}, \dots, y_{n})$ instead of $\varphi(y_{1}/x_{1}, \dots, y_{n}/x_{n})$. We use $\vec{x},\vec{y}, \dots$ for sequence of variables; and we write $\todo \vec{x} \varphi(\vec{x})$ in the place of $\todo x_1 \dots \todo x_n\varphi(x_1, \dots ,x_n)$. 
\end{defn}


\section{Models: basic notions}

\begin{defn}
A \textit{frame} is a tuple $\bl\W,\R\br$ in which:

\begin{itemize}
\item $\W \neq \vazio$.
\item $\R \subseteq \W\times \W$.
\end{itemize}
\end{defn}


\begin{defn}
A \textit{skeleton}\footnote{Sometimes called \textit{augmented frame}.} is a quadruple $\Frame$ in which $\bl\W,\R\br$ is a frame and: 
\begin{itemize} 
\item $\D \neq \vazio$.
\item $\barD: \W \impli \Pa(\D)$, and for every $w$ of $\W$, $\barD(w) \neq \vazio$. 
\item $\D = \bigcup_{w \in \W} \barD_{w}$.
\end{itemize}

\qquad The intuition behind the notion of skeleton is the same as in \cite{Kripke63}: $\W$ is the set of all `possible worlds'; $\R$ is the accessibility relation between worlds; $\barD$ is a function which gives to each world a domain of individuals, and $\D$ is the set of all possibles individuals.

\qquad We use $w, v, u, w^{\prime}, w_0, w_1, \dots$ as variables for worlds. From now on we write $\barD_{w}$ instead of $\barD(w)$. In the cases where $\barD$ is a constant function we write $\bl\W,\R, \D\br$ instead of $\Frame$.
\end{defn}

\begin{defn}
A \textit{(modal) model} for $\Li$ is a quintuple $\M = \strucA$ in which $\Frame$ is a skeleton and $\I$ is an \textit{interpretation function}, i.e., a function assigning to each $n$-ary relational symbol $P$ of $\Li$ and each possible world $w$ an $n$-ary relation $\I(P,w)$ on $\D$. 

\qquad We use $\M, \N, \M\p, \dots$ as variables for models. 
\end{defn}

\begin{defn}
Let $\Li$ and $\Li^{\prime}$ be languages such that $\Li^{\prime} \subseteq \Li$, $\M = \strucA$ be a model for $\Li$ and $\M^{\prime} = \bl\W\p,\R\p, \D\p, \barD\p,\I\p\br$ be a model for $\Li^{\prime}$. We call $\M^{\prime}$ a \textit{reduct} of $\M$ (and $\M$ an \textit{expansion} for $\M^{\prime}$) iff $\W =\W^{\prime}$, $\R =\R^{\prime}$, $\D =\D^{\prime}$, $\barD =\barD^{\prime}$, and $\I$ and $\I^{\prime}$ agree on the symbols of $\Li^{\prime}$. We write $\M^{\prime} = \M|_{\Li^{\prime}}$.
\end{defn}

\begin{defn}
A \textit{valuation} in a model $\M = \strucA$ is a function $h$ from the set of variables to $\D$.\footnote{Although is more natural to use $v$ to denote a valuation, it is easy to get lost in the proofs when we use $v$ for valuations and $w$ and $u$ for worlds.} We say that $h\p$ is an $x$\textit{-variant} of $h$ if the two valuations agree on all variables except possibly $x$. Similarly, we say that a valuation $h\p$ is an $x$\textit{-variant of $h$ at $w$} if $h\p$ is an $x$-variant of $h$ and $h\p(x) \in \barD_w$.
\end{defn}

\begin{defn}
Let $\M = \strucA$ be a model for $\Li$, $\varphi$ a formula of $\Li$, $h$ a valuation in $\M$ and $w \in \W$. The notion \textit{$\varphi$ is true at world $w$ of $\M$ with respect to valuation $h$}, in symbols $\M,w \vSs \varphi$, is defined recursively as follows: 

\begin{itemize} 
\item[] $\M,w \vSs x = y$ iff $h(x) = h(y)$. 
\item[] $\M,w \vSs Px_{1} \dots x_{n}$ iff $\bl h(x_{1}), \dots, h(x_{n})\br \in \I(P,w)$. 
\item[] $\M,w \vSs \nao \psi$ iff $\M,w \nvSs \psi$.
\item[] $\M,w \vSs \psi \ou \theta$ iff $\M,w \vSs \psi$ or $\M,w \vSs \theta$.
\item[] $\M,w \vSs \Diamond \psi$ iff there is a $w\p \in \W$ such that $w\R w\p$ and $\M,w\p \vSs \psi$.
\item[] $\M,w \vSs \ex x \psi$ iff there is an $x$-variant $h\p$ of $h$ at $w$ such that $\M,w \vSp \psi$.
\end{itemize}

\qquad This definition enables us to speak of the truth of a formula at a world in a model without mentioning the valuation. We write $\M,w \models \varphi$ if for every valuation $h$, $\M,w \vSs \varphi$; and when that is the case we say that \textit{$\varphi$ is true in $\M$ at $w$}. We write $\M \models \varphi$ if for every world $w$ of $\M$, $\M,w \models \varphi$. And we say that a formula $\varphi$ is\textit{ valid in a class of models}, if for every model $\M$ of this class, $\M \models \varphi$.

\qquad Let $\Gamma$ be a set of formulas of $\Li$ (we also call $\Gamma$ a \textit{theory}); then $\M,w \models \Gamma$ if for every $\varphi \in \Gamma$, $\M,w \models \varphi$. In this case, we say that the pair  $\M,w$ is \textit{a model for} $\Gamma$. Two formulas $\varphi$ and $\psi$ are \textit{equivalent} if for every model $\M$, $\M \models \varphi$ iff $\M \models \psi$.

% Similarly, $\M \models \Gamma$ if for every $\varphi \in \Gamma$, $\M \models \varphi$.

%, or that \textit{$\M, w$ is a model for $\varphi$}. Similarly, if $\M,w \models T$, we say that \textit{$\M,w$ is a model for $T$}.
\end{defn}

\qquad There are some basic propositions about the relation $\models$. Since their proofs are straightforward and they can be found in many different textbooks, we are going to state these propositions without proof.

\begin{pro}
Suppose that $\M$ and $\M\p$ are models for $\Li$ and $\Li^{\prime}$, respectively; that $\Li\subseteq\Li\p$; and that $\M$ is the reduct of $\M\p$ to $\Li$. Then for every world $w$ of $\M$, for every valuation $h$ in $\M$, if $\varphi$ is a formula of $\Li$ then:

\begin{center}
 $\M,w \vSs \varphi$ iff  $\M\p,w \vSs \varphi$.
\end{center} 
\end{pro}

\begin{pro}
Let $\M$ be a model for $\Li$, $w$ a world of $\M$, $h_{1}$ and $h_{2}$ valuations in $\M$ and $\varphi$ a formula of $\Li$. If $h_{1}$ and $h_{2}$ agree on all the free variables of $\varphi$, then

\begin{center}
 $\M,w \models_{h_{1}} \varphi$ iff  $\M,w \models_{h_{2}} \varphi$.
\end{center} 
\end{pro}






\begin{defn}
Let $\M = \strucA$ be a model for $\Li$:

\begin{itemize} 
\item $\M$ is an \textit{S5-model} iff $\R$ is an equivalence relation.
\item $\M$ is an \textit{universal model} iff $\R = \W\times \W$.

\item $\M$ is a \textit{constant domain model} iff for every $w,v \in \W$, $\barD_{w} = \barD_{v}$.
\item $\M$ is a \textit{monotonic model} iff for every $w,v \in \W$, if $w\R v$, then $\barD_{w} \subseteq \barD_{v}$.  
\item $\M$ is an \textit{anti-monotonic model} iff for every $w,v \in \W$, if $w\R v$, then $\barD_{v} \subseteq \barD_{w}$.   
\item $\M$ is a \textit{locally constant domain model} iff for every $w,v \in \W$, if $w\R v$, then $\barD_{w} = \barD_{v}$. 
\end{itemize}
\end{defn}

\qquad Very often, in different books and papers on first-order modal logic, there is the mentioning of the `Barcan Formula'. The following explains the connection between locally constant domain models and this formula.  

\begin{defn}
Let $\M = \strucA$ be a model for $\Li$:

\begin{itemize} 
\item We say that \textit{$\M$ satisfies the Barcan Formula} iff for every $\varphi \in Fml(\Li)$ of the form $\todo x \Box \psi \impli \Box \todo x\psi$, we have that $\M \models \varphi$.
\item We say that \textit{$\M$ satisfies the Converse Barcan Formula} iff for every $\varphi \in Fml(\Li)$ of the form $\Box \todo x\psi \impli \todo x \Box \psi$, we have that $\M \models \varphi$.
\end{itemize}
\end{defn}

\qquad By well-know equivalences of first-order modal logic, we have: 

\begin{itemize} 
\item[] $\M$ satisfies the Barcan Formula iff for every $\varphi \in Fml(\Li)$ of the form $\Diamond \ex x\psi \impli \ex x \Diamond \psi$, we have that $\M \models \varphi$.
\item[] $\M$ satisfies the Converse Barcan Formula iff for every $\varphi \in Fml(\Li)$ of the form $\ex x \Diamond \psi \impli \Diamond \ex x\psi$, we have that $\M \models \varphi$.
\end{itemize}

\begin{pro}
Let $\M = \strucA$ be a model for $\Li$:
\begin{enumerate}[(a)]
\item $\M$ is an anti-monotonic model iff $\M$ satisfies the Barcan Formula.
\item $\M$ is a monotonic model iff $\M$ satisfies the Converse Barcan Formula.
\item $\M$ is a locally constant domain model iff $\M$  satisfies the Barcan Formula and its converse.
\end{enumerate}
\end{pro}

\begin{proof}
(a) ($\Rightarrow$) Let $\varphi \in Fml(\Li)$ be a formula of the form $\todo x \Box \psi \impli \Box \todo x\psi$, $w \in \W$ and $h$ a valuation. If $\M,w \vSs \todo x \Box \psi$, then for every $x$-variant $h\p$ of $h$ at $w$ $\M,w \vSp \Box \psi$. Let $v$ be a member of $\W$ such that $w\R v$. By hypothesis, $\barD_{v} \subseteq \barD_{w}$, so every $x$-variant $h\p$ of $h$ at $v$ is an $x$-variant $h\p$ of $h$ at $w$, thus $\M,v \vSs \todo x\psi$. Since $v$ was arbitrarily chosen, $\M,w \vSs \Box \todo x \psi$, and hence $\M,w \vSs \varphi$.

\qquad ($\Leftarrow$) Suppose that $\M$ satisfies the Barcan Formula and 
$\M$ is not an anti-monotonic model; then there are $w,v \in \W$ such that $w\R v$ and $\barD_{v} \nsubseteq \barD_{w}$. Hence, there is an $a \in \D$ such that $a \in \barD_{v}$ and $a \notin \barD_{w}$. Then for a valuation $h$ such that $h(y) = a$, $\M,v \vSs \ex x (x =y)$; and, since $w\R v$, $\M,w \vSs \Diamond \ex x (x =y)$.  By hypothesis, $\M,w \models \Diamond \ex x (x=y) \impli \ex x \Diamond (x =y)$. So, in particular, $\M,w \vSs \Diamond \ex x (x=y) \impli \ex x \Diamond (x =y)$; hence, $\M,w \vSs \ex x \Diamond (x =y)$. Then, there is an $x$-variant $h\p$ of $h$ at $w$ such that $\M,w \vSp \Diamond (x =y)$; so there is a $w\p \in \W$ such that $w\R w\p$ and $\M,w\p \vSp (x=y)$, hence $h\p(x) =h\p(y)$. Since $h\p(x) \in \barD_{w}$, $a \in \barD_{w}$; a contradiction. Therefore, if $\M$ satisfies the Barcan Formula, then $\M$ is an anti-monotonic model.       

\qquad (b) ($\Rightarrow$) Let $\varphi \in Fml(\Li)$ be a formula of the form $\Box \todo x\psi \impli \todo x \Box \psi$, $w \in \W$ and $h$ a valuation. If $\M,w \vSs \Box \todo x\psi$, then let $v$ be a member of $\W$ such that $w\R v$; so $\M,v \vSs \todo x\psi$. Then, for every  $x$-variant $h\p$ of $h$ at $v$, $\M,v \vSp \psi$. By hypothesis, $\barD_{w} \subseteq \barD_{v}$; therefore every $x$-variant $h\p$ of $h$ at $w$ is an $x$-variant $h\p$ of $h$ at $v$, thus $\M,v \vSp \psi$ for every $x$-variant $h\p$ of $h$ at $w$. Since $v$ was arbitrarily chosen, $\M,w \vSs \Box \psi$ for every $x$-variant $h\p$ of $h$ at $w$. So $\M,w \vSs \todo x \Box \psi$ and hence $\M,w \vSs \varphi$.   

\qquad ($\Leftarrow$) Suppose that $\M$ satisfies the Converse Barcan Formula and $\M$ is not a monotonic model; then there are $w,v \in \W$ such that $w\R v$ and $\barD_{w} \nsubseteq \barD_{v}$. Hence, there is an $a \in \D$ such that $a \in \barD_{w}$ and $a \notin \barD_{v}$. So for a valuation $h$ such that $h(x)=a$, $\M,v \vSs \todo y (y \neq x)$, and since $w\R v$, $\M,w \vSs \Diamond \todo y (y\neq x)$ and so $\M,w \models \ex x \Diamond \todo y (y\neq x)$. By hypothesis, $\M,w\models \ex x \Diamond \todo y (y\neq x) \impli \Diamond \ex x \todo y (y\neq x)$. So,  $\M,w\models \Diamond \ex x \todo y (y\neq x)$. Thus there is a $w\p \in \W$ such that $w\R w\p$ and $\M,w\p \models \ex x \todo y (y\neq x)$; this clearly implies a contradiction. Therefore, if $\M$ satisfies the Converse Barcan Formula, then $\M$ is a monotonic model.   

\qquad (c) The result follows directly from (a) and (b). 
\end{proof}

\qquad Strictly speaking, both the Barcan Formula and the Converse Barcan Formula are not formulas, they are formula schemes. So it is natural to ask if there is a formula which has the same `expressive power' as the Barcan Formula and the Converse Barcan Formula. In fact, dealing with S5-models we can find this formula.

\begin{pro}
Let $\M = \strucA$ be an S5-model for $\Li$, then:
\begin{center}
$\M \models \Box \todo x \Box \ex y (y=x)$ iff $\M$ satisfies the Barcan Formula and its converse.
\end{center}
\end{pro}


\begin{proof}
($\Rightarrow$) Let $w$ and $v$ be members of $\W$ such that $w \R v$. If $a \in \barD_{w}$, then since $\M,w \models \Box\todo x \Box \ex y (y=x)$ and $w\R w$, we have that $\M,w \models \todo x \Box \ex y (y=x)$. In particular, for a valuation $h$ such that $h(x) = a$, $\M,w \vSs \Box \ex y (y=x)$. Then, $\M,v \vSs \ex y (y=x)$. So there is an $x$-variant $h\p$ of $h$ at $v$ such that $\M,v \vSp y=x$, thus $h\p (y) = h\p (x)$ and so $a \in \barD_{v}$. Hence, $\barD_{w} \subseteq \barD_{v}$. We can prove that $\barD_{v} \subseteq \barD_{w}$ in a similar way. Therefore, $\M$  is a locally constant domain model; by Proposition 3, $\M$ satisfies the Barcan Formula and its converse.   

\qquad ($\Leftarrow$) Suppose that $\M$ satisfies the Barcan Formula and its converse and there is a $w \in \W$ such that $\M,w \not\models \Box \todo x \Box \ex y (y=x)$. So, for some valuation $h$, $\M,w \nvSs \Box \todo x \Box \ex y (y=x)$. By Proposition 3, $\M$  is a locally constant domain model; and by equivalences of first-order modal logic, $\M,w \vSs \Diamond \ex x \Diamond \todo y (y \neq x)$. Hence there is a $v \in \W$ such that $w\R v$ and $\M,v \vSs \ex x \Diamond \todo y (y \neq x)$. So there is an $x$-variant $h\p$ of $h$ at $v$ such that $\M,v \vSp \Diamond \todo y (y \neq x)$. Then there is a $w\p \in \W$ such that $v\R w\p$ and $\M,w\p \vSp \todo y (y \neq x)$. Therefore, $h\p(x) \in \barD_{v}\backslash\barD_{w\p}$;  contradicting the assumption that $\M$  is a locally constant domain model.  
\end{proof}

\section{First-order S5: two versions}

\qquad Before we advance, we need to address some technical details concerning S5-models. In order to save time we are going to skip the proofs of the propositions in this section.  

\qquad First, since $\R$ is an equivalence relation in an S5-model, all the different notions of monotonic, anti-monotonic and locally constant domain model become equivalent when we work with an S5-model. Therefore, we shall only distinguish between locally constant domain models and \textit{varying domain models} (models with no restriction on the domains). 

\qquad Second, the distinction between constant domain models and locally constant domain models can be dropped. Of course, as mathematical structures constant domain models and locally constant domain models are very different objects. But from the point of view of modal formulas they are the same. The following proposition states this fact more clearly:

\begin{pro}
Let $\varphi$ be a formula of $\Li$. $\varphi$ is valid in the class of constant domain models for $\Li$ iff $\varphi$ is valid in the class of locally constant domain models for $\Li$.
\end{pro}


\qquad Third, sometimes both for technical and theoretical reasons it is more useful to deal with universal models instead of S5-models. And as before, although they are different mathematical structures, from the point of view of the valid formulas we can take them as the same:

\begin{pro}
Let $\varphi$ be a formula of $\Li$. $\varphi$ is valid in the class of universal models for $\Li$ iff $\varphi$ is valid in the class of S5-models for $\Li$. 
\end{pro}


\qquad We can now define the two main kinds of models that we are going to work with. Propositions 5 and 6 serve to show the non-arbitrariness of the following definition and to connect it with the results of the previous section. 

\begin{defn}
For a fixed language $\Li$ we say that:
\begin{itemize}

\item a \textit{model for first-order S5 with constant domains}, denoted FOS5-model, is a universal and constant domain model.

\item a \textit{model for first-order S5 with varying domains}, denoted FOS5V-model, is a universal and varying domain model.
	
\end{itemize}
	
	
\end{defn}

\qquad The following definitions apply both to FOS5 and FOS5V models; to avoid duplication of definitions we use L as a variable for FOS5 and FOS5V.  From now on, when dealing with FOS5V-models we omit the accessibility relation, and when working with FOS5-models we omit the $\barD$ function too.  

\pagebreak
\begin{defn}
Let $\Li$ be a language and let $\{\varphi\}, \{\psi\}$ and $\Gamma$ be sets of sentences of $\Li$:

\begin{itemize} 
\item $\varphi$ is \textit{L-valid}, in symbols $\vSL \varphi$, iff $\varphi$ is valid in the class of L-models. We say that $\varphi$ is \textit{L-satisfiable} iff there is an L-model $\M$ and a world $w$ of $\M$ such that $\M,w \models \varphi$. And we say that $\varphi$ is \textit{L-unsatisfiable} iff $\nao \varphi$ is L-valid.   
\item $\varphi$ is a \textit{consequence of $\Gamma$ in L}, in symbols $\Gamma \vSL \varphi$, iff for every pair $\M,w$, if $\M,w$ is an L-model for $\Gamma$, then $\M,w \models \varphi$. Instead of $\{\psi\} \vSL \varphi$ we write $\psi \vSL \varphi$.
\end{itemize}
\end{defn}

\qquad For example, from propositional modal logic it is well-known that:

\begin{center}
$\vSL \Box (\varphi \impli \psi) \impli (\Box \varphi \impli \Box \psi)$\\
$\vSL \Box \varphi \impli \varphi$\\
$\vSL \Box \varphi \impli \Box \Box \varphi$\\
$\vSL \nao \Box \varphi \impli \Box \nao \Box \varphi$\\
\end{center}


\qquad And using what we have seen so far, we have: 

\begin{center}
$ \vS \Box \todo x \Box \ex y (y=x) \impli (\Box \todo x Px \see  \todo x \Box Px)$\\
$\vSB  \Box \todo x Px \see  \todo x \Box Px$\\
\end{center}

\qquad These last two examples are just instances of a more general fact that is an immediate consequence of Propositions 3 and 4.

\begin{pro}
A sentence $\varphi$ of $\Li$ is FOS5-valid iff $\Box \todo x \Box \ex y (y=x) \impli \varphi$ is FOS5V-valid. 
\end{pro}

Now we have all the ingredients to present a notion of logic.

\begin{defn}
The logic L is a tuple $\bl Lan, \vSL\br$ where $Lan$ is a function which associates to every language $\Li$ a set $sen(\Li)$, the set of sentences of $\Li$; and $\vSL$ is the relation as defined above.
\end{defn}

\qquad A last basic topic worth noticing is that we can define an unary relation symbol $E$ such that $Ex$ expresses that the individual denoted by $x$ exists in the world in question. The definition of this relation, often called \textit{existence predicate},  is:

\begin{center}
$Ex:= \ex y (y=x) $
\end{center}

\qquad Obviously, $\M,w \vSs Ex$ iff $h(x) \in \barD_{w}$. The following proposition states some useful facts about the relation $\models$ and $Ex$. 

\begin{pro}
For a formula $\varphi$ of $\Li$ such that $fv(\varphi) = \{x_1, \dots, x_n\}$, let $E\vec{x}$ be an abbreviation of $Ex_{1} \e \dots \e Ex_{n}$, then:
\begin{itemize}
\item $\vS \todo \vec{x}\varphi$ iff $\vS (E\vec{x} \impli \varphi)$.
\item $\vSB \todo \vec{x}\varphi$ iff $\vSB \varphi$.
\item If $\vS\varphi$, then $\vS \todo \vec{x}\varphi$.
\end{itemize}
\end{pro}




\chapter{Interpolation and Definability}
\qquad This chapter is completely based on the paper \cite{Fine79} by Kit Fine. Only the last section is based on other material, the already mentioned review by Saul Kripke \cite{Kripke83}.



\section{Models: isomorphism}


\begin{defn}
Let $\M = \strucAS$ be a model for $\Li$ and $w \in \W$.

\begin{itemize} 
\item \textit{The external model} of $\M$ at $w$ is the triple $ \M_{w} = \bl\D, \barD_{w}, \I_{w}\br$ where $\I_{w}$ is a function on $\Li$ such that $\I_{w}(P) = \{\bl a_{1}, \dots,a_{n} \br \in \D^{n}$ $|$ $ \bl a_{1}, \dots,a_{n} \br \in \I(P,w)\}$, for every $n$-ary relation symbol $P\in \Li$.  
\item \textit{The internal model} of $\M$ at $w$ is the tuple $ \bar{\M}_{w} = \bl\barD_{w}, \bar{\I}_{w}\br$ where $\bar{\I}_{w}$ is a function on $\Li$ such that $\bar{\I}_{w}(P) = \{ \bl \textbf{}a_{1}, \dots,a_{n} \br \in \barD_{w}^{n}$ $|$ $ \bl a_{1}, \dots,a_{n} \br \in \I(P,w)\}$, for every $n$-ary relation symbol $P \in \Li$.
\end{itemize}
\end{defn}

\begin{defn}
Let $\M$, $\M_{w}$ and $\bar{\M}_{w}$ be as in the previous definition. We can easily define a notion of isomorphism for models of the form $\bar{\M}_{w}$ and $\M_{w}$. For the former, the notion is the same as in the classical case. For the latter, let $\N = \strucBS$ be a model for $\Li$, $v \in \V$ and $\N_{v} = (\B, \barB_{v}, \J_{v})$. Let $\sigma$ be an one-one function from $\D$ onto $\B$; we say that $\sigma$ is an isomorphism between $\M_{w}$ and $\N_{v}$, in symbols $\sigma: \M_{w} \iso \N_{v}$,  iff:   


\begin{itemize} 
\item for every $a_{1}, \dots, a_{n} \in \D$, for every $n$-ary relation symbol $P \in \Li$, $ \bl a_{1}, \dots, a_{n} \br \in \I_{w}(P)$ iff $ \bl \sigma(a_{1}), \dots, \sigma(a_{n}) \br \in \J_{v}(P)$.
\item $\sigma[\barD_{w}] = \barB_{v}$.
\end{itemize}
\end{defn}


\begin{defn}
Let $\M = \strucAS$ and $\N = \strucBS$ be models for $\Li$. We say that $\sigma$ is an isomorphism from $\M$ onto $\N$, in symbols $\sigma: \M \iso \N$, iff $\sigma$ is an one-one function from $\D$ onto $\B$ such that:
\begin{enumerate}[(i)]
\item For every $w \in \W$ there is a $v \in \V$ such that $\sigma: \M_{w} \iso \N_{v}$.
\item For every $v \in \V$ there is a $w \in \W$ such that $\sigma: \M_{w} \iso \N_{v}$.
\end{enumerate}
\end{defn}

\qquad Let $\M$ and $\N$ be models for $\Li$, and let $\sigma$ be a function from $\D$ to $\B$. If $h$ is a valuation in $\M$ we write $h^{\sigma}$ to denote the valuation $\sigma \circ h$ in $\N$.

\begin{lema}
Let $\M = \strucAS$ and $\N = \strucBS$ be models for $\Li$, $w \in \W$, $v \in \V$ and $\sigma: \D \impli \B$ such that $\sigma: \M \iso \N$ and $\sigma: \M_{w} \iso \N_{v}$ . Then for every valuation $h$ and every formula $\varphi$ of $\Li$:

\begin{center}
$\M,w \vSs \varphi$ iff $\N,v \models_{h^{\sigma}} \varphi$ 
\end{center}
  
\end{lema}


\begin{proof}
Induction on $\varphi$. \\




($\varphi$ is $x=y$)\\

$\M,w \vSs x = y$ \\
iff $h(x) = h(y)$\\
iff, since $\sigma$ is injective, $\sigma(h(x)) = \sigma(h(y))$\\
iff $h^{\sigma}(x) = h^{\sigma}(y)$\\
iff  $\N,v \models_{h^{\sigma}} x = y$.\\



($\varphi$ is $Px_{1}\dots x_{n}$)\\

$\M,w \vSs Px_{1}\dots x_{n}$\\
iff $\bl h(x_{1}), \dots, h(x_{n})\br \in \I(P,w)$\\
iff $\bl h(x_{1}), \dots, h(x_{n}) \br \in \I_{w}(P)$\\
iff, by hypothesis,  $\bl \sigma(h(x_{1})), \dots, \sigma(h(x_{n})) \br \in \J_{v}(P)$\\
iff $\bl \sigma(h(x_{1})), \dots, \sigma(h(x_{n})) \br \in \J(P,v)$\\
iff $\bl h^{\sigma}(x_{1}), \dots, h^{\sigma}(x_{n}) \br \in \J(P,v)$\\
iff  $\N,v \models_{h^{\sigma}} Px_{1}\dots x_{n}$.\\

\qquad If $\varphi$ is $\nao \psi$ or $\psi \ou \theta$, then the result follows from the induction hypothesis.\\



($\varphi$ is $\Diamond \psi$)\\

\qquad If $\M,w \vSs \Diamond \psi$, then there is a $w\p \in \W$ such that $\M,w\p \vSs \psi$. Since $\sigma: \M \iso \N$, then, by condition (i) of Definition 17, there is a $v\p \in \V$ such that $\sigma: \M_{w^{\prime}} \iso \N_{v^{\prime}}$. By induction hypothesis,

\begin{center}
$\M,w^{\prime} \vSs \psi$ iff $\N,v^{\prime} \models_{h^{\sigma}} \psi$ 
\end{center}

\qquad So, $\N,v^{\prime} \models_{h^{\sigma}} \psi$, and hence $\N,v \models_{h^{\sigma}} \Diamond\psi$. The converse implication follows from the condition (ii) of Definition 17 and the induction hypothesis.\\


($\varphi$ is $\ex x \psi$)\\

\qquad On the one hand, if $\M,w \vSs \ex x \psi$, then for an $x$-variant $h\p$ of $h$ at $w$, $\M,w \models_{h\p} \psi$. By induction hypothesis, $\N,v \models_{{h\p}^{\sigma}}  \psi$. Since $h\p(x) \in \barD_{w}$ and $\sigma[\barD_{w}] = \barB_{v}$, then ${h\p}^{\sigma}(x)\in \barB_{v}$. So, ${h\p}^{\sigma}$ is an $x$-variant of $h^{\sigma}$ at $v$. Therefore  $\N,v \models_{h^{\sigma}} \ex x\psi$.

\qquad On the other hand, if $\N,v \models_{h^{\sigma}} \ex x \psi$, then for some $x$-variant $h\p$ of $h^{\sigma}$ at $v$,  $\N,v \vSp \psi$. Since $h\p(x) \in \barB_{v}$ and $\sigma[\barD_{w}] = \barB_{v}$, there is an $a \in \barD_{w}$ such that $\sigma(a)=h\p(x)$. Let $h^{*}$ be a valuation in $\M$ such that for every variable $y$



$$
h^{*}(y) = \left\{
\begin{array}{rcl}
h(y) & \mbox{if} & y \neq x\\
a & \mbox{if} & y = x\\
\end{array}
\right.
$$


\qquad Clearly, ${h^{*}}^{\sigma} = h\p$ and $h^{*}$ is an $x$-variant of $h$ at $w$ . Since $\N,v \models_{{h^{*}}^{\sigma}} \psi$, then, by induction hypothesis, $\M,w \models_{h^{*}} \psi$, and so $\M,w \vSs \ex x \psi$. 
\end{proof}

\begin{lema}
Let $\M = \strucAS$ and $\N = \strucBS$ be models for $\Li$, $w \in \W$, $v \in \V$ and $\rho: \barD_{w} \impli \barB_{v}$ such that $\rho: \bar{\M}_{w} \iso \bar{\N}_{v}$ and for every $\rho^{\prime} \subseteq_{fin} \rho$ there is a $\sigma$ such that $\rho^{\prime} \subseteq \sigma$ and  $\sigma: \M \iso \N$. In these conditions, for every formula $\varphi$ of $\Li$ and for every valuation $h$ such that $h[fv(\varphi)] \subseteq \barD_{w}$:

\begin{center}
$\M,w \vSs \varphi$ iff $\N,v \models_{h^{\rho}} \varphi$ 
\end{center}
  
\end{lema}

\begin{proof}
Induction on $\varphi$.\\


($\varphi$ is $x=y$)\\

$\M,w \vSs x = y$ \\
iff $h(x) = h(y)$\\
iff, since $\rho$ is injective, $\rho(h(x)) = \rho(h(y))$\\
iff $h^{\rho}(x) = h^{\rho}(y)$\\
iff  $\N,v \models_{h^{\rho}} x = y$.\\

($\varphi$ is $Px_{1}\dots x_{n}$)\\

$\M,w \vSs Px_{1}\dots x_{n}$\\
iff $\bl h(x_{1}), \dots, h(x_{n})\br \in \I(P,w)$\\
iff $\bl h(x_{1}), \dots, h(x_{n}) \br \in \bar{\I_{w}}(P)$\\
iff, by hypothesis,  $\bl \rho(h(x_{1})), \dots, \rho(h(x_{n})) \br \in \bar{\J_{v}}(P)$\\
iff $\bl \rho(h(x_{1}), \dots, \rho(h(x_{n})) \br \in \J(P,v)$\\
iff $\bl h^{\rho}(x_{1}), \dots, h^{\rho}(x_{n}) \br \in \J(P,v)$\\
iff  $\N,v \models_{h^{\rho}} Px_{1}\dots x_{n}$.\\

\qquad If $\varphi$ is $\nao \psi$ or $\psi \ou \theta$, then the result follows from the induction hypothesis.\\


($\varphi$ is $\Diamond \psi$)\\

\qquad If $\M,w \vSs \Diamond \psi$, then there is a $w\p \in \W$ such that $\M,w\p \vSs \psi$. Since there is only a finite number of free variables occurring in $\psi$, if $\rho^{\prime} = \rho\upharpoonleft$ $h[fv(\varphi)]$, then $\rho^{\prime} \subseteq_{fin} \rho$. By hypothesis, there is a $\sigma$ such that $\rho^{\prime} \subseteq \sigma$ and $\sigma: \M \iso \N$. By condition (i) of Definition 17, there is a $v^{\prime} \in \V$ such that $\sigma: \M_{w^{\prime}} \iso \N_{v^{\prime}}$. So all the conditions of Lemma 1 are fulfilled; then for every valuation $h\p$ and every formula $\theta$ of $\Li$:

\begin{center}
$\M,w^{\prime} \vSp \theta$ iff $\N,v^{\prime} \models_{{h\p}^{\sigma}} \theta$.
\end{center}

In particular we have,

\begin{center}
$\M,w^{\prime} \vSs \psi$ iff $\N,v^{\prime} \models_{{h}^{\sigma}} \psi$.
\end{center}

And since $\M,w^{\prime} \vSs \psi$, we have $\N,v^{\prime} \models_{{h}^{\sigma}} \psi$. 

\qquad Now, by the definition of $\sigma$, $\sigma$ and $\rho\p$ agree on all the elements of $h[fv(\varphi)]$.\\
So, if $y \in fv(\varphi)$, then:
\begin{eqnarray*}
h^{\sigma}(y) & = & \sigma (h(y))\\
& = & \rho\p (h(y))\\
& = & \rho (h(y))\\
& = & h^{\rho}(y)\\
\end{eqnarray*}

\qquad Therefore, $h^{\sigma}$ and $h^{\rho}$ agree on all the free variables of $\psi$; then, by Proposition 2, $\N,v^{\prime} \models_{{h}^{\rho}} \psi$. And so, $\N,v \models_{{h}^{\rho}} \Diamond \psi$. The converse implication follows from the condition (ii) of Definition 17 and Lemma 1.\\

\qquad If $\varphi$ is $\ex x \psi$, then the result follows from the induction hypothesis and the fact that $\rho[\barD_{w}] = \barB_{v}$.
\end{proof}


\section{Interpolation and definability as properties}

\qquad In this section we will assume that some countable language $\Li$ is fixed.

\begin{defn}
Let L be a logic and let $\Gamma$ be a set of sentences of $\Li$. Then:
\begin{itemize}   
\item L has the \textit{Interpolation property} (or \textit{the Interpolation Theorem holds for L}) iff for any sentences $\varphi$ and $\psi$ of $\Li$,  if $\vSL \varphi \impli\psi$, then there is a formula $\theta$ such that $\vSL \varphi \impli \theta$, $\vSL \theta \impli \psi$ and the non-logical symbols that occur in $\theta$ occur both in $\varphi$ and $\psi$.    
\item Let $\Li$ be a language such that the $n$-ary relation symbol $P$ belongs to $\Li$. Let $P\p$ be a new $n$-ary relation symbol not occurring on $\Li$, $\Li^{\prime} = (\Li \backslash \{P\}) \cup \{P^{\prime}\}$ and $\Gamma^{\prime}$ be the result of replacing each occurrence of $P$ in the sentences of $\Gamma$ with $P^{\prime}$. \textit{$\Gamma$ implicitly defines $P$ in L} if $\Gamma \cup \Gamma^{\prime} \vSL \todo \vec{x}(P\vec{x} \see P\p\vec{x})$. \textit{$\Gamma$ explicitly defines $P$ in L} if $\Gamma \vSL \todo \vec{x}(P\vec{x} \see \theta)$ for some formula $\theta \in Fml(\Li \backslash \{P\})$. We say that the logic L has the \textit{Definability property} (or \textit{Beth's Definability Theorem holds for L}) iff whenever $\Gamma$ defines $P$ implicitly in L, also $\Gamma$ defines $P$ explicitly in L.
\end{itemize}
\end{defn}


 
\begin{pro}
If L has the Interpolation property then L has the Definability property.
\end{pro}

\begin{proof}
Here we shall present the proof only for FOS5V. We do that because the proof for FOS5 is very close to the proof for the classical case.



\qquad Suppose that $\Gamma$ implicitly defines $P$ in FOS5V, i.e.

\begin{center}
$\Gamma \cup \Gamma^{\prime} \vS \todo \vec{x}(P\vec{x} \see P\p\vec{x})$
\end{center}

\qquad Hence, by Proposition 8,

\begin{center}
$\Gamma \cup \Gamma\p \vS E\vec{x} \impli (P\vec{x} \see P\p\vec{x})$
\end{center}


\qquad And by propositional logic, 



\begin{center}
$\Gamma \cup \Gamma\p \vS E\vec{x} \impli (P\vec{x} \impli P\p\vec{x})$.
\end{center}




\qquad By Compactness\footnote{A proof of the Compactness Theorem for first-order modal logic can be found in \cite{Fine78}.} there is $\Gamma_0 \subseteq_{fin} \Gamma \cup \Gamma^{\prime}$ such that $\Gamma_{0} \vS E\vec{x} \impli (P\vec{x} \impli P\p\vec{x})$. Let $\varphi$ be the conjunction of all sentences of $\Gamma\cap \Gamma_0$ and $\psi$ be the conjunction of all sentences of $\Gamma^{\prime}\cap \Gamma_0$. So, 



\begin{center}
$\varphi\e \psi \vS E\vec{x} \impli (P\vec{x} \impli P\p\vec{x})$
\end{center}


\qquad It is easy to check that for every sentence $\varphi$ and $\psi$, $\varphi \vS \psi$ iff $\vS \varphi \impli\psi$. Thus, using this fact we have 

\begin{center}
$\vS \varphi\e \psi\impli (E\vec{x} \impli (P\vec{x} \impli P\p\vec{x}))$
\end{center}


\qquad By propositional logic,

\begin{center}
$\vS (E\vec{x} \e \varphi\e P\vec{x}) \impli (\psi \impli P\p\vec{x})$
\end{center}

\qquad By hypothesis, FOS5V has the Interpolation property; so there is a $\theta$ such that $\theta \in Fml(\Li\cap\Li^{\prime})$, $\vS (E\vec{x} \e \varphi\e P\vec{x}) \impli \theta$ and $\vS \theta \impli (\psi \impli P\p\vec{x})$.

\pagebreak

\qquad By propositional logic,

\begin{center}
$\vS E\vec{x} \impli (\varphi\impli (P\vec{x} \impli \theta))$\\
$\vS \psi \impli (\theta \impli P\p\vec{x})$
\end{center}

\qquad Let $\psi^{*} \in Fml(\Li)$ be the sentence obtained from $\psi$ by replacing every occurrence of $P^{\prime}$ by $P$. It can be easily seen that $\vS \psi^{*} \impli (\theta \impli P\vec{x})$.

\qquad So, by Proposition 8 and by the fact that both $\varphi$ and $\psi^{*}$ are  sentences, we have 

\begin{center}
$\vS \varphi\impli \todo \vec{x} (P\vec{x} \impli \theta)$ \\
$\vS \psi^{*} \impli \todo \vec{x} (\theta \impli P\vec{x})$
\end{center}






\qquad Now, from the choice of $\Gamma^{\prime}$, both $\varphi$ and $\psi^{*}$ are conjunctions of sentences of $\Gamma$, so we have $\Gamma \vS \varphi$ and $\Gamma \vS \psi^{*}$. Hence, 

\begin{center}
$\Gamma \vS \todo \vec{x} (P\vec{x} \impli \theta)$\\
$\Gamma \vS \todo \vec{x} (\theta \impli P\vec{x})$
\end{center}




And so,

\begin{center}
$\Gamma \vS \todo \vec{x} (P\vec{x} \see \theta)$
\end{center} 

\qquad Directly from the construction of $\Li^{\prime}$ it follows that $\theta\in Fml(\Li\backslash\{P\})$. Therefore, $\Gamma$ explicitly defines $P$ in FOS5V.
\end{proof}


\qquad We are going to focus our attention on some aspects regarding propositional letters, because in the next section both counterexamples to the Definability property for FOS5V and FOS5 use propositional letters. So it is useful to point out some details.

\qquad First, if $P$ is a propositional letter $p$, we have $\Gamma \vS \todo \vec{x} (p \see \theta)$. And this implies $\Gamma \vS p \see \todo \vec{x} \theta$. So, when working with propositional letters, we say that $\Gamma$ explicitly defines $p$ in L if $\Gamma \vSL p \see \theta$ for some \textit{sentence} $\theta \in Fml(\Li \backslash \{p\})$.

\qquad Second, let  $\M = \strucAS$ and $w \in \W$. Clearly, $\M,w \models p$ iff $\I(p,w)= \I_{w}(p) = \bar{\I}_{w}(p) = \{\bl \br \}$ and $\M,w \not\models p$ iff $\I(p,w)= \I_{w}(p) = \bar{\I}_{w}(p) = \vazio$.

\begin{defn}
Let $\Li$ be a language such that $p \in \Li$. We say that \textit{$\Gamma$ preserves $p$ in L} iff for all L-models for $\Gamma$ $\M,w$ and $\N,w$ with the same set of worlds and possible individuals and with respective interpretation functions $\I$ and $\J$, if for every $j \in (\Li \backslash \{p\})$ and every $v \in \W$ $\I(j,v) = \J(j,v)$, then $\I_{v}(p) = \J_{v}(p)$. 

\end{defn}

\begin{pro}
Let $\Li$ be a language such that $p \in \Li$ and $\Gamma \subseteq sen(\Li)$. $\Gamma$ preserves $p$ in L iff $\Gamma$ implicitly defines $p$ in L.  
\end{pro}

\begin{proof}
($\Rightarrow$) Let $\M = \strucAS$ be an L-model for $\Li \cup \Li^{\prime}$, $w \in \W$ and $\M,w \models \Gamma\cup \Gamma^{\prime}$. Let $\M|_{\Li} = \bl \W,\D,\barD,\I^{\prime}\br$ and $\M|_{\Li^{\prime}} = \bl \W,\D,\barD,\I^{\prime\prime}\br$. Hence, by Proposition 1,  $\M|_{\Li},w \models \Gamma$ and $\M|_{\Li^{\prime}},w \models \Gamma^{\prime}$. Let $\N = \bl\W,\D,\barD,\I^{*}\br$ be an L-model for $\Li$ such that for every $j \in (\Li \backslash \{p\})$ and every $v \in \W$, $\I^{*}(j,v) = \I^{\prime}(j,v)$ and $\I^{*}(p,v) = \I^{\prime\prime}(p,v)$. It is evident that $\N,w \models \Gamma$.

\qquad Now, suppose that $\M,w \not\models p\see p^{\prime}$. Then, either $\M,w \models p$ and $\M,w \not\models p^{\prime}$ or $\M,w \not\models p$ and $\M,w \models p^{\prime}$. In the first case, by Proposition 1, $\M|_{\Li},w \models p$ and $\M|_{\Li^{\prime}},w \not\models p^{\prime}$. Then, by the definition of $\I^{*}$, $\I_{w}^{\prime}(p)= \{\bl \br\}$ and $\I_{w}^{*}(p)=\vazio$. Since both $\M|_{\Li},w$ and $\N,w$ are L-models for $\Gamma$ and for every $j \in (\Li \backslash \{p\})$ and every $v \in \W$, $\I^{*}(j,v) = \I^{\prime}(j,v)$; then, by hypothesis,  $\I_{v}^{\prime}(p)= \I_{v}^{*}(p)$, in particular, $\I_{w}^{\prime}(p)= \I_{w}^{*}(p)$; a contradiction. In the second case we can deduce a contradiction in a similar way. Therefore, $\M,w \models p\see p^{\prime}$, and so $\Gamma\cup \Gamma^{\prime} \vSL p\see p^{\prime}$.    

\qquad ($\Leftarrow$) Let $\M,w$ and $\N,w$ be L-models for $\Gamma$ such that $\M = \strucAS$, $\N = \bl \W,\D,\barD,\J \br$ and for every $j \in (\Li \backslash \{p\})$ and every $v \in \W$, $\I(j,v) = \J(j,v)$. Let $\N^{\prime}$ be an $L$-model for $\Li^{\prime}$ such that $\N^{\prime} = (\W,\D,\barD,\J^{\prime})$, $\N^{\prime}|_{(\Li\backslash\{p\})} =\N|_{(\Li\backslash\{p\})}$ and for every $v \in \W$, $\J^{\prime}(p^{\prime},v) = \J(p,v)$. It is evident that $\N^{\prime},w \models \Gamma^{\prime}$.  

\qquad Let $\M\p$ be an L-model for $\Li \cup \Li^{\prime}$ such that $\M\p|_{\Li} = \M$ and $\M\p|_{\Li^{\prime}} = \N^{\prime}$. Hence, by Proposition 1,  $\M\p,w \models \Gamma$ and $\M\p,w \models \Gamma^{\prime}$, thus $\M\p,w \models \Gamma \cup \Gamma^{\prime}$, By hypothesis, $\M\p,w \models p \see p^{\prime}$, i.e.    


\begin{center}
$\M\p,w \models p$ iff $\M\p,w \models p^{\prime}$ \\
\end{center} 


\qquad By Proposition 1,
 
\begin{center}
$\M\p|_{\Li},w \models p$ iff $\M\p|_{\Li^{\prime}},w \models p^{\prime}$ 
\end{center}


\qquad By definition,

\begin{center}
$\M,w \models p$ iff $\N^{\prime},w \models p^{\prime}$ 
\end{center}


\qquad By the construction of $\N\p$,

\begin{center}
$\M,w \models p$ iff $\N,w \models p$ 
\end{center}

\qquad Hence, 

\begin{center}
$\I_{w}(p) = \J_{w}(p)$ 
\end{center}

Therefore, $\Gamma$ preserves $p$ in L.
\end{proof}

\section{Failure of Interpolation and Beth's Definability  Theorems in FOS5V}

\begin{pro}
Let $\Li =\{P,p\}$ and $\Gamma =\{\Box \todo x \Box (Px \impli p), \Diamond \ex x \Box (p \impli Px)\}$; then:
\begin{enumerate}[(a)]
\item $\Gamma$ implicitly defines $p$ in FOS5V.
\item $\Gamma$ does not explicitly define $p$ in FOS5V.
\end{enumerate}
\end{pro}

\begin{proof}
(a) In view of Proposition 10, we have to show only that $\Gamma$ preserves $p$ in FOS5V. Let $\M,w$ and $\N,w$ be FOS5V-models for $\Gamma$ such that $\M = \bl \W,\D,\barD,\I \br$, $\N = \bl \W,\D,\barD,\J \br$ and for every $w\p \in \W$, $\I(P,w\p) = \J(P,w\p)$. 

\qquad Suppose that $\bl \br \in \I_{w}(p)$; then $\M,w \models p$. Since $\M,w$ is a model for $\Gamma$, $\M,w \models \Diamond \ex x \Box (p \impli Px)$, so there is a $w^{\prime} \in \W$ such that $\M,w^{\prime} \models \ex x \Box (p \impli Px)$. Then, for some valuation $h$ such that $h(x) \in \barD_{w^{\prime}}$, we have that $\M,w^{\prime} \vSs \Box (p \impli Px)$. So, for every $w^{\prime\prime} \in \W$, $\M,w^{\prime\prime} \vSs p \impli Px$. In particular, $\M,w \vSs p \impli Px$. Since $\M,w \vSs p$, then $\M,w \vSs Px$, i.e. $\bl h(x) \br \in \I(P,w)$. So, by hypothesis, $\bl h(x) \br \in \J(P,w)$.

\qquad Now, since $\N,w$ is a model for $\Gamma$, $\N,w \vSs \Box \todo x \Box (Px \impli p)$.  So, for every $w^{\prime\prime} \in W$, $\N,w^{\prime\prime} \vSs \todo x \Box (Px \impli p)$. In particular, $\N,w^{\prime} \vSs \todo x \Box (Px \impli p)$. So for every $x$-variant $h\p$ of $h$, $\N,w^{\prime} \vSp \Box (Px \impli p)$. In particular, $\N,w^{\prime} \vSs \Box (Px \impli p)$. Hence, we have $\N,w \vSs Px \impli p$. Since $\bl h(x) \br \in \J(P,w)$; $\N,w \vSs Px$, and so $\N,w \vSs p$, i.e. $\bl \br \in \J_{w}(p)$. 

\qquad Therefore, $\I_{w}(p) \subseteq \J_{w}(p)$. We can show with a similar argument, that $\J_{w}(p) \subseteq \I_{w}(p)$. Hence, $\I_{w}(p) = \J_{w}(p)$.   

\qquad (b) Let $\M = \strucAS$ be an FOS5V-model for $\{P\}$ where:
\begin{itemize}
\item $\W = \{w,v,u\}$;
\item $\D = \{a,b\}$;
\item $\barD_{w} = \barD_{v} = \{a\}$, $\barD_{u} =\{a,b\}$;
\item $\I(P,w) =\{\bl b \br \}$, $\I(p,w)=\{\bl \br\}$ and\\
$\I(P,v) = \I(P,u) = \I(p,v) = \I(p,u) = \vazio$.
\end{itemize}

\qquad It can be easily seen that for a valuation $h$ such that $h(x) =b$, we have:
\begin{center}
$\M,w \vSs p \impli Px$\\
$\M,v \vSs p \impli Px$\\
$\M,u \vSs p \impli Px$.
\end{center}

\qquad So, $\M,u \models \ex x \Box (p \impli Px)$. Hence, $\M,w \models \Diamond \ex x\Box (p \impli Px)$ and $\M,v \models \Diamond \ex x \Box (p \impli Px)$.  

\qquad In a similar way, we have for every valuation $h$:

\begin{center}
$\M,w \vSs Px \impli p$\\
$\M,v \vSs Px \impli p$\\
$\M,u \vSs Px \impli p$.
\end{center}
\qquad Hence,

\begin{center}
$\M,w \vSs \Box (Px \impli p)$\\
$\M,v \vSs \Box (Px \impli p)$\\
$\M,u \vSs \Box (Px \impli p)$.
\end{center}
\qquad Then,

\begin{center}
$\M,w \models \todo x\Box (Px \impli p)$\\
$\M,v \models \todo x\Box (Px \impli p)$\\
$\M,u \models \todo x\Box (Px \impli p)$.
\end{center}

\qquad So, $\M,w \models \Box \todo x\Box (Px \impli p)$ and $\M,v \models \Box \todo x\Box (Px \impli p)$. Therefore, both $\M,w$ and $\M,v$ are FOS5V-models for $\Gamma$.

\qquad Now, let $\M^{\prime} = \bl \W,\D,\barD,\I^{\prime}\br$ be a model for $\{P\}$ such that $\M^{\prime} = \M|_{\{P\}}$; then,

\begin{center}
$\bar{\M}_{w}^{\prime} = \bl \{a\}, {\bar{\I\p}}_{w}\br$\\
$\bar{\M}_{v}^{\prime} = \bl \{a\}, {\bar{\I\p}}_{v} \br$.
\end{center}

\qquad It is evident that ${\bar{\I\p}}_{w}(P) = {\bar{\I\p}}_{v}(P) = \vazio$. Let $\rho$ be the identity function on $\{a\}$ and $\sigma$ the identity function on $\{a,b\}$; then clearly $\rho: \bar{\M}_{w}^{\prime} \iso \bar{\M}_{v}^{\prime}$ and for every $\rho^{\prime} \subseteq_{fin} \rho$, $\sigma$ is a function such that $\rho^{\prime} \subseteq \sigma$ and  $\sigma: \M^{\prime} \iso \M^{\prime}$. Since all the conditions of Lemma 2 have been established, it follows that for every $\varphi \in sen(\{P\})$

\begin{center}
$\M^{\prime},w \models \varphi$ iff $\M^{\prime},v \models \varphi$.
\end{center}

\qquad So, by this fact and by Proposition 1, we have

\begin{center}
(+) $\M,w \models \varphi$ iff $\M,v \models \varphi$, for every $\varphi \in sen(\{P\})$.
\end{center}

\qquad Now, suppose that $\Gamma$ explicitly defines $p$ in FOS5V. So there is a $\theta \in sen(\{P\})$ such that $\Gamma \vS p \see \theta$. Since both $\M,w$ and $\M,v$ are FOS5V-models for $\Gamma$, $\M,w \models p \see \theta$ and $\M,v \models p \see \theta$. By the definition of $\M$, $\M,w \models p$, so $\M,w \models \theta$. By (+), $\M,v \models \theta$, hence $\M,v \models p$; a contradiction. Therefore, $\Gamma$ does not explicitly define $p$ in FOS5V.  
\end{proof}


\begin{teor}
Beth's Definability Theorem and the Interpolation Theorem fail for FOS5V. 
\end{teor}

\begin{proof}
By Proposition 11, FOS5V does not have the Definability property, hence, by Proposition 9, FOS5V does not have the Interpolation property.    
\end{proof}

\section{Failure of Interpolation and Beth's Definability  Theorems in FOS5}
\qquad Before continuing, we are going to state some basic facts about permutations without proof.

\begin{defn}
Let $\tau$ be a permutation on $A$. We say that $\tau$ is an \textit{essentially finite permutation} on $A$ iff $D_{\tau}= \{a \in A$ $|$ $ \tau(a)\neq a\}$ is a finite set.
\end{defn}

\begin{pro}
If $\tau$ and $\sigma$ are essentially finite permutations on $A$, then $\sigma\circ\tau$ is an essentially finite permutation on $A$.
\end{pro}

\begin{pro}
If $\sigma$ is an essentially finite permutation on $A$, then $\sigma^{-1}$ is an essentially finite permutation on $A$.
\end{pro}

\begin{pro}
Let $\tau$ be a permutation on $A$. If $\tau^{\prime} \subseteq_{fin} \tau$, then there is a $\sigma$ such that $\tau^{\prime} \subseteq \sigma$ and $\sigma$ is an essentially finite permutation on $A$.
\end{pro}

\begin{pro}
Let $\Li =\{P,p\}$ and $\Gamma =\{p \impli \Diamond\todo x (Px \impli \Box (p \impli \nao Px)), \nao p \impli \Box \ex x (Px \e \Box (\nao p \impli Px))\}$; then:
\begin{enumerate}[(a)]
\item $\Gamma$ implicitly defines $p$ in FOS5.
\item $\Gamma$ does not explicitly define $p$ in FOS5.
\end{enumerate}
\end{pro}

\begin{proof}
(a) We proceed exactly like in Proposition 11. Let $\M,w$ and $\N,w$ be FOS5-models for $\Gamma$ such that $\M = \bl \W,\D,\I\br$, $\N = \bl \W,\D,\J\br$ and for every $v \in \W$, $\I(P,v) = \J(P,v)$. 

\qquad Suppose that $\I_{w}(p) \neq \J_{w}(p)$; then either $\I_{w}(p) = \{ \bl \br\}$ and $\J_{w}(p) = \vazio$ or 
$\I_{w}(p) = \vazio$ and $\J_{w}(p) =  \{\bl \br\}$. In the first case, since $\M, w$ is a model for $\Gamma$, $\M, w \models \Diamond\todo x (Px \impli \Box (p \impli \nao Px))$. Then, there is a $w\p \in \W$ such that $\M, w\p \models \todo x (Px \impli \Box (p \impli \nao Px))$. And since $\N, w$ is a model for $\Gamma$, then $\N, w \models  \Box \ex x (Px \e \Box (\nao p \impli Px))$. In particular, $\N, w\p \models \ex x (Px \e \Box (\nao p \impli Px))$. So, there is a valuation $h$ such that $\N, w\p \vSs  Px \e \Box (\nao p \impli Px)$. So $\bl h(x) \br \in \J(P,w\p)$ and $\N, w\p \vSs \Box (\nao p \impli Px)$. Thus, $\N, w \vSs \nao p \impli Px$. And since $\J_{w}(p) = \vazio$, $\N, w \vSs Px$, i.e. $\bl h(x) \br \in \J(P,w)$.   

\qquad  Since $\M, w\p \models \todo x (Px \impli \Box (p \impli \nao Px))$, we have that $\M, w\p \vSs Px \impli \Box (p \impli \nao Px)$. By hypothesis, $\bl h(x) \br \in \I(P,w\p)$ and $\bl h(x) \br \in \I(P,w)$. So $\M, w\p \vSs \Box (p \impli \nao Px)$. In particular, $\M, w \vSs p \impli \nao Px$. Hence, $\M, w \vSs \nao Px$, i.e. $\bl h(x) \br \notin \I(P,w)$; a contradiction. In the second case, we can deduce a contradiction in a similar manner. Therefore, $\I_{w}(p) = \J_{w}(p)$.

\qquad (b) Let $\M = \strucASB$ be an FOS5-model for $\{P\}$ where:
\begin{itemize}
\item $\W = \{ \bl k, \tau \br$ $|$ $k \in \{0,1,2\}$ and $\tau$ is an essentially finite permutation on $\Z\}$;
\item $\D = \Z$;
\item Let $N, O$ and $E$ be the sets of the natural numbers, odd natural numbers and even natural numbers, respectively. If $a \in \Z$, then:

\begin{center}
$\bl a \br \in \I(P,\bl 0,\tau \br)$ iff $a \in \tau[N]$\\
$\bl a \br \in \I(P,\bl 1,\tau \br)$ iff $a \in \tau[0]$\\
$\bl a \br \in \I(P,\bl 2,\tau \br)$ iff $a \in \tau[E]$\\
\end{center}
\end{itemize}

\qquad Let $i$ be the identity function on $\Z$; $w_0 = \bl 0,i \br$, $w_1 = \bl1,i\br$ and $w_2 = \bl2,i\br$. Let $\M_{w_0} = \bl \Z, \Z, \I_{w_0} \br$, $\M_{w_1} = \bl \Z, \Z, \I_{w_1} \br$ and $\rho$ be any permutation on $\Z$ such that $\rho[N]=O$. Then, for every $a \in \Z$:

\begin{center}
$\bl a\br  \in  \I_{w_0}(P)$ iff $\bl a \br \in \I(P, w_{0})$ iff $a \in i[N]$ iff $a \in N$ iff $\rho(a) \in O$ iff $\rho(a) \in i[O]$ iff $ \bl \rho(a) \br \in \I(P,w_{1})$ iff $\bl \rho(a) \br \in  \I_{w_1}(P)$.
\end{center}

\qquad Thus, $\rho: \M_{w_0} \iso \M_{w_1}$. Now, consider the following:

\begin{center}
(+) For every $\rho\p \subseteq_{fin}\rho$, there is a $\sigma$ such that $\rho\p \subseteq \sigma$ and $\sigma: \M \iso \M$.
\end{center}

\qquad (\textit{Proof} of (+)) If $\rho\p \subseteq_{fin}\rho$, then, by Proposition 14, there is a $\sigma$ such that $\rho\p \subseteq \sigma$ and $\sigma$ is an essentially finite permutation on $\Z$. Let $w =\bl k,\tau \br$ be a member of $\W$; by Proposition 12, $\bl k,\sigma \circ \tau \br$ is a member of $\W$. Let $M \in \{N,O,E\}$; then:

\qquad On the one hand, if $\bl a \br \in \I_{\bl k,\tau \br}(P)$, then $a \in \tau[M]$, so there is a $b \in M$ such that $a = \tau(b)$. Thus, $\sigma(a) = \sigma(\tau(b)) = \sigma \circ\tau(b)$, then $\sigma(a) \in \sigma \circ\tau[M]$, and so $\bl \sigma(a) \br \in \I_{\bl k,\sigma \circ\tau \br}(P)$.

\qquad On the other hand, if $\bl \sigma(a) \br \in \I_{\bl k,\sigma \circ\tau \br}(P)$, then $\sigma(a) \in \sigma \circ\tau[M]$, so there is a $b \in M$ such that $\sigma(a) = \sigma \circ\tau(b)$, i.e. $\sigma(a) = \sigma(\tau(b))$. Since $\sigma$ is injective, $a = \tau(b)$, thus $a \in \tau[M]$, and so $\bl a \br \in \I_{\bl k,\tau \br}(P)$.

\qquad Hence $\sigma: \M_{\bl k,\tau \br} \iso \M_{\bl k,\sigma \circ \tau)}$, i.e. the condition (i) of Definition 17 is satisfied.

\qquad Let $w =\bl k,\tau \br$ be a member of $\W$; by Propositions 12 and 13, $\bl k,\sigma^{-1} \circ \tau \br$ is a member of $\W$. Let $M \in \{N,O,E\}$, then:


\qquad On the one hand, if $\bl a \br \in \I_{\bl k,\sigma^{-1} \circ \tau \br}(P)$, then $a \in \sigma^{-1} \circ \tau[M]$, so there is a $b \in M$ such that $\sigma^{-1} \circ \tau(b) =a$, i.e. $\sigma^{-1}(\tau(b)) =a$ . Thus,  $\sigma(\sigma^{-1}(\tau(b))) =\sigma(a)$, and so $\tau(b) =\sigma(a)$, then $\sigma(a) \in \tau[M]$ and so $\bl \sigma(a) \br \in \I_{\bl k,\tau\br}(P)$. 

\qquad On the other hand, if $\bl \sigma(a) \br \in \I_{\bl k,\tau \br}(P)$, then $\sigma(a) \in \tau[M]$, so there is a $b \in M$ such that $\tau(b) = \sigma(a)$. Thus, $\sigma^{-1}(\tau(b)) = \sigma^{-1}(\sigma(a))$, i.e. $\sigma^{-1} \circ \tau(b) =a$, then $a \in \sigma^{-1} \circ \tau[M]$, and so  $\bl a \br \in \I_{\bl k,\sigma^{-1} \circ \tau \br}(P)$.

\qquad Hence $\sigma: \M_{\bl k,\sigma^{-1} \circ \tau \br} \iso \M_{\bl k,\tau \br}$, i.e. the condition (ii) of Definition 17 is satisfied.  Therefore, $\sigma: \M \iso \M$. $\Box$

\qquad Now, since $\rho: \M_{w_0} \iso \M_{w_1}$, it is evident that $\rho: \bar{\M}_{w_0} \iso \bar{\M}_{w_1}$. By this fact and by (+), all the conditions of Lemma 2 have been established, it follows that:

\begin{center}
(++) $\M,w_{0} \models \theta$ iff $\M,w_{1} \models \theta$, for every $\theta \in sen(\{P\})$.
\end{center}

\qquad Let $\M\p = \bl \W,\D,\I\p \br$ be the expansion for $\M$ to $\Li$ where $\bl \br \in \I\p(p,w)$ iff $w \neq w_{0}$. And let $\M\pp = \bl \W,\D,\I\pp)$ be the expansion for $\M$ to $\Li$ where $ \bl \br \in \I\pp(p,w)$ iff $w = w_{1}$. 

\begin{center}
(+++) $\M\p,w_{0}$ and $\M\pp,w_{1}$ are FOS5-models for $\Gamma$.
\end{center}


\qquad (\textit{Proof} of (+++)) First, it is clear that $\M\p,w_{0} \models p \impli \Diamond\todo x (Px \impli \Box (p \impli \nao Px))$. Now, let $w = \bl k,\tau \br$ be a member of $\W$, and $M \in \{N,O,E\}$. Clearly, $M$ is an infinite set and $M \subseteq N$. Suppose that $\tau[M] \subseteq \Z\backslash N$, then $M \subseteq D_{\tau}$; contradicting the assumption that $\tau$ is an essentially finite permutation on $\Z$. Therefore, there is an $a \in \Z$ such that $a \in \tau[M]$ and $a \in N$. Let $h$ be valuation such that $h(x)=a$.  Since for every $w\p \in \W \backslash \{w_0\}$, $\M\p,w\p\vSs p$ and $\M\p,w_{0}\vSs Px$, then:

\begin{center}
$\M\p,w \vSs \Box(\nao p \impli Px)$.
\end{center}

\qquad Since $h(x) \in \tau[M]$, 
\begin{center}
$\M\p,w\vSs Px \e \Box(\nao p \impli Px)$
\end{center}
and so

\begin{center}
$\M\p,w\models \ex x (Px \e \Box(\nao p \impli Px))$.
\end{center}

\qquad Since $w$ was arbitrarily chosen, $\M\p,w_{0}\models \Box \ex x (Px \e \Box(\nao p \impli Px))$, and so, $\M\p,w_{0}\models \nao p \impli \Box \ex x (Px \e \Box(\nao p \impli Px))$. Therefore, $\M\p,w_{0} \models \Gamma$.  

\qquad Second, it is clear that $\M\pp,w_{1} \models \nao p \impli \Box \ex x (Px \e \Box (\nao p \impli Px))$. Now, let $h$ be a valuation. If $h(x) \in E$, then for every $w \in \W \backslash \{w_1\}$, $\M\pp,w \not\vSs p$ and $\M\pp,w_{1} \vSs \nao Px$, then: 

\begin{center}
$\M\pp,w_{2}\vSs \Box(p \impli \nao Px)$.
\end{center}

\qquad Hence, 

\begin{center}
$\M\pp,w_{2}\models \todo x (Px \impli \Box(p \impli \nao Px))$.
\end{center}

\qquad Thus $\M\pp,w_{1}\models \Diamond \todo x (Px \impli \Box(p \impli \nao Px))$, and so $\M\pp,w_{1}\models p \impli \Diamond \todo x (Px \impli \Box(p \impli \nao Px))$. Therefore, $\M\pp,w_{1} \models \Gamma$. $\Box$ 


\qquad Now, suppose that $\Gamma$ explicitly defines $p$ in FOS5. So there is a $\theta \in sen(\{P\})$ such that $\Gamma \vSB p \see \theta$. So, by (+++), $\M\p,w_{0} \models p \see \theta$ and $\M\pp,w_{1} \models p \see \theta$. Since $\M\pp,w_{1} \models p$, then $\M\pp,w_{1} \models \theta$. Hence, by Proposition 1, $\M,w_{1} \models \theta$. By (++), $\M,w_{0} \models \theta$. So, again by Proposition 1, $\M\p,w_{0} \models \theta$. Thus $\M\p,w_{0} \models p$, a contradiction. Therefore, $\Gamma$ does not explicitly define $p$ in FOS5.
\end{proof}

\begin{teor}
Beth's Definability Theorem and the Interpolation Theorem fail for FOS5. 
\end{teor}

\begin{proof}
Direct from Propositions 9 and 15.  
\end{proof}

\section{Inner and outer quantifiers}

\qquad Fix a language $\Li$. We are going to add the new logical symbols $\Sigma$ and $\Pi$. Together with these symbols, we will define new kinds of modal formulas. For the sake of brevity, from now on we are only going to indicate the structure of the formulas, we are going to skip the full recursive definition.  


\begin{defn} (Extended formulas)
	\begin{center}
		$ \varphi :: = x =y$ $|$ $Px_1 \dots x_n$ $|$ $\nao \varphi$ $|$ $\varphi \ou \varphi$ $|$ $\ex x \varphi$ $|$  $\Box \varphi$ $|$ $\Sigma x \varphi$ $|$ $\Pi x \varphi$
	\end{center}
\end{defn}

\qquad $\ex$ and $\todo$ are called \textit{inner quantifiers}; $\Sigma$ and $\Pi$ are called \textit{outer quantifiers}. We write $\FLi^{+}$ to denote the set of all extended formulas.


\begin{defn}
	Let $\M = \strucAS$ be an FOS5V-model for $\Li$, $\varphi \in \FLi^{+}$, $h$ a valuation in $\M$ and $w \in \W$. The notion $\M,w \vSs \varphi$ is defined as before; the new clauses are:
	
	\begin{itemize} 

		\item[] $\M,w \vSs \Sigma x \psi$ iff there is an $x$-variant $h\p$ of $h$ such that $\M,w \vSp \psi$.
		
		\item[] $\M,w \vSs \Pi x \psi$ iff for every $x$-variant $h\p$ of $h$  $\M,w \vSp \psi$.
		
	\end{itemize}

\end{defn}


\qquad It can be easily seem that for every $\varphi(x) \in \FLi^{+}$:


\begin{center}
	$ \vS \Sigma x \varphi(x) \see \nao \Pi x \nao \varphi(x)$\\
	$ \vS \Pi x \varphi(x) \see \nao \Sigma x \nao \varphi(x)$
\end{center}

\begin{defn}
We say that the outer quantifiers $\Sigma$ and $\Pi$ are \textit{definable in FOS5V} iff for every $\varphi (x) \in \FLi^{+}$ there are sentences $\psi,\theta\in \FLi$ such that $\psi$ and $\theta$ have exactly the same non-logical symbols occurring in $\varphi (x)$ and 


\begin{center}
	$ \vS \Sigma x \varphi(x) \see \psi$\\
	$ \vS \Pi x \varphi(x) \see \theta$
\end{center}
\end{defn}


\begin{pro}
The outer quantifiers $\Sigma$ and $\Pi$ are not definable in FOS5V.
\end{pro}

\begin{proof}
By the equivalences stated above, is enough to show that $\Sigma$ is not definable in FOS5V.


\qquad Let $\Gamma =\{\Box \todo x \Box (Px \impli p), \Diamond \ex x \Box (p \impli Px)\}$. First, we shall show that 

\begin{center}
$\Gamma \vS p \see \Sigma x Px$
\end{center}

\qquad Let $\M,w$ be a FOS5V-model for $\Gamma$. First, suppose $\M,w \models p$. Since $\M,w \models  \Diamond \ex x \Box (p \impli Px)$, then for some $w\p \in \W$, $\M,w\p \models \ex x \Box (p \impli Px)$.  So for some valuation $h$ such that $h(x) \in \barD_{w\p}$, $\M,w\p \vSs \Box (p \impli Px)$. Hence, $\M,w \vSs p \impli Px$, so $\M,w \vSs Px$. Thus, $\M,w \models \Sigma x Px$. 

\qquad  Second, suppose $\M,w \models \Sigma x Px$. Then there is a valuation $h$ and a $w\p \in \W$ such that $\M,w \vSs Px$ and $h(x) \in \barD_{w\p}$. Since $\M,w \models \Box \todo x \Box (Px \impli p)$, then  $\M,w\p \models \todo x \Box (Px \impli p)$. So, $\M,w\p \vSs \Box (Px \impli p)$. Hence,  $\M,w \vSs Px \impli p$. Thus, $\M,w \vSs p$, i.e., $\M,w \models p$.     




\qquad Now, suppose that $\Sigma$ is definable in FOS5V. Then there is a $\psi \in sen(\{P\})$ such that

\begin{center}
	$ \vS \Sigma x Px \see \psi$
\end{center}


\qquad Hence, $\Gamma \vS p \see \psi$. Thus, $\Gamma$ explicitly defines $p$ in FOS5V. But this contradicts Proposition 11.


\end{proof}




\chapter{Justification Logic: a very short introduction}
\section{History and motivation}


\qquad Justification logic is one of those few subjects in which a historical introduction is more fruitful than a plain exposition of the syntax and the semantics of the logic.

\qquad In the debate around foundations of mathematics one of the philosophical positions that arose was Brouwer's intuitionism. Briefly, intuitionism says that the truth of a mathematical statement should be identified with the proof of that statement. Summarizing the core idea of this position in a slogan: \textit{truth means provability}. Starting from this core idea an informal semantics was created. Now, this semantics is known as Brouwer–Heyting–Kolmogorov (BHK) semantics. It gives an informal meaning to the logical connectives $\bot, \e, \ou, \impli, \nao$ in the following way:

\begin{itemize}
	\item $\bot$ is a proposition which has no proof (an absurdity, e.g. $0=1$).
	\item A proof of $\varphi \e \psi$ consist of a proof of $\varphi$ and a proof of $\psi$.
	\item A proof of $\varphi \ou \psi$ is given by exhibiting either a proof of $\varphi$ or a proof of $\psi$.
	\item A proof of $\varphi \impli \psi$ is a construction which, given a proof of $\varphi$, returns a proof of $\psi$.
	\item A proof of $\nao \varphi$ is a construction which transforms any proof of $\varphi$ into a proof of a contradiction. \footnote{By this definition, we can clearly treat $\nao \varphi$ as an abbreviation of $\varphi \impli \bot$.} 
\end{itemize}

\qquad Using this semantics we can give an informal argument to show that some formulas are intuitionistic validities (formulas like $\varphi \impli \varphi$, $\varphi \impli (\psi \impli \varphi)$ and $\bot \impli \varphi$) and show that some formulas that are classical validities are not validities by this interpretation (formulas like $\varphi \ou \nao \varphi$ and $\nao\nao\varphi \impli \varphi$). More important than to decide whether some formula is a validity or not, this semantics gives us a way to grasp the intended reasoning that \textit{intuitionistic logic} (Int) wants to capture.

\qquad The first step toward a formalization of this semantics was given by Gödel in 1933 \cite{Goedel33}. He added a new unary operator $B$ to classical logic; $B\varphi$ should be read as `$\varphi$ is provable' ($B$ stand for `beweisbar', the German word for `provable'). This new operator was added in order to express the notion of provability in classical mathematics. To describe the behavior of this operator Gödel constructed the following calculus:

\begin{itemize}
 \item[] All tautologies
 \item[] $B\varphi \impli \varphi$
 \item[] $B(\varphi \impli \psi) \impli$ $(B\varphi \impli B\psi)$
 \item[] $B\varphi \impli BB\varphi$
 \item[] (\textit{Modus Ponens}) $\teo \varphi$, $\teo \varphi\impli\psi$ $\Rightarrow$ $\teo \psi$
 \item[] (\textit{Internalization})  $\teo \varphi$ $\Rightarrow$ $\teo B\varphi$
\end{itemize}



\qquad Since this axiom system is equivalent to Lewis's S4 when we translate  $B\varphi$ by  $\Box\varphi$, we will refer to this calculus of provability in classical mathematics simply as S4.



\qquad Based on the intuitionistic notion of truth as provability, it is possible to define the following translation from formulas of intuitionistic logic to formulas of S4:
\pagebreak
\begin{itemize}
	\item $p^{B} = Bp$;
	\item $\bot^{B} = \bot$;	
	\item $(\varphi \e \psi)^{B} = (\varphi^{B} \e \psi^{B})$;
	\item $(\varphi \ou \psi)^{B} = (\varphi^{B} \ou \psi^{B})$;	
	\item $(\varphi \impli \psi)^{B} = B(\varphi^{B} \impli \psi^{B})$.
\end{itemize}

\qquad It was shown by Gödel, McKinsey and Tarski (for all the references see \cite{Artemov06}) that this translation `makes sense', i.e., that the following theorem holds:

\begin{center}
	For every formula $\varphi$, Int $\teo \varphi$ iff S4 $\teo \varphi^{B}$.
\end{center}

\qquad The next step is to give a formal interpretation of the $B$ operator. One natural interpretation is the following: fix a first-order version of Peano Arithmetic (\textbf{PA}); $B$ should be interpreted as the predicate $\ex y Proof (y,x)$ which asserts that there exits a proof (in \textbf{PA}) with Gödel number $y$ for a formula with Gödel number $x$. This predicate has the following property:


\begin{center}
	For every sentence $\varphi$ in the language of \textbf{PA}, $\textbf{PA} \teo \varphi$ iff $Proof (n,\ulcorner \varphi\urcorner)$ holds for some $n$.
\end{center} 

\qquad For simplicity, we will use $Prov(x)$ as an abbreviation of $\ex y Proof (y,x)$. Let $*$ be a bijection between the sentences of \textbf{PA} and the propositional variables. We can extend the mapping $*$ to give an arithmetical interpretation of all S4 formulas as follows:

\begin{itemize}
	\item $\bot^{*} = \bot$;
	\item $(\varphi \e \psi)^{*} = (\varphi^{*} \e \psi^{*})$;
	\item $(\varphi \ou \psi)^{*} = (\varphi^{*} \ou \psi^{*})$;	
	\item $(\varphi \impli \psi)^{*} = (\varphi^{*} \impli \psi^{*})$;
	\item $(B\varphi)^{*} = Prov(\ulcorner \varphi^{*}\urcorner)$.
\end{itemize}

\qquad On the one hand, it was straightforward how to interpret the modal formulas in the language of \textbf{PA}; on the other hand it was not clear how to give a formal interpretation of this provability calculus (S4) in \textbf{PA}. In \cite{Goedel33} Gödel pointed out that S4 does not correspond to the calculus of the predicate $Prov(x)$ in \textbf{PA}. Simply because S4 proves the formula $B(B(\bot) \impli \bot)$. Using the above translation this formula correspond to $Prov(\ulcorner Prov(\ulcorner\bot\urcorner) \impli \bot\urcorner)$. And since the following sentences are equivalent in \textbf{PA}:

\begin{center}
	$Prov(\ulcorner\bot\urcorner) \impli \bot$\\
	$\nao Prov(\ulcorner\bot\urcorner)$\\
	$Consist(\textbf{PA})$,
\end{center}
$Prov(\ulcorner Prov(\ulcorner\bot\urcorner) \impli \bot\urcorner)$ means that the consistency of \textbf{PA} is internally provable in \textbf{PA}, which contradicts Gödel's Second Incompleteness Theorem.

\qquad In a lecture in 1938 \cite{Goedel38} Gödel suggested a way to remedy this problem. Instead of using the implicit representation of proofs by the existential quantifier in the formula $\ex y Proof (y,x)$ one can use explicit variables for proofs (like $t$) in the formula $Proof (t,x)$. In these lines, Gödel proposed expanding the language of classical propositional logic with variables for proofs and adding the following ternary operator $tB(\varphi,\psi)$ which should be read as `$t$ is a derivation of $\psi$ from $\varphi$'. Using $tB(\varphi)$ as an abbreviation of $tB(\top,\varphi)$, Gödel formulated the following axiom system:

\begin{itemize}
	\item[] All tautologies
	\item[] $tB(\varphi)\impli \varphi$
	\item[] $tB(\varphi,\psi) \impli (sB(\psi,\theta) \impli f(t,s)B(\varphi,\theta))$\footnote{To understand the motivation behind this function $f$ consider the following. Suppose $t$ is a derivation of $\psi$ from $\varphi$ and $s$ is a derivation of $\theta$ from $\psi$. Then it can be easily seen that the concatenation of $t$ and $s$, $t^{\smallfrown}s$, is a derivation of $\theta$ from $\varphi$. So, if $t$ is a derivation of $\psi$ from $\varphi$ and $s$ is a derivation of $\theta$ from $\psi$, then $f(t,s)= t^{\smallfrown}s$.  }
	\item[] $tB(\varphi) \impli t\p B(tB(\varphi))$
	\item[] (\textit{Modus Ponens}) $\teo \varphi$, $\teo \varphi\impli\psi$ $\Rightarrow$ $\teo \psi$
	\item[] (\textit{Internalization})  $\teo \varphi$ $\Rightarrow$ $\teo tB(\varphi)$ (where $t$ is an derivation of $\varphi$).
\end{itemize}


\qquad Gödel just formulated this system but he did not give a proof of how this system could be used to be a bridge between Int and \textbf{PA}. Independently of Gödel's system presented in \cite{Goedel38} (the lecture was published only in 1998), Sergei Artemov (in \cite{Artemov01}) proposed the use of explicit variables and constants for proofs and some basic operations between proofs (Application `$\cdot$', Sum `$+$' and Verifier `$!$'). Instead of having $B\varphi$ (or the more modern notation of provability logic $\Box \varphi$), the non-classical formulas are of the form $t$$:$$\varphi$ (which should be read as `$t$ is a proof of $\varphi$'); where $t$ is a simple or complex term composed of proof variables or constants. With this new language Artemov stipulated the following axiom system to capture the behavior of this explicit provability:     

\begin{itemize}
	\item[] All tautologies
	\item[] $t$$:$$\varphi \impli \varphi$
	\item[] $t$$:$$(\varphi \impli \psi) \impli$ $(s$$:$$\varphi \impli$ $[t\cdot s]$$:$$\psi)$
	\item[] $t$$:$$\varphi \impli$ $!t$$:$$t$$:$$\varphi$
	\item[] $t$$:$$\varphi \impli$ $[t+s]$$:$$\varphi$ 
	\item[] $s$$:$$\varphi \impli$ $[t+s]$$:$$\varphi$
	\item[] (\textit{Modus Ponens}) $\teo \varphi$, $\teo \varphi\impli\psi$ $\Rightarrow$ $\teo \psi$
	\item[] (\textit{axiom necessitation})  $\teo c$$:$$\varphi$, where $\varphi$ is an axiom and $c$ is a justification constant.
\end{itemize}

\qquad This logic was called \textit{Logic of Proofs} (LP) and it was the first example of justification logic.


\qquad If $\varphi$ is a S4 formula, there is a mapping $r$ (called a \textit{realization}) from the occurrences of $B$'s (or boxes) into terms. The result of this mapping on $\varphi$ is denoted $\varphi^{r}$. The following theorem express the connection between S4 and LP:

\textbf{(Realization Theorem between S4 and LP)} For every $\varphi$ in the language of S4, there is a realization $r$ such that

\begin{center}
S4 $\teo \varphi$ iff LP $\teo \varphi^{r}$ 
\end{center}

\qquad There is a way to define an interpretation $*$ of the LP formulas into the sentences of \textbf{PA} (for details see \cite{Artemov01}). And with all this machinery Artemov was able to prove the following result:

\textbf{(Provability Completeness of Intuitionistic Logic)} For every $\varphi$, for every interpretation $*$, there is a realization $r$ such that

\begin{center}
Int $\teo \varphi$ iff S4 $\teo \varphi^{B}$ iff LP $\teo (\varphi^{B})^{r}$ iff \textbf{PA} $\teo ((\varphi^{B})^{r})^{*}$
\end{center}


\qquad This result shows that instead of the philosophical attitude of understanding intuitionistic logic as a reasoning different from the reasoning that classical logic wants to capture, we can interpret intuitionistic logic as provability in \textit{classical mathematics}. Thus, the primitive notions that appear in the BHK semantics (`proof' and `construction') can have a formal meaning in a classical setting.


\qquad Going beyond the specific problem of the formalization of BHK semantics, justification logic can be seen as a new tool to introduce the notion of \textit{justifications} in the well-established discussion of epistemic logic (for a more detailed discussion see \cite{Artemov08}). Instead of using modal formulas like $\Box \varphi$ to express:

\begin{center}
For a given agent, $\varphi$ is known,	
\end{center}  
we use justification formulas like $t$$:$$\varphi$ to express:

\begin{center}
For a given agent, $\varphi$ is known for the reason $t$.	
\end{center}  

\qquad Informally, we can see the terms $t, s, \dots$ as justifications and the operators $+, \cdot, !$ can be seen as means of epistemic action. In fact, this point of view enables us to see justification logic as something bigger than the logic of the explicit provability; justification logic can be seen as a logic of \textit{explicit knowledge}.

\qquad Our main interest in justification logic lies in this connection with epistemic logic. We are not going to focus on the arithmetical interpretation of this logic, instead we are going to work only with the Kripke-style semantics introduced by Melvin Fitting for this logic. But it is important to have the provability interpretation in mind because some of the choices made to formulate specific aspects of justification logic are directly motivated by the relationship with provability logic and the arithmetical interpretation. 

\section{The propositional case: language and axiom system}

\begin{defn} (Basic vocabulary)	
	\begin{itemize} 
		\item $p, q, p\p, q\p, \dots$ (\textit{propositional variables});
		\item $\impli, \bot$ (\textit{boolean connectives});
		\item $x,y,z, \dots \dots$(\textit{justification variables});
		\item $a, b, \dots$, with indices, $1, 2, \dots$ (\textit{justification constants});
		\item $+$, $\cdot$ (\textit{justification operators});
		\item $),($ (\textit{parentheses}).
	\end{itemize}
\end{defn}

\begin{defn} (Justification terms)
	\begin{center}
		$ t :: = x$   $|$ $c$ $|$  $(t \cdot t)$ $|$ $(t + t)$ 
	\end{center}
\end{defn}



\begin{defn} (Justification formulas)
	\begin{center}
		$ \varphi :: = p$   $|$ $\bot$ $|$  $(\varphi \impli \varphi)$ $|$ $t$$:$$\varphi$
	\end{center}
\end{defn}

\qquad We define $\nao, \e, \see$ and $\ou$ as usual. Sometimes, to help readability, we use the brackets `$],[$' together with `$),($'.

\qquad The minimal justification logic J$_{0}$ is axiomatized by the following axiom schemes and inference rules:\\

All tautologies\\

(\textit{Application Axiom}) $t$$:$$(\varphi \impli \psi) \impli$ $(s$$:$$\varphi \impli$ $[t\cdot s]$$:$$\psi)$\\

(\textit{Sum Axioms}) $t$$:$$\varphi \impli$ $[t+s]$$:$$\varphi$, $s$$:$$\varphi \impli$ $[t+s]$$:$$\varphi$\\

(\textit{Modus Ponens}) $\teo \varphi$, $\teo \varphi\impli\psi$ $\Rightarrow$ $\teo \psi$ \\

(\textit{axiom necessitation})  $\teo c$$:$$\varphi$, where $\varphi$ is an axiom and $c$ is a justification constant.\\

\qquad The notion of derivation in this system, J$_{0}$ $\teo \varphi$, is defined as usual. 

\begin{defn}
Let $\C$ be a non-empty set of formulas. We say that $\C$ is a \textit{constant specification}, if for every $\varphi \in \C$, $\varphi = c$$:$$\psi$ where $c$ is a justification constant and $\psi$ is an axiom. A proof meets constant specification $\C$ provided that whenever the inference rule `axiom necessitation' is used to introduce $c$$:$$\psi$, then $c$$:$$\psi \in \C$.

\qquad We say that a constant specification $\C$ is \textit{axiomatically appropriate} if i) for every axiom $\varphi$ there is a justification constant $c_{1}$ such that $c_{1}$$:$$\varphi\in \C$; and ii) if $c_n, c_{n-1}, \dots , c_{1}$$:$$\varphi\in \C$, then $c_{n+1}, c_{n}, \dots , c_{1}$$:$$\varphi\in \C$, for each $n \geq 1$. 	
\end{defn}


\qquad For a constant specification $\C$, by J$_{\C}$ we mean J$_{0}$ plus formulas from $\C$ as additional axioms.

\begin{teor}
(Internalization) Suppose $\C$ is an axiomatically appropriate constant specification. In these conditions, J$_{\C}$ satisfies internalization. That is, if J$_{\C}$ $\teo \varphi$ then J$_{\C}$ $\teo  t$$:$$\varphi$, for some justification term $t$.
\end{teor}

\qquad There are some well-know examples of justification logic other than J$_{0}$; in this thesis we are going to mention only two of them. The first one is the already mentioned Logic of Proofs (LP): it extends the language of J$_{0}$ with the unary justification operator $!$ and has the following additional axiom schemes:\\

(\textit{Factivity Axiom}) $t$$:$$\varphi \impli \varphi$\\

(\textit{Positive Introspection Axiom}) $t$$:$$\varphi \impli$ $!t$$:$$t$$:$$\varphi$\\

\qquad The second one is called JT45, it extends the language of LP with the unary justification operator $?$ and has the following additional axiom scheme:\\

(\textit{Negative Introspection Axiom}) $\nao t$$:$$\varphi \impli$ $?t$$:$$\nao t$$:$$\varphi$\\

\qquad We have stated the Internalization Theorem above for J$_{0}$, but this theorem also holds for LP and JT45. Because of the Positive Introspection Axiom we can prove this result for LP and JT45 with a weaker notion of axiomatically appropriate constant specification $\C$. In both of these logics we just say that a constant specification $\C$ is axiomatically appropriate if for every axiom $\varphi$ there is a justification constant $c$ such that $c$$:$$\varphi\in \C$. It should be noted that the Internalization Theorem is just an explicit form of the necessitation rule.

\qquad Informally speaking, the \textit{forgetful projection} of a justification formula $\varphi$, denoted $\varphi^{\circ}$, is the result of replacing every subformula $t$$:$$\psi$ with $\Box\psi$. We also commented on the notion of \textit{realization}. With these two notions we can state more clearly the relationship between modal logic and justification logic.



\begin{defn}
Suppose KL is a normal modal logic and let JL be a justification logic mentioned above. We say that JL is a \textit{counterpart} of KL if the following holds:

\begin{itemize}
\item  If JL $\teo \varphi$, then KL $\teo \varphi^{\circ}$.
\item  If KL $\teo \varphi$, then there is a realization $r$ such that JL $\teo \varphi^{r}$.
\end{itemize}
\end{defn}


\qquad It can be proved that for an axiomatically appropriate constant specification $\C$:

\begin{center}
J$_{\C}$ is a counterpart of K\\

LP is a counterpart of S4\\

JT45 is a counterpart of S5
\end{center}

 


\section{From propositional logic to first-order}

\qquad Before we start presenting the first-order version of JT45 we need to remember some properties of derivations in classical first-order logic. It is useful to remember these details, because first-order justification logic tries to mirror some aspects of the individual variables in classical first-order derivations.     

\qquad Let $\varphi(x)$ be any tautology, and let $t$ be the following derivation:

\begin{enumerate}[1.]
	\item $\varphi(x)$ 
	\item $\todo x \varphi(x)$                 (generalization)
	\item $\todo x\varphi(x) \impli (Px \impli \todo x\varphi(x))$ (tautology)
	\item $Px \impli \todo x\varphi(x)$ (Modus Ponens)
\end{enumerate}

\qquad Although $x$ is free in the formula $Px \impli \todo x\varphi(x)$, if $c$ is an individual term we cannot substitute $c$ for $x$ in $t$ in order to obtain a derivation $t(c / x)$ of $Pc \impli \todo x\varphi(x)$ (if we do that we ruin the derivation at 2).
	
\qquad Now, let $s$ be the following derivation:
	
\begin{enumerate}[1.]
\item $\varphi(x)$ 
\item $\todo x \varphi(x)$                 (generalization)
\item $\todo x\varphi(x) \impli (Py \impli \todo x\varphi(x))$ (tautology)
\item $Py \impli \todo x\varphi(x)$ (Modus Ponens)
\end{enumerate}
	
\qquad $y$ is free in the formula $Py \impli \todo x\varphi(x)$ and moreover for every individual term $c$ the result of substituting $c$ for $y$ in $s$, $s(c/y)$, is a  derivation of $Pc \impli \todo x\varphi(x)$.


\qquad These examples show us that there are two different roles of variables in a derivation: a variable can be a \textit{formal symbol} that can be subjected to generalization or a \textit{place-holder} that can be substituted for. In $t$, $x$ is both a formal symbol and a place-holder. And in $s$, $x$ is a formal symbol and $y$ is a place-holder.
	
\qquad This consideration motivates the following definition:
	
\begin{center}
$x$ \textit{is free in the derivation} $t$ of the formula $\varphi$ iff for every individual term $c$, $t(c/x)$ is a derivation of $\varphi(c/x)$.
\end{center}

	
\qquad In propositional justification logic we write $t$$: $$\varphi$ to express that $t$ is a derivation of $\varphi$. In order to represent the distinct roles of variables in first-order justification logic, we are going to write formulas of the form:

\begin{center}
$t$$:$$Px \impli \todo x\varphi(x)$\\
$s$$:_{\{y\}}$$Py \impli \todo x\varphi(x)$
\end{center}
	
	
\qquad The role of $\{y\}$ in $s$$:_{\{y\}}$$Py \impli \todo x\varphi(x)$ is to point out that $y$ is free in the derivation $s$ of $Py \impli \todo x\varphi(x)$. 





\chapter{First-order JT45}
\qquad This chapter is based on three different texts. We have used \cite{Artemov11} and \cite{Fitting14} to lay down the basic syntax and semantics of first-order JT45. To prove completeness we have used an unpublished paper by Melvin Fitting. The first time Sergei Artemov constructed the quantified version of LP, it could support a constant domain semantics. In the unpublished paper Fitting proved completeness for that early version of first-order LP. Since Artemov changed the construction of the quantified version of LP, Fitting left that paper unpublished. The Completeness Theorem presented in this chapter is just an adaptation of the proof strategy presented in that paper (the use of \textit{templates}) for first-order JT45. 



\section{Language and axiom system}


\qquad For this whole chapter we set $\Li = \{P, Q, P\p, Q\p, \dots \}$ to be a countable relational language with no propositional letters.


\begin{defn} (Basic vocabulary)
	
	\begin{itemize} 
		\item $x_{0}, x_{1}, x_{2}, \dots$ (\textit{individual variables});
		\item $\impli, \bot$ (\textit{boolean connectives});
		\item $\todo$ (\textit{universal quantifier});
		\item $p_{0}, p_{1}, p_{2}, \dots$(\textit{justification variables});
		\item $c_{0}, c_{1}, c_{2A}, \dots$ (\textit{justification constants});
		\item $+$, $\cdot$, $!$, $?$, $gen_{x}$ (\textit{justification operators -- for every individual variable $x$, there is an operator $gen_{x}$})\footnote{To be precise, there is a operator $gen_{i}$ for each $i \in \omega$. We identify each operator $gen_{i}$ with the individual variables $x_{i}$. There is no occurrence of a variable in a justification operator, it is just a label.};
		\item $(\cdot):_{X} (\cdot)$,(for every finite set of individual variables $X$);
		\item $),($ (\textit{parentheses}).
	\end{itemize}
\end{defn}

\begin{defn} (First-order justification terms)
	\begin{center}
		$ t :: = p_{i}$   $|$ $c$ $|$  $(t \cdot t)$ $|$ $(t + t)$ $|$  $!t$ $|$ $?t$ $|$ $gen_{x}(t)$
	\end{center}
\end{defn}

\begin{defn} (First-order justification formulas)
	\begin{center}
		$ \varphi :: = Px_1 \dots x_n$   $|$ $\bot$ $|$  $(\varphi \impli \varphi)$ $|$ $\todo x \varphi$ $|$  $t$$:_{X}$$\varphi$
	\end{center}
\end{defn}


\qquad The set of all formulas is denoted by $\Fj$. We are assuming that the set of individual variables, justification variables and justification constants are all countable sets. Thus, it is easy to check that $\Fj$ itself is a countable set. 

\begin{defn}
	We define the notion of free variables of $\varphi$, $fv(\varphi)$, recursively as follows:
	
	\begin{itemize} 
		\item If $\varphi$ is atomic, then $fv(\varphi)$ is the set of all variables occurring in $\varphi$.
		\item If $\varphi$ is $(\psi \impli \theta)$, then $fv(\varphi)$ is $fv(\psi) \cup fv(\theta)$.
		\item If $\varphi$ is $\todo x \psi$, then $fv(\varphi)$ is $fv(\psi) \backslash \{x\}$.
		\item If $\varphi$ is $t$$:_{X}$$\psi$, then  $fv(\varphi)$ is $X$.
	\end{itemize}
	
	
	\qquad Similarly as in the classical case, we must define the notion of an individual variable $y$ being free for $x$ in the formula $\varphi$. The definition is the same as in the classical case, we only add the following clause: $y$ is free for $x$ in $t$$:_{X}$$\varphi$ if two conditions are met, i) $y$ is free for $x$ in $\varphi$ (in the classical sense), ii) if $y \in fv(\varphi)$, then $y \in X$.
\end{defn}

\qquad We write $Xy$ instead of $X \cup \{y\}$; in this case it is assumed that $y \notin X$. And we use $t$$:$$\varphi$ as an abbreviation for $t$$:_{\vazio}$$\varphi$

\qquad  The first-order JT45, FOJT45, is axiomatized by the following axiom schemes and inference rules:\\

\textbf{A1} classical axioms of first-order logic\\

\textbf{A2} $t$$:_{Xy}$$\varphi \impli$ $t$$:_{X}$$\varphi$, provided $y$ does not occur free in $\varphi$\\

\textbf{A3} $t$$:_{X}$$\varphi \impli$ $t$$:_{Xy}$$\varphi$ \\

\textbf{B1} $t$$:_{X}$$\varphi \impli \varphi$\\

\textbf{B2} $t$$:_{X}$$(\varphi \impli \psi) \impli$ $(s$$:_{X}$$\varphi \impli$ $[t\cdot s]$$:_{X}$$\psi)$\\

\textbf{B3} $t$$:_{X}$$\varphi \impli$ $[t+s]$$:_{X}$$\varphi$, $s$$:_{X}$$\varphi \impli$ $[t+s]$$:_{X}$$\varphi$\\ 

\textbf{B4} $t$$:_{X}$$\varphi \impli$ $!t$$:_{X}$$t$$:_{X}$$\varphi$\\


\textbf{B5} $\nao t$$:_{X}$$\varphi \impli$ $?t$$:_{X}$$\nao t$$:_{X}$$\varphi$\\


\textbf{B6} $t$$:_{X}$$\varphi \impli$ $gen_{x}(t)$$:_{X}$$ \todo x \varphi$, provided $x \notin X$\\


\textbf{R1} (\textit{Modus Ponens}) $\teo \varphi$, $\teo \varphi\impli\psi$ $\Rightarrow$ $\teo \psi$ \\

\textbf{R2} (\textit{generalization})  $\teo \varphi$ $\Rightarrow$ $\teo \todo x \varphi$ \\

\textbf{R3} (\textit{axiom necessitation})  $\teo c$$:$$\varphi$, where $\varphi$ is an axiom and $c$ is a justification constant.\\

\qquad We use $\Gamma, \Delta, \Theta, \dots$ as variables for sets of formulas. The notion of $\Gamma \teo \varphi$ is defined as usual. The only thing that should be noted is that, if $\Gamma$ deduces $\varphi$ using the generalization rule, then this rule was not applied to a variable which occurs free in the formulas of $\Gamma$. 

\qquad Since derivations depend on the constant specification being considered, we sometimes write $\teo_{\C} \varphi$ to point out that the proof of $\varphi$ meets the constant specification $\C$.

\begin{lema}
	(\textit{Deduction})  $\Gamma,\varphi \teo \psi$ iff  $\Gamma \teo \varphi \impli \psi$.
\end{lema}

\begin{proof}
	A similar proof as the one from the classical case.
\end{proof}


\begin{teor}
	(\textit{Internalization}) Let $\C$ be an axiomatically appropriate constant specification; $p_{0}, \dots, p_{k}$ be justification variables; $X_{0}, \dots, X_{k}$ be finite sets of individual variables, and $X =X_{0} \cup \dots \cup X_{k}$. In these conditions, if  $p_{0}$$:_{X_{0}}$$\varphi_{0}, \dots, p_{k}$$:_{X_{k}}$$\varphi_{k} \teo_{\C} \psi$, then there is a justification term $t(p_{0}, \dots, p_{k})$ such that 
	
	\begin{center}
		$p_{0}$$:_{X_{0}}$$\varphi_{0}, \dots, 
		p_{k}$$:_{X_{k}}$$\varphi_{k} \teo_{\C} t$$:_{X}$$\psi$.
	\end{center}
	
\end{teor}

\begin{proof}
	The same proof as presented in \cite[p. 7]{Artemov11}.
\end{proof}



\begin{pro}
	(\textit{Explicit counterpart of the Barcan Formula and its converse}) Let $y$ be an individual variable. For  every finite set of individual variables $X$ such that $y \notin X$, for every formula $\varphi(y)$ and every justification term $t$, there are justification terms $CB(t)$ and $B(t)$ such that: 
	\begin{center}
		$\teo t$$:_{X}$$\todo y \varphi(y) \impli \todo y CB(t)$$:_{Xy}$$\varphi(y)$\\
		
		$\teo \todo y t$$:_{Xy}$$\varphi(y) \impli B(t)$$:_{X}$$\todo y \varphi(y)$
	\end{center}
\end{pro}



\begin{proof}
	\qquad In Appendix.
\end{proof}



\begin{pro}
	Let $y$ be an individual variable. For  every finite set of individual variables $X$ such that $y \notin X$, for every formula $\varphi(y)$ and every justification term $t$, there is a justification term $s(t)$ such that: 
	\begin{center}
		$\teo \ex y t$$:_{Xy}$$\varphi(y) \impli s(t)$$:_{X}$$\ex y \varphi(y)$
	\end{center}
\end{pro}



\begin{proof}
	\qquad In Appendix.
\end{proof}







\section{Semantics: basic definitions}

\qquad In Chapters 2 and 3 we have used valuation functions to define the relation $\models$. In the present case it is more convenient to define the semantic notions adding constants to the basic language. That is the path that we take here. So, for any non-empty set $\D$ we are going to use the elements of $\D$ as constants. And  we are going to use $\vec{a}, \vec{b},  \dots$ to denote sequences of constants.

\begin{defn}
	Let $\D$ be a non-empty set. The set of all $\D$-formulas, $\D$-$\Fj$, is defined as follows:
	
	\begin{center}
		$\D$-$\Fj = \{\varphi (\vec{a})$ $|$  $\varphi(\vec{x}) \in \Fj$ and $\vec{a} \in \D\}$.
	\end{center}
	
\end{defn}

\qquad As usual, for a $\D$-formula $\varphi$, we say that $\varphi$ is closed if  $\varphi$ has no free variables.

\begin{defn}
	A \textit{Fitting model} is a structure $\M = \model$ where $\bl \W, \R, \D \br$ is a skeleton, $\R$ is an equivalence relation\footnote{Of course, we can define a Fitting model more generally for any kind of relation $\R$, but for our purposes we are going to use this restricted definition.},  $\I$ is an \textit{interpretation function} and:
	
	\begin{itemize} 
		\item $\E$ is an \textit{evidence function}, i.e., for any justification term $t$ and $\D$-formula $\varphi$, $\E(t,\varphi) \subseteq \W$.
	\end{itemize}
	
\end{defn}



\begin{defn}
	\textit{Evidence Function Conditions}. Let $\M = \model$ be a Fitting model. We require the evidence function to meet the following conditions:
	
	
	\begin{itemize} 
		\item[] \textbf{$\cdot$ Condition} $\E (t, \varphi \impli \psi) \cap \E(s, \varphi) \subseteq \E([t\cdot s], \psi).$
		\item[] \textbf{$+$ Condition} $\E (s, \varphi) \cup \E(t, \varphi) \subseteq \E([s+t], \varphi).$
		\item[] \textbf{$!$ Condition} $\E (t, \varphi) \subseteq \E(!t, t$$:_{X}\varphi)$, where $X$ is the set of constant occurring in $\varphi$.
		\item[] \textbf{$?$ Condition} $\W  \backslash \E (t, \varphi) \subseteq \E(?t,\nao t$$:_{X}\varphi)$, where $X$ is the set of constants occurring in $\varphi$.
		\item[] \textbf{$\R$ Closure Condition} If $w \in \E (t, \varphi)$ and $w \R w\p$, then $w\p \in \E (t, \varphi)$.
		\item[] \textbf{Instantiation Condition} If $w \in \E (t, \varphi(x))$ and $a \in \D$, then $w \in \E (t, \varphi(a))$.
		\item[] \textbf{$gen_{x}$ Condition} $\E (t, \varphi) \subseteq \E(gen_{x}(t),\todo x\varphi)$.
	\end{itemize}
\end{defn}

\qquad We say that a model $\M = \model$ \textit{meets constant specification $\C$} iff whenever $c$$:$$\varphi \in \C$, then $\E (c, \varphi) = \W$.


\begin{defn}
	Let $\M = \model$ be a Fitting model, $\varphi$ a closed $\D$-formula and $w \in \W$. The notion that \textit{$\varphi$ is true at world $w$ of $\M$}, in symbols $\M,w \models \varphi$, is defined recursively as follows: 
	\begin{itemize} 
		\item $\M,w \models P(\vec{a})$ iff $\bl \vec{a}\br \in \I(P,w)$. 
		\item $\M,w \nmodels \bot$. 
		\item $\M,w \models \psi \impli \theta$ iff $\M,w \nmodels \psi$ or $\M,w \models \theta$.
		\item $\M,w \models \todo x \psi(x)$ iff for every $a \in \D$, $\M,w \models \psi(a)$.        
		\item Assume $t$$:_{X}$$\psi(\vec{x})$ is closed and $\vec{x}$ are all the free variables of $\psi$. Then, $\M,w \models t$$:_{X}$$\psi(\vec{x})$ iff
		\begin{enumerate}[(a)]
			\item $w \in \E (t, \psi(\vec{x}))$ and
			\item for every $w\p \in \W$ such that $w\R w\p$, $\M,w\p \models \psi(\vec{a})$ for every $\vec{a} \in \D$.
		\end{enumerate}
		
	\end{itemize}
	
\end{defn}



\begin{defn}
	Let $\varphi \in \Fj$ be a closed formula. We say that $\varphi$ is \textit{valid in the Fitting model} $\M = \model$ provided for every $w \in W$, $\M,w \models \varphi$. A formula with free individual variables is valid if its universal closure is valid.
\end{defn}


\begin{defn}
	A \textit{Fitting model for FOJT45} is a Fitting model $\M = \model$ where $\E$ is a \textit{strong evidence function}, i.e., for every term $t$ and $\D$-formula $\varphi$, $\E(t,\varphi) \subseteq \{w \in \W$ $|$ $ \M,w \models t$$:_{X}$$\varphi\}$ where $X$ is the set of constant occurring in $\varphi$.
	
	
	\qquad For a formula $\varphi$ and constant specification $\C$, we write $\models_{\C}\varphi$ if for every Fitting model for FOJT45 $\M$ meeting $\C$, $\varphi$ is valid in $\M$.
\end{defn}



\section{Semantics: non-validity}

\qquad Before we deal with soundness and completeness, it is useful to know some examples of non-validity in order to see that the provisions of some axioms make sense. There is only a minor problem, we require that Fitting models for FOJT45 have a strong evidence function, and it is not so easy to construct models with that property. The following proposition helps us to circumnavigate this issue.


\begin{pro}
	If $\M = \model$ is a Fitting model such that for every justification term $t$ and $\D$-formula $\varphi$, $\E(t,\varphi) = \W$, then there is a Fitting model for FOJT45 $\M^{*} = \bl\W,\R,\D,\I,\E^{*} \br$ such that for every $w \in \W$ and every formula $\varphi$, $\M,w \models \varphi$ iff $\M^{*},w \models \varphi$.   
\end{pro}

\begin{proof}
	\qquad Let $\M^{*} =\bl\W,\R,\D,\I,\E^{*} \br$ where for every justification term and $\D$-formula $\varphi$,
	
	\begin{center}
		$\E^{*}(t,\varphi) = \{w \in \W$ $|$ $ \M,w \models t$$:_{X}$$\varphi\}$
	\end{center}
where $X$ is the set of constants occurring in $\varphi$. 
	
	\qquad It is straightforward to check that $\M^{*}$ is indeed a Fitting model. Now consider the following:\\
	
	(+) For every $w \in \W$ and every closed $\D$-formula $\varphi$, $\M,w \models \varphi$ iff $\M^{*},w \models \varphi$.\\
	
	(Proof of (+)) Induction on $\varphi$. Crucial case, $\varphi$ is $t$$:_{X}$$\psi$. For simplicity, let us assume that $\varphi$ is $t$$:_{\{a\}}$$\psi(a,y)$.
	
	\qquad ($\Rightarrow$) If $\M,w \models t$$:_{\{a\}}$$\psi(a,y)$, then by definition $w \in \E^{*}(t, \psi(a,y))$ and for every $w\p \in \W$, if $w\R w\p$, then $\M,w\p \models \psi(a,b)$ for every $b \in \D$. By the induction hypothesis, for every $w\p \in \W$, if $w\R w\p$, then $\M^{*},w\p \models \psi(a,b)$ for every $b \in \D$. Thus, $\M^{*},w \models t$$:_{\{a\}}$$\psi(a,y)$.
	
	\qquad ($\Leftarrow$) If $\M^{*},w \models t$$:_{\{a\}}$$\psi(a,y)$, then $w \in \E^{*}(t, \psi(a,y))$. By definition, $\M,w \models t$$:_{\{a\}}$$\psi(a,y)$. $\Box$\\
	
	
	\qquad By (+) we have that,
	
	\begin{center}
		$\E^{*}(t,\varphi) = \{w \in \W$ $|$ $ \M,w \models t$$:_{X}$$\varphi\} = \{w \in \W$ $|$ $ \M^{*},w \models t$$:_{X}$$\varphi\}$
	\end{center}
	
	
	\qquad Hence, $\E^{*}$ is a strong evidence function and $\M$ and $\M^{*}$ agree on all $\D$-formulas. Therefore, $\M^{*}$ is a Fitting model for FOJT45 and $\M$ and $\M^{*}$ agree on all formulas.
\end{proof}


\qquad With this proposition we can construct non-validity examples similar to those presented in \cite{Fitting14}.

\qquad \textbf{Example 1:} the restriction on axiom \textbf{A2} is needed. Take, for example, the formula $t$$:_{\{x,y\}}$$Qxy \impli t$$:_{\{x\}}$$Qxy$; let $\M =\model$ be a Fitting model where:
\begin{itemize}
	\item $\W = \{w_0, w_1\}$;
	\item $\R = \W \times \W$;
	\item $\D = \{a, b\}$;
	\item $\I(w_{0},Q) = \I(w_{1},Q) = \{\bl a,b \br\}$;
	\item $\E(t,\varphi) = \W$, for every term $t$ and formula $\varphi$.
\end{itemize}


\qquad Clearly, $\M,w_0 \models t$$:_{\{a,b\}}$$Qab$ and $\M,w_0 \nmodels t$$:_{\{a\}}$$Qay$. Hence, $\M,w_0 \nmodels t$$:_{\{x,y\}}$$Qxy \impli t$$:_{\{x\}}$$Qxy$. By Proposition 19, $t$$:_{\{x,y\}}$$Qxy \impli t$$:_{\{x\}}$$Qxy$ is not valid in every Fitting model for FOJT45.


\qquad \textbf{Example 2:} The proviso of axiom \textbf{B6} is necessary. Take, for example, the formula $t$$:_{\{x\}}$$Qx \impli gen_{x}(t)$$:_{\{x\}}$$\todo x Qx$; let $\M =\model$ be a Fitting model where:
\begin{itemize}
	\item $\W = \{w_0\}$;
	\item $\R = \W \times \W$;
	\item $\D = \{a,b\}$;
	\item $\I(w_{0},Q) =  \{a\}$;
	\item $\E(t,\varphi) = \W$, for every term $t$ and formula $\W$.
\end{itemize}


\qquad Clearly, $\M,w_0 \models t$$:_{\{a\}}$$Qa$ and since $\M,w_0 \nmodels Qb$, then $\M,w_0 \nmodels \todo x Qx$, and so $\M,w_0 \nmodels gen_{x}(t)$$:_{\{a\}}$$\todo x Qx$. Hence, $\M,w \nmodels t$$:_{\{x\}}$$Qx \impli gen_{x}(t)$$:_{\{x\}}$$\todo x Qx$. Again by Proposition 19, $t$$:_{\{x\}}$$Qx \impli gen_{x}(t)$$:_{\{x\}}$$\todo x Qx$ is not valid in every Fitting model for FOJT45.

\section{Soundness and Completeness}

\subsection{Soundness}

\begin{teor}
	(\textit{Soundness}) Let $\C$ be a constant specification. For every formula $\varphi \in \Fj$, if $\teo_{\C} \varphi$, then $\models_{\C}\varphi$.
\end{teor}    

\begin{proof}
	The proof is by induction on the theorems of the axiom system using the constant specification $\C$. The argument is exactly the same as presented in \cite[pp. 9-10]{Fitting14}. We are going to show validity for the specific axiom of FOJT45.
	
	\qquad Suppose $\varphi$ is an instance of \textbf{B5}, i.e., $\varphi$ is $\nao t$$:_{X}$$\psi \impli$ $?t$$:_{X}$$\nao t$$:_{X}$$\psi$. For simplicity, assume $X= \{x\}$ and $\psi = \psi(x,y)$. So, we have that $\teo_{\C}\nao t$$:_{\{x\}}$$\psi(x,y) \impli$ $?t$$:_{\{x\}}$$\nao t$$:_{\{x\}}$$\psi(x,y)$.
	
	\qquad Let $\M = \model$ be a Fitting model for FOJT45 meeting $\C$, $w \in \W$ and $a \in \D$. Suppose $\M, w \models \nao t$$:_{\{a\}}$$\psi(a,y)$. Then, $\M, w \nmodels t$$:_{\{a\}}$$\psi(a,y)$. By the definition of the strong evidence function, $w \notin \E (t, \psi(a,y))$. By the ? condition, $w \in \E(?t,\nao t_{\{a\}}$$:\psi(a,y))$. Again, by the strong evidence function $\M, w \models ?t$$:_{\{a\}}$$\nao t$$:_{\{a\}}$$\psi(a,y)$.
	
\end{proof}


\subsection{An obstacle in the proof of the Completeness Theorem}

\qquad There are two ways that we can prove the Completeness Theorem, one simple and the other more complex. Here we shall present the complex version. Although we are going to have much more work (if compared to the simple version) it is worthwhile because, we believe that \textit{the methods that we are going to use in the next subsections can be used to prove the semantical version of the Realization Theorems for FOJT45} (in Chapter 6 we give a more detailed exposition of that theorem).   

\qquad The general strategy is the same as presented in \cite[pp. 256-265]{Hughes96}. Let us just briefly comment on what is the obstacle that we find when trying to adapt the proof from the modal case to the justification case. In one step of the proof \cite[pp. 259-260]{Hughes96} we need to establish the following: \\

(+) There is an individual variable $y^{*}$ such that $\Gamma^{\#} \cup \{ \gamma_{n} \e (\delta(y^{*}/ x) \impli \todo x \delta) \}$ is consistent,\\ 
where $\Gamma^{\#} = \{\varphi$ $|$ $\Box \varphi \in \Gamma\}$ and $\Gamma$ is a maximal consistent set. We begin proving (+) with the following argument. Suppose (+) is false.

\qquad (1) Then for every individual variable $y$,  $\Gamma^{\#} \cup \{ \gamma_{n} \e (\delta(y/ x) \impli \todo x \delta) \}$ is inconsistent. Hence, for some $\beta_{1}, \dots, \beta_{k} \in \Gamma^{\#}$ we have that

\begin{center}
	$\teo (\beta_{1} \e \dots \e \beta_{k}) \impli( \gamma_{n} \impli \nao (\delta(y/ x) \impli \todo x \delta))$;
\end{center}
by the usual reasoning in modal logic,


\begin{center}
	$\teo (\Box\beta_{1} \e \dots \e \Box\beta_{k}) \impli\Box( \gamma_{n} \impli \nao (\delta(y/ x) \impli \todo x \delta))$
\end{center}

\qquad Since $\Box\beta_{1}, \dots, \Box\beta_{k} \in \Gamma$, then $\Box( \gamma_{n} \impli \nao (\delta(y/ x) \impli \todo x \delta)) \in \Gamma$.

\qquad (2) It is assumed that $\Gamma$ has the `$\todo$-property', i.e., for every formula $\varphi (x)$ there is an individual variable $y^{*}$ such that   $\varphi(y^{*}/ x) \impli \todo x \varphi \in \Gamma$.

\qquad Now, using these two facts we can conclude the following: let $z$ be a variable that does not occur in $\gamma_{n}$ and $\delta$. By (2), there is a variable $y^{*}$ such that

\begin{center}
	$\Box(\gamma_{n} \impli \nao (\delta(y^{*}/ x) \impli \todo x \delta)) \impli \todo z \Box(\gamma_{n} \impli \nao (\delta(z/ x) \impli \todo x \delta)) \in \Gamma$
\end{center}

\qquad And by (1) for the particular case when $y = y^{*}$,


\begin{center}
	$\Box( \gamma_{n} \impli \nao (\delta(y^{*}/ x) \impli \todo x \delta)) \in \Gamma$.
\end{center}

\qquad So, by the maximal consistency of $\Gamma$ we can conclude that $\todo z \Box(\gamma_{n} \impli \nao (\delta(z/ x) \impli \todo x \delta)) \in \Gamma$. The rest of the proof of (+) is not important for our point here.

\qquad The adaptation of this step for the first-order justification logic is problematic because \textit{justification terms internalize Hilbert-style derivations}.

\qquad It should be noted that for two different individual variables $y$ and $y\p$ if $\Gamma^{\#} \cup \{ \gamma_{n} \e (\delta(y/ x) \impli \todo x \delta) \}$ and $\Gamma^{\#} \cup \{ \gamma_{n} \e (\delta(y\p/ x) \impli \todo x \delta) \}$ are inconsistent sets, then there are two finite subsets of $\Gamma^{\#}$, $\{ \beta_{1}, \dots, \beta_{k} \}$ and $\{ \beta\p_{1}, \dots, \beta\p_{k\p} \}$ such that


\begin{center}
	$\teo (\beta_{1} \e \dots \e \beta_{k}) \impli( \gamma_{n} \impli \nao (\delta(y/ x) \impli \todo x \delta))$\\
	$\teo (\beta\p_{1} \e \dots \e \beta\p_{k\p}) \impli( \gamma_{n} \impli \nao (\delta(y\p/ x) \impli \todo x \delta))$
\end{center}
and we cannot assume that $\{ \beta_{1}, \dots, \beta_{k} \}=\{ \beta\p_{1}, \dots, \beta\p_{k\p} \}$. So, for each variable $y$ we may have a different derivation.

\qquad If we adopt the argument (1) for first-order justification logic we would have that for each individual variable $y$ 

\begin{center}
	$t^{y}$$:_{X}$$( \gamma_{n} \impli \nao (\delta(y/ x) \impli \todo x \delta)) \in \Gamma$,
\end{center}
where $t^{y}$ is a term constructed by the Internalization Theorem, the axiom \textbf{B2} and the fact that $\Gamma^{\#} \cup \{ \gamma_{n} \e (\delta(y/ x) \impli \todo x \delta) \}$ is inconsistent. Hence, $t^{y}$ \textit{depends on the individual variable} $y$. 

\qquad Now, let us try to continue the argument. Let $z$ be a variable that does not occur in $\gamma_{n}$ and $\delta$. If we adapt (2) for justification logic, we would have that for every individual variable $y$ there is an individual variable $y^{*}$ such that

\begin{center}
	$t^{y}$$:_{X}$$(\gamma_{n} \impli \nao (\delta(y^{*}/ x) \impli \todo x \delta)) \impli \todo z t^{y}$$:_{X}$$(\gamma_{n} \impli \nao (\delta(z/ x) \impli \todo x \delta)) \in \Gamma$
\end{center}


\qquad But from this adapted version of (2) we cannot conclude that there is a variable $y^{*}$ such that


\begin{center}
	$t^{y^{*}}$$:_{X}$$(\gamma_{n} \impli \nao (\delta(y^{*}/ x) \impli \todo x \delta)) \impli \todo z t^{y^{*}}$$:_{X}$$(\gamma_{n} \impli \nao (\delta(z/ x) \impli \todo x \delta)) \in \Gamma$
\end{center}

\qquad  So  we cannot use (1) to conclude that $\todo z t^{y^{*}}$$:_{X}$$(\gamma_{n} \impli \nao (\delta(z/ x) \impli \todo x \delta)) \in \Gamma$.


\qquad  A way to remedy this problem is to make the `$\todo$-property' stronger. If  $\varphi(y^{*}/ x) \impli \todo x \varphi \in \Gamma$ we say that $y^{*}$ instantiates the formula $\todo x \varphi$. We want that the same individual variable is used to simultaneously instantiate an infinite list of formulas of the same form. In order to guarantee this feature we are going to use the notion of \textit{templates}. But in doing so we need to stablish some facts about templates. That makes the proof bigger than it should be, and that is why we divided the proof of the Completeness Theorem into different subsections.  




\subsection{Language extension}

\qquad The basic idea is to extend the language in order to prove a Henkin-style Completeness Theorem. Instead of using constants to construct our canonical model we shall add a new kind of variable called `witness variable'. We do that because when working with maximal consistent sets we need to be able to do formal derivations and so bind some witness variables.



\begin{defn}
	Two formulas are \textit{variable variants} provided each can be turned into the other by a uniform renaming of free individual variables, bound individual variables and labels of justification terms. We are always assuming that the renaming is safe, i.e., the new variables that are being introduced do not occur in the original formula.
\end{defn}

\begin{defn}
	A constant specification $\C$ is \textit{variant closed} iff whenever $\varphi$ and $\psi$ are variable variants, then $c$$:$$\varphi \in \C$ iff $c$$:$$\psi \in \C$.
\end{defn}


\begin{defn}
	Fix a countable set \textbf{V} $=\{a_{0}, a_{1}, a_{2}, \dots \}$ of additional individual variables that are not in the original language. We define a new set of formulas $\Fjv$ in the same fashion as $\Fj$. It should be noted that variables of \textbf{V} can be bound. We add every finite subset of $\textbf{V}\cup \{x_{0}, x_{1}, \dots \}$ to the language; and for every $a \in \textbf{V}$ we add the justification operator $gen_{a}$.\footnote{To be precise, we add $gen_{\omega +i}$ for each $i \in \omega$. And we identify each operator $gen_{\omega +i}$ with $a_{i}$.} It can be easily checked that $\Fjv$ is a countable set.
	
\end{defn}

\qquad Until the end of this chapter we write `individual variables' to denote the members of $\textbf{V}\cup \{x_{0}, x_{1}, \dots \}$, `basic variables' to denote the members of $\{x_{0}, x_{1}, \dots \}$ and `witness variables' to denote the members of \textbf{V}.


\qquad We are interested in using $\textbf{V}$ as the domain $\D$ of the canonical model, so from now on we shall call a $\D$-formula a formula of $\Fjv$ where the members of $\textbf{V}$ \textit{occur only free} (not bound, nor as labels of justification terms). And we say that a $\D$-formula is closed if no basic variable occurrences are free.



\qquad Together with this new language we construct a new axiomatic system for FOJT45 based on the formulas from $\Fjv$.


\begin{defn}
	Let $\C$ be a variant closed constant specification for the basic system. $\Cv$ is the smallest set satisfying the following:
	
	\begin{itemize}
		\item[] If $\varphi \in \C$, $\psi \in \Fjv$ and $\varphi$ and $\psi$ are variable variants, then $\psi \in \Cv$.
	\end{itemize}    
\end{defn}



\qquad From this definition we can make some observations:
\begin{itemize}
	\item $\C \subseteq \Cv$.
	\item $\Cv$ is variant closed.
	\item If $\C$ is axiomatically appropriate, then $\Cv$ is axiomatically appropriate. 
	\item We can prove the Deduction Lemma, the Internalization Theorem, Propositions 17 and 18 for the new axiom system.
\end{itemize}

\begin{pro}
	Let $\C$ be a variant closed constant specification for the basic system and $\Cv$ its extension for $\Fjv$. In these conditions, for every $\varphi \in \Fj$, if $\teocv \varphi$, then $\teoc \varphi$. 
\end{pro}

\begin{proof}
	Let $\psi_{1}, \psi_{2}, \dots , \psi_{n} = \varphi$ be a FOJT45 proof in the language of $\Fjv$ using $\Cv$. Let $a_{1}, \dots, a_{k}$ be all the witness variables that occur free, bound or as a label in the proof. Let $y_{1}, \dots, y_{k}$ be basic variables that do not appear free, bound or as a label in the proof. And let $(\psi_{i})^{-}$ be the result of replacing each $a_{j}$ with $y_{j}$ throughout.
	
	\qquad We shall show that  $(\psi_{1})^{-}, (\psi_{2})^{-}, \dots , (\psi_{n})^{-}$ is a FOJT45 proof in the language of $\Fj$ using $\C$. And so $\teoc (\psi_{n})^{-}$, i.e., $\teoc \varphi$.  
	
	\qquad If $\psi_{i}$ is an axiom, since we are using axiom schemes and the introduced variables are new (to prevent that any proviso be violated), then $(\psi_{i})^{-}$ is also an axiom.    
	
	\qquad If $\psi_{i}$ is a member of $\Cv$, then there is a $\phi \in \C$ such that $\psi_{i}$ and $\phi$ are variable variants. Now, $(\psi_{i})^{-}$ and $\phi$ may not be variable variants, because they may have some basic variable in common. But we can construct a formula $\theta \in \Fj$ such that $\theta$ has no variable in common with $(\psi_{i})^{-}$ and $\phi$, $\theta$ and $(\psi_{i})^{-}$ are variable variants, and $\theta$ and $\phi$ are variable variants. Since $\phi \in \C$ and $\C$ is variant close, $(\psi_{i})^{-}\in \C$.    
	
	\qquad If $\psi_{i}$ is deduced from $\psi_{i_1}$ and $\psi_{i_2} = \psi_{i_1}\impli \psi_{i}$ by modus ponens, then $(\psi_{i_2})^{-}$ is $(\psi_{i_1})^{-} \impli (\psi_{i})^{-}$. So $(\psi_{i})^{-}$ also follows from $(\psi_{i_2})^{-}$ and $(\psi_{i_1})^{-}$ by modus ponens. 
	
	
	\qquad If $\psi_{i}$ is deduced from $\psi_{l}$ by generalization, then $\psi_{i}$ is $\todo x \psi_{l}$. If $x$ is a basic variable, then $\todo x (\psi_{l})^{-}$ is deduced from $(\psi_{l})^{-}$ by generalization. If $x = a_{j}$, then $\todo y_{j} (\psi_{l})^{-}$ is deduced from $(\psi_{l})^{-}$ by generalization.    
	
\end{proof}


\begin{pro}
	(\textit{Controlled Internalization}) Let $\C$ be a constant specification variant closed and axiomatically appropriate, $\Cv$ its expansion to $\Fjv$ and $\varphi \in \Fjv$. If $\varphi$ is a $\D$-formula and $\teocv \varphi$, then there is a justification term $t$ of $\Fj$ such that
	
	\begin{center}
		$\teocv t$$:$$\varphi$
	\end{center}
\end{pro}

\begin{proof}
	Let $a_{1}, \dots, a_{n}$ be the witness variables occurring free in $\varphi$. So we can write $\varphi$ as $\varphi(a_{1}, \dots, a_{n})$. Let $x_{1}, \dots, x_{n}$ be basic variables that do not occur in the proof of $\varphi(a_{1}, \dots, a_{n})$. By an argument similar to the one presented in the proof of Proposition 20, we have that
	
	\begin{center}
		$\teoc \varphi(x_{1}, \dots, x_{n})$
	\end{center}
	
	\qquad Since $\C$ is axiomatically appropriated, by the Internalization Theorem there is a justification term $s$ of $\Fj$ such that
	
	\begin{center}
		$\teoc s$$:$$\varphi(x_{1}, \dots, x_{n})$
	\end{center}
	
	\qquad Let `$gen_{\vec{x}}(s)$' be the abreviation of `$gen_{x_{1}}(gen_{x_{2}} \dots (gen_{x_{n}}(s)))$'. By repeated use of the axiom \textbf{B6},
	
	\begin{center}
		$\teoc gen_{\vec{x}}(s)$$:$$\todo x_{1} \dots \todo x_{n} \varphi(x_{1}, \dots, x_{n})$
	\end{center}
	
	\qquad Now, since the axiom system in the language of $\Fjv$ using $\Cv$ is an extension of the basic axiom system using $\C$, we have that
	
	\begin{center}
		$\teocv gen_{\vec{x}}(s)$$:$$\todo x_{1} \dots \todo x_{n} \varphi(x_{1}, \dots, x_{n})$
	\end{center}
	
	
	\qquad By the fact that $\Cv$ is axiomatically appropriate, we have that the following formulas are elements of $\Cv$: 
	
	
	\begin{center}
		$c_{1}$$:$$[\todo x_{1} \todo x_{2} \dots \todo x_{n} \varphi(x_{1},x_{2}, \dots, x_{n}) \impli  \todo x_{2} \dots \todo x_{n} \varphi(a_{1},x_{2}, \dots, x_{n}) ]$\\
		$c_{2}$$:$$[\todo x_{2} \todo x_{3} \dots \todo x_{n} \varphi(a_{1},x_{2},x_{3}, \dots, x_{n}) \impli  \todo x_{3} \dots \todo x_{n} \varphi(a_{1},a_{2},x_{3}, \dots, x_{n}) ]$\\
		$\vdots$\\
		$c_{n}$$:$$[\todo x_{n} \varphi(a_{1}, \dots, a_{n-1}, x_{n}) \impli  \varphi(a_{1}, \dots, a_{n-1}, a_{n}) ]$.
	\end{center}
	
	\qquad Hence, by repeated use of axiom \textbf{B2} and modus ponens,
	
	
	\begin{center}
		$\teocv [c_{n}\cdot$ $\dots$ $\cdot [c_{1} \cdot gen_{\vec{x}}(s)]]$$:$$\varphi(a_{1}, \dots, a_{n})$
	\end{center}
	
	\qquad Take $t$ as $[c_{n}\cdot$ $\dots$ $\cdot [c_{1} \cdot gen_{\vec{x}}(s)]]$. 
\end{proof}


\qquad It should be noted that in the proofs of Proposition 17 and 18 we can use Proposition 21 in the place of the Internalization Theorem. So if $\varphi(y)$ is a $\D$-formula and $t$ is a term of $\Fj$, then the terms constructed by Propositions 17 and 18 -- $CB(t)$, $B(t)$ and $s(t)$ -- are also justification terms of $\Fj$.





\begin{defn}
	Let $\C$ be a variant closed constant specification for the basic language and $\Gamma \subseteq \Fj$. We say that $\Gamma$ is $\C$-\textit{inconsistent} iff $\Gamma \teo_{\C} \bot$. By the Deduction Lemma, 
	$\Gamma$ is $\C$-inconsistent iff there is a finite subset $\{\psi_{1}, \dots, \psi_{n}\}$ of $\Gamma$ such that $\teo_{\C} (\psi_{1} \e \dots \e \psi_{n}) \impli \bot$. A set $\Gamma$ is $\C$-\textit{consistent} if it is not $\C$-inconsistent. And we say that $\Gamma$ is $\C$-\textit{maximal consistent} whenever $\Gamma$ is $\C$-consistent and $\Gamma$ has no proper extension that is $\C$-consistent. We have similar notions for $\C(\textbf{V})$.
\end{defn}

\qquad It follows from Proposition 20 that for every set of basic formulas $\Gamma$, if $\Gamma$ is $\C$-consistent, then $\Gamma$ is $\C(\textbf{V})$-consistent.



\begin{pro}
	(\textit{Lindenbaum})  Let $\C$ be a constant specification variant closed and $\Cv$ its extension. If $\Gamma \subseteq \Fjv$ is $\Cv$-consistent then there is a $\Gamma\p \subseteq \Fjv$ such
	that $\Gamma \subseteq \Gamma\p$ and $\Gamma\p$ is a $\Cv$-maximal consistent set.
\end{pro}

\begin{proof}
	A similar proof as the one from the classical case.
\end{proof}


\subsection{Templates}


\begin{defn} (Template vocabulary)    
	\begin{itemize} 
		\item $\tp_{0}$, $\tp_{1}$, $\tp_{2}$, $\dots$ (\textit{propositional variables});
		\item $\nao, \ou, \e $ (\textit{boolean connectives});
		\item $\Box$ (\textit{necessity});
		\item $),($ (\textit{parentheses}).
	\end{itemize}
\end{defn}



\qquad We are going to use $\tp$, $\tq$ and $\tr$ as meta-variables for propositional variables. Similarly, we write $\tvp$ to denote a sequence of propositional variables.

\begin{defn} 
	We define the notions of \textit{template} $F$ and the occurrence set of $F$, $occ(F)$, recursively as follows:
	
	\begin{enumerate}[a)]
		\item 
		\begin{itemize}
			\item $\tp$ is a template.
			\item $occ(\tp) = \{ \tp \}$.
		\end{itemize}
		
		
		
		\item 
		\begin{itemize}
			\item If $F$ is a template, then $\nao F$ is a template. 
			\item $occ(\nao F) = occ(F)$.
		\end{itemize}
		
		
		
		\item 
		\begin{itemize}
			\item If $F$ and $G$ are templates and if  $occ(F)\cap occ(G) =\vazio$, then $F \ou G$ is a template. 
			\item $occ(F\ou G) = occ(F)\cup occ(G)$.
		\end{itemize}
		
		
		
		\item 
		\begin{itemize}
			\item If $F$ and $G$ are templates and if  $occ(F)\cap occ(G) =\vazio$, then $F \e G$ is a template. 
			\item $occ(F\e G) = occ(F)\cup occ(G)$.
		\end{itemize}
		
		
		\item 
		\begin{itemize}
			\item If $F$ is a template, then $\Box F$ is a template. 
			\item $occ(\Box F) = occ(F)$.
		\end{itemize}
		
	\end{enumerate}
\end{defn}


\qquad Similarly as in the case when we work with formulas, we can define the notion of \textit{complexity} of a template (the number of occurrences of boolean and modal connectives). So we shall define some notions recursively based on the complexity of templates  and prove some facts by induction on the complexity of templates.


\begin{defn} 
	Let $\tvp$ be an $n$-ary sequence of propositional variables, $\vvarphi$ be an $n$-ary sequence of $\D$-formulas and    $F(\tvp)$ a template. We define the \textit{instantiation set} $\Arrowvert F(\vvarphi) \Arrowvert$ recursively as follows:
	
	\begin{enumerate}[a)]
		
		\item If $F(\tvp)$ is $\tp_{i}$, then $\Arrowvert F(\vvarphi) \Arrowvert = \{ \varphi_{i}\}$.
		
		\item If $F(\tvp)$ is $\nao G(\tvp)$, then  $\Arrowvert F(\vvarphi) \Arrowvert = \{ \nao \psi$ $|$ $\psi \in  \Arrowvert G(\vvarphi) \Arrowvert   \}$.
		
		\item If $F(\tvp)$ is $G(\tvp) \ou H(\tvp)$, then  $\Arrowvert F(\vvarphi) \Arrowvert = \{ \psi\ou \theta $ $|$ $\psi \in  \Arrowvert G(\vvarphi) \Arrowvert$ and $\theta \in  \Arrowvert H(\vvarphi) \Arrowvert  \}$.
		
		\item If $F(\tvp)$ is $G(\tvp) \e H(\tvp)$, then   $\Arrowvert F(\vvarphi) \Arrowvert = \{ \psi\e \theta $ $|$ $\psi \in  \Arrowvert G(\vvarphi) \Arrowvert$ and $\theta \in  \Arrowvert H(\vvarphi) \Arrowvert  \}$.
		
		
		\item If $F(\tvp)$ is  $\Box G(\tvp)$, then   $\Arrowvert F(\vvarphi) \Arrowvert = \{ t$$:_{X}$$\psi$ $|$ $\psi \in  \Arrowvert G(\vvarphi) \Arrowvert   \}$; where $t$ is a justification term of $\Fj$ and $X$ is the set of all witness variables occurring in $\psi$.
	\end{enumerate}
\end{defn}

\qquad Clearly, for every template $F(\tvp)$ and every sequence $\vvarphi$ of $\D$-formulas, $\Arrowvert F(\vvarphi) \Arrowvert$ is a set of $\D$-formulas.  


\begin{defn}
	We say that the template $F$ is \textit{positive} if all the  boolean connectives that occur in $F$ are $\e$ and $\ou$. Similarly, we say that $F$ is \textit{disjunctive} if all the boolean connectives that occur in $F$ are $\ou$.           
\end{defn}


\qquad From now to the end of this subsection we shall prove some facts about templates. We are always assuming that there is a fixed constant specification variant closed and axiomatically appropriate $\C$ for the basic language, and that $\Cv$ is its extension. To make things simple, we will not refer to this assumption in every proposition and, in this subsection only, we shall write `$\teo$' to denote `$\teocv$', `consistent' to denote `$\Cv$-consistent', `inconsistent' to denote `$\Cv$-inconsistent' and `maximal-consistent' to denote `$\Cv$-maximal consistent'.



\begin{pro}(\textit{Semi-Replacement})
	Let $F(\tvp,\tq)$ be a positive template,  $\varphi$ and $\psi$ $\D$-formulas, and $\vvarphi$ a sequence of $\D$-formulas. In these conditions, if $\teo \varphi \impli \psi$, then for every $\phi \in \Arrowvert F(\vvarphi,\varphi) \Arrowvert$ there is a $\theta \in \Arrowvert F(\vvarphi,\psi) \Arrowvert$ such that
	
	
	\begin{center}
		$\teo \phi \impli \theta$
	\end{center}    
\end{pro}

\begin{proof} (Induction on the complexity of $F(\tvp,\tq)$).\\
	
	
	($F(\tvp,\tq)$ is atomic)\\
	
	\qquad i) $F(\tvp,\tq) = \tp_{i}$. Then for any $\phi \in \Arrowvert F(\vvarphi,\varphi) \Arrowvert = \{ \varphi_{i}\}$, $\phi = \varphi_{i}$. Since $\varphi_{i} \in  \Arrowvert F(\vvarphi,\psi) \Arrowvert = \{ \varphi_{i}\}$, take $\theta$ as $\varphi_{i}$. 
	
	\qquad ii) $F(\tvp,\tq) = \tq$. Then for any $\phi \in \Arrowvert F(\vvarphi,\varphi) \Arrowvert = \{ \varphi\}$, $\phi = \varphi$. Since $\psi \in  \Arrowvert F(\vvarphi,\psi) \Arrowvert = \{ \psi\}$, take $\theta$ as $\psi$. \\
	
	
	
	($F(\tvp,\tq)$ is $G(\tvp, \tq)\ou H(\tvp, \tq)$)\\
	
	\qquad Let $\phi \in \Arrowvert F(\vvarphi,\varphi) \Arrowvert$. So $\phi$ is $\phi\p \ou \phi\pp$ where  $\phi\p \in \Arrowvert G(\vvarphi,\varphi) \Arrowvert$ and  $\phi\pp \in \Arrowvert H(\vvarphi,\varphi) \Arrowvert$. By the induction hypothesis, there are $\theta\p \in \Arrowvert G(\vvarphi,\psi) \Arrowvert$ and  $\theta\pp \in \Arrowvert H(\vvarphi,\psi) \Arrowvert$ such that
	
	\begin{center}
		$\teo \phi\p \impli \theta\p$ and $\teo \phi\pp \impli \theta\pp$
	\end{center}
Hence,
	
	\begin{center}
		$\teo \phi\p \ou \phi\pp  \impli \theta\p \ou \theta\pp$.
	\end{center}
	
	\qquad Since $\theta\p \ou \theta\pp \in \Arrowvert F(\vvarphi,\psi) \Arrowvert$, take $\theta$ as $\theta\p \ou \theta\pp$. \\
	
	\qquad If $F(\tvp,\tq)$ is $G(\tvp, \tq)\e H(\tvp, \tq)$, then the argument is similar to the previous one.\\
	
	($F(\tvp,\tq)$ is $\Box G(\tvp, \tq)$)\\
	
	
	\qquad Let $\phi \in \Arrowvert F(\vvarphi,\varphi) \Arrowvert$. So $\phi$ is $t$$:_{X}$$\phi\p$ where $\phi\p \in \Arrowvert G(\vvarphi,\varphi) \Arrowvert$. By the induction hypothesis, there is a $\theta\p \in \Arrowvert G(\vvarphi,\psi) \Arrowvert$ such that $\teo \phi\p \impli \theta\p$. By Proposition 21, there is a justification term $s$ of $\Fj$ such that
	
	\begin{center}
		$\teo s$$:$$(\phi\p \impli \theta\p)$
	\end{center}
By repeated use of axiom \textbf{A3} and classical reasoning  
	
	\begin{center}
		$\teo s$$:_{X}$$(\phi\p \impli \theta\p)$
	\end{center}
By axiom \textbf{B2} and modus ponens 
	
	\begin{center}
		$\teo t$$:_{X}$$\phi\p \impli [s\cdot t]$$:_{X}$$ \theta\p$.
	\end{center}
	
	
	\qquad Let $Y$ be the set of all witness variables that occur in $\theta\p$. By repeated use of axioms \textbf{A2} and \textbf{A3}, we have that
	
	
	\begin{center}
		$\teo [s\cdot t]$$:_{X}$$ \theta\p \impli [s\cdot t]$$:_{Y}$$ \theta\p$
	\end{center} 
Hence, 
	
	\begin{center}
		$\teo t$$:_{X}$$\phi\p \impli [s\cdot t]$$:_{Y}$$ \theta\p$.
	\end{center} 
	
	\qquad Since $[s\cdot t]$$:_{Y}$$ \theta\p \in \Arrowvert F(\vvarphi,\psi) \Arrowvert$, take $\theta$ as $[s\cdot t]$$:_{Y}$$ \theta\p$.
	
\end{proof}




\begin{coro}(\textit{Variable Change})
	Let $\Gamma \subseteq \Fjv$, $F(\tvp,\tq)$ a positive template, $\vvarphi$ a sequence of $\D$-formulas, $\todo x \varphi(x)$ a $\D$-formula, and $y$ a basic variable that does not occur free in $\todo x \varphi(x)$. In these conditions, if $\Gamma \cup \Arrowvert \nao F(\vvarphi,\todo x\varphi(x)) \Arrowvert$ is consistent, then $\Gamma \cup \Arrowvert \nao F(\vvarphi,\todo y\varphi(y)) \Arrowvert$ is consistent.     
\end{coro}



\begin{proof} Suppose that $\Gamma \cup \Arrowvert \nao F(\vvarphi,\todo x\varphi(x)) \Arrowvert$ is consistent and  $\Gamma \cup \Arrowvert \nao F(\vvarphi,\todo y\varphi(y)) \Arrowvert$ is inconsistent. Then, there are $\psi_{1}, \dots, \psi_{n} \in \Arrowvert F(\vvarphi,\todo y\varphi(y)) \Arrowvert$ such that
	
	
	
	\begin{center}
		$\Gamma \teo \psi_{1} \ou \dots \ou \psi_{n}$
	\end{center} 
By classical logic,
	
	
	\begin{center}
		$\teo \todo y\varphi(y) \impli \todo x\varphi(x) $.
	\end{center} 
	
	
	\qquad Hence by Proposition 23, for each $\psi_{i}$ there is a $\theta_{i} \in  \Arrowvert F(\vvarphi,\todo x\varphi(x)) \Arrowvert$
	such that     
	
	\begin{center}
		$\teo \psi_{i} \impli \theta_{i}$
	\end{center} 
Thus,
	
	\begin{center}
		$\Gamma \teo \theta_{1} \ou \dots \ou \theta_{n}$.
	\end{center} 
	
	\qquad And since each $\nao \theta_{i} \in  \Arrowvert \nao F(\vvarphi,\todo x\varphi(x)) \Arrowvert$, $\Gamma \cup \Arrowvert \nao F(\vvarphi,\todo x\varphi(x)) \Arrowvert$ is inconsistent; a contradiction.
\end{proof}

\begin{pro}(\textit{Vacuous Quantification})
	Let $F(\tvp)$ be a disjunctive template, and $\vvarphi$ a sequence of $\D$-formulas none of which contain free occurrences of the basic variable $y$. In these conditions, for each $\psi \in \Arrowvert F(\vvarphi) \Arrowvert$ there is some $\theta \in \Arrowvert F(\vvarphi) \Arrowvert$ such that
	
	
	\begin{center}
		$\teo \ex y\psi \impli \theta$
	\end{center}    
\end{pro}

\begin{proof} (Induction on the complexity of $F(\tvp)$)\\
	
	
	($F(\tvp)$ is $\tp_{i}$)\\
	
	\qquad  For each $\psi \in \Arrowvert F(\vvarphi) \Arrowvert =\{ \varphi_{i}  \}$,  $\psi = \varphi_{i}$. Since $y$ does not occur free in $\varphi_{i}$, $\teo \ex y\varphi_{i} \impli \varphi_{i}$. We can take $\theta$ as $\varphi_{i}$.\\
	
	($F(\tvp)$ is $G(\tvp)\ou H(\tvp)$)\\
	
	\qquad Let $\psi \in \Arrowvert F(\vvarphi) \Arrowvert$. So $\psi$ is $\psi\p \ou \psi\pp$ where  $\psi\p \in \Arrowvert G(\vvarphi) \Arrowvert$ and  $\psi\pp \in \Arrowvert H(\vvarphi) \Arrowvert$. By the induction hypothesis, there are $\theta\p \in \Arrowvert G(\vvarphi) \Arrowvert$ and  $\theta\pp \in \Arrowvert H(\vvarphi) \Arrowvert$ such that
	
	\begin{center}
		$\teo \ex y \psi\p \impli \theta\p$ and $\teo \ex y\psi\pp \impli \theta\pp$
	\end{center}    
By classical logic,
	
	
	\begin{center}
		$\teo \ex y (\psi\p \ou \psi\pp) \see  (\ex y\psi\p \ou \ex y\psi\pp)$
	\end{center}     
Hence,
	
	\begin{center}
		$\teo \ex y (\psi\p \ou \psi\pp) \impli \theta\p \ou \theta\pp$.
	\end{center}
	
	
	\qquad Since $\theta\p \ou \theta\pp \in \Arrowvert F(\vvarphi) \Arrowvert$, take $\theta$ as $\theta\p \ou \theta\pp$.\\
	
	
	
	($F(\tvp)$ is $\Box G(\tvp)$)\\
	
	
	\qquad Let $\psi \in \Arrowvert F(\vvarphi) \Arrowvert$. So $\psi$ is $t$$:_{X}$$\phi$ where $\phi \in \Arrowvert G(\vvarphi) \Arrowvert$. By the axiom \textbf{A3}, 
	
	\begin{center}
		$\teo t$$:_{X}$$\phi \impli t$$:_{Xy}$$\phi$
	\end{center}
By classical logic,
	
	
	\begin{center}
		$\teo \ex y t$$:_{X}$$\phi \impli \ex y t$$:_{Xy}$$\phi$.
	\end{center}     
	
	\qquad By definition, $X$ is a set of witness variables and since $y$ is a basic variable we have that $y \notin X$; so by Proposition 18,
	
	\begin{center}
		$\teo \ex y t$$:_{Xy}$$\phi \impli s(t)$$:_{X}$$\ex y \phi$
	\end{center}
	
	
	
	\qquad By induction hypothesis, there is a $\theta\p \in \Arrowvert G(\vvarphi) \Arrowvert$ such that $\teo \ex y \phi \impli \theta\p$. By Proposition 21 and by the axiom \textbf{A3}, there is a justification term $s\p$ of $\Fj$ such that
	
	\begin{center}
		$\teo s\p$$:_{X}$$(\ex y \phi \impli \theta\p)$
	\end{center}    
By axiom \textbf{B2},    
	
	
	\begin{center}
		$\teo s(t)$$:_{X}$$\ex y \phi \impli [s\p \cdot s(t)]$$:_{X}$$\theta\p$.
	\end{center}
	
	\qquad Let $Y$ be the set of all witness variables that occur in $\theta\p$. By repeated use of axioms \textbf{A2} and \textbf{A3}, we have that
	
	
	\begin{center}
		$\teo [s\p \cdot s(t)]$$:_{X}$$\theta\p \impli  [s\p \cdot s(t)]$$:_{Y}$$\theta\p$
	\end{center} 
Hence, 
	
	\begin{center}
		$\teo \ex y t$$:_{X}$$\phi \impli [s\p \cdot s(t)]$$:_{Y}$$\theta\p$.
	\end{center}
	
	
	\qquad Since $[s\p \cdot s(t)]$$:_{Y}$$\theta\p \in \Arrowvert F(\vvarphi) \Arrowvert$, take $\theta$ as $[s\p \cdot s(t)]$$:_{Y}$$\theta\p$.\\    
	
\end{proof}


\begin{pro}(\textit{Generalized Barcan})
	Let $F(\tvp,\tq)$ be a disjunctive template, $y$ a basic variable, $\varphi(y)$ a $\D$-formula, and  $\vvarphi$ a sequence of $\D$-formulas none of which contain free occurrences of $y$. In these conditions, for each $\psi \in \Arrowvert F(\vvarphi,\varphi(y)) \Arrowvert$ there is some $\theta \in \Arrowvert F(\vvarphi,\todo y \varphi(y)) \Arrowvert$ such that
	
	
	\begin{center}
		$\teo \todo y\psi \impli \theta$
	\end{center}    
\end{pro}

\begin{proof} (Induction on the complexity of $F(\tvp,\tq)$)\\
	
	\qquad If $F(\tvp,\tq)$ is atomic, then the result is trivial.\\
	
	($F(\tvp,\tq)$ is $G(\tvp,\tq)\ou H(\tvp,\tq)$)\\
	
	
	\qquad By the definition of template, the propositional variable $\tq$ can occur at most once in $F(\tvp,\tq)$. So either it does not occur in $G(\tvp,\tq)$ or it does not occur in  $H(\tvp,\tq)$. Assume that it does not occur in $H(\tvp,\tq)$ (the other case is symmetric); then we can assume that $H(\tvp,\tq)$ is $H(\tvp)$.    
	
	\qquad Let $\psi \in \Arrowvert F(\vvarphi,\varphi(y)) \Arrowvert$. So $\psi$ is $\phi\p \ou \phi\pp$ where  $\phi\p \in \Arrowvert G(\vvarphi,\varphi(y)) \Arrowvert$ and  $\phi\pp \in \Arrowvert H(\vvarphi) \Arrowvert$. By classical logic, we have that 
	
	\begin{center}    
		$\teo \todo y (\phi\p \ou \phi\pp) \impli (\todo y \phi\p \ou \ex y \phi\pp)$
	\end{center}
	
	\qquad Since $y$ does not occur free in any formula of $\vvarphi$, then by Proposition 24 there is some $\theta\pp \in \Arrowvert H(\vvarphi) \Arrowvert$ such that 
	
	
	\begin{center}    
		$\teo \ex y \phi\pp \impli \theta\pp$
	\end{center}
	
	\qquad By the induction hypothesis, there is $\theta\p \in \Arrowvert G(\vvarphi,\todo y \varphi (y)) \Arrowvert$ such that
	
	\begin{center}
		$\teo \todo y \psi\p \impli \theta\p$ 
	\end{center}    
Hence,    
	
	\begin{center}    
		$\teo \todo y (\phi\p \ou \phi\pp) \impli \theta\p \ou \theta\pp$.
	\end{center}    
	
	
	\qquad And so we can take $\theta$ as $\theta\p \ou \theta\pp$.    \\
	
	
	($F(\tvp,\tq)$ is $\Box G(\tvp,\tq)$)\\
	
	\qquad Let $\psi \in \Arrowvert F(\vvarphi, \varphi(y)) \Arrowvert$. So $\psi$ is $t$$:_{X}$$\phi$ where $\phi \in \Arrowvert G(\vvarphi,\varphi(y)) \Arrowvert$. By definition, $X$ is a set of witness variables, then $y \notin X$. So, by Proposition 17 
	
	
	\begin{center}
		$\teo \todo y t$$:_{Xy}$$\phi \impli B(t)$$:_{X}$$\todo y \phi$
	\end{center}
By axiom \textbf{A3},
	
	
	\begin{center}
		$\teo t$$:_{X}$$\phi \impli t$$:_{Xy}$$\phi$
	\end{center}
By classical logic,
	
	
	\begin{center}
		$\teo \todo y t$$:_{X}$$\phi \impli \todo y t$$:_{Xy}$$\phi$
	\end{center}
So,
	
	\begin{center}
		$\teo \todo y t$$:_{X}$$\phi \impli B(t)$$:_{X}$$\todo y \phi$.
	\end{center}
	
	\qquad By the induction hypothesis, there is a $\theta\p \in \Arrowvert G(\vvarphi, \todo y \varphi(y)) \Arrowvert$ such that $\teo \todo y \phi \impli \theta\p$. By Proposition 21 and by the axiom \textbf{A3}, there is a justification term $s$ of $\Fj$ such that
	
	\begin{center}
		$\teo s$$:_{X}$$(\todo y \phi \impli \theta\p)$
	\end{center}    
By axiom \textbf{B2},    
	
	
	\begin{center}
		$\teo B(t)$$:_{X}$$\todo y \phi \impli [s \cdot B(t)]$$:_{X}$$\theta\p$.
	\end{center}
	
	\qquad Let $Y$ be the set of all witness variables that occur in $\theta\p$. By repeated use of axioms \textbf{A2} and \textbf{A3}, we have that
	
	
	\begin{center}
		$\teo [s \cdot B(t)]$$:_{X}$$\theta\p \impli  [s \cdot B(t)]$$:_{Y}$$\theta\p$
	\end{center} 
Hence, 
	
	\begin{center}
		$\teo \todo y t$$:_{X}$$\phi \impli [s \cdot B(t)]$$:_{Y}$$\theta\p$.
	\end{center}
	
	\qquad Take $\theta$ as $[s \cdot B(t)]$$:_{Y}$$\theta\p$.\\    
	
	
	
	
	
\end{proof}


\begin{pro}(\textit{Formula Combining}) Let $F(\tvp)$ be a disjunctive template, and $\vvarphi$ a sequence of $\D$-formulas. In these conditions, for any $\psi_{1}, \dots, \psi_{k}  \in \Arrowvert F(\vvarphi) \Arrowvert$ there is some formula $\theta \in \Arrowvert F(\vvarphi) \Arrowvert$ such that
	
	
	\begin{center}
		$\teo (\psi_{1} \ou \dots \ou \psi_{k}) \impli \theta$
	\end{center}    
\end{pro}

\begin{proof} (Induction on the complexity of $F(\tvp)$.)\\
	
	\qquad If $F(\tvp)$ is atomic, then the result is trivial.\\
	

	
	($F(\tvp)$ is $G(\tvp)\ou H(\tvp)$)\\
	
	\qquad Let $\psi_{1}, \dots, \psi_{k} \in \Arrowvert F(\vvarphi) \Arrowvert$. So there are $\phi_{1}^{\p}, \dots, \phi_{k}^{\p} \in \Arrowvert G(\vvarphi) \Arrowvert$ and  $\phi_{1}^{\pp}, \dots, \phi_{k}^{\pp} \in \Arrowvert H(\vvarphi) \Arrowvert$, such that $\psi_{i} = \phi_{i}^{\p} \ou \phi_{i}^{\pp}$. By the induction hypothesis, there are $\theta^{\p} \in \Arrowvert G(\vvarphi) \Arrowvert$ and $\theta^{\pp} \in \Arrowvert H(\vvarphi) \Arrowvert$ such that  
	
	
	\begin{center}
		$\teo (\phi_{1}^{\p} \ou \dots \ou \phi_{k}^{\p}) \impli \theta\p$\\
		$\teo (\phi_{1}^{\pp} \ou \dots \ou \phi_{k}^{\pp}) \impli \theta\pp$
	\end{center}
Hence, 
	
	
	\begin{center}
		$\teo ((\phi_{1}^{\p} \ou \dots \ou \phi_{k}^{\p}) \ou (\phi_{1}^{\pp} \ou \dots \ou \phi_{k}^{\pp})) \impli \theta\p \ou \theta\pp$
	\end{center}
And so, 
	
	\begin{center}
		$\teo ((\phi_{1}^{\p} \ou \phi_{1}^{\pp})  \ou \dots \ou (\phi_{k}^{\p} \ou \phi_{k}^{\pp}))  \impli \theta\p \ou \theta\pp$
	\end{center}
i.e.,
	
	\begin{center}
		$\teo (\psi_{1} \ou \dots \ou \psi_{k}) \impli \theta\p \ou \theta\pp$.
	\end{center}
	
	
	\qquad Take $\theta$ as $\theta\p \ou \theta\pp$.\\    
	
\pagebreak
	
	($F(\tvp)$ is $\Box G(\tvp)$)\\    
	
	\qquad Let $\psi_{1}, \dots, \psi_{k} \in \Arrowvert F(\vvarphi) \Arrowvert$. So there are justification terms $t_{1}, \dots, t_{k}$ and $\phi_{1}, \dots, \phi_{k} \in \Arrowvert G(\vvarphi) \Arrowvert$ such that $\psi_{i} =     t_{i}$$:_{X_{i}}$$\phi_{i}$.
	By the induction hypothesis, there is $\theta^{\p} \in \Arrowvert G(\vvarphi) \Arrowvert$ such that  
	
	
	\begin{center}
		$\teo (\phi_{1} \ou \dots \ou \phi_{k}) \impli \theta\p$\
	\end{center}
Hence, by classical reasoning, for each $i$,
	
	\begin{center}
		$\teo \phi_{i} \impli \theta\p$\
	\end{center}
	
\qquad So, by Proposition 21 and by the axiom \textbf{A3} there are  justification terms $s_{1}, \dots, s_{k}$ of $\Fj$ such that for each $i$,
	
	
	\begin{center}
		$\teo s_{i}$$:_{X_{i}}$$(\phi_{i} \impli \theta\p)$\
	\end{center}
By axiom \textbf{B2},
	
	
	\begin{center}
		$\teo t_{i}$$:_{X_{i}}$$\phi_{i}  \impli [s_{i} \cdot t_{i}]$$:_{X_{i}}$$\theta\p$\
	\end{center}
By an appropriate use of axiom \textbf{B3}, we have that for each $i$, 
	
	
	\begin{center}
		$\teo [s_{i} \cdot t_{i}]$$:_{X_{i}}$$\theta\p  \impli [[s_{1} \cdot t_{1}] +$ $\dots$ $+ [s_{k} \cdot t_{k}]]$$:_{X_{i}}$$\theta\p$.
	\end{center}
	
	
	\qquad Let $Y$ be the set of all witness variables that occur in $\theta\p$. By repeated use of axioms \textbf{A2} and \textbf{A3}, we have that
	
	
	\begin{center}
		$\teo [[s_{1} \cdot t_{1}] +$ $\dots$ $+ [s_{k} \cdot t_{k}]]$$:_{X_{i}}$$\theta\p  \impli [[s_{1} \cdot t_{1}] +$ $\dots$ $+ [s_{k} \cdot t_{k}]]$$:_{Y}$$\theta\p$
	\end{center}
Hence, for each $i$, 
	
	\begin{center}
		$\teo t_{i}$$:_{X_{i}}$$\phi_{i}  \impli  [[s_{1} \cdot t_{1}] +$ $\dots$ $+ [s_{k} \cdot t_{k}]]$$:_{Y}$$\theta\p$
	\end{center}
So,
	
	\begin{center}
		$\teo (t_{1}$$:_{X_{1}}$$\phi_{1} \ou \dots \ou t_{k}$$:_{X_{k}}$$\phi_{k} )   \impli  [[s_{1} \cdot t_{1}] +$ $\dots$ $+ [s_{k} \cdot t_{k}]]$$:_{Y}$$\theta\p$.
	\end{center}
	
	\qquad Since $[[s_{1} \cdot t_{1}] +$ $\dots$ $+ [s_{k} \cdot t_{k}]]$$:_{Y}$$\theta\p \in \Arrowvert F(\vvarphi) \Arrowvert$, we can take $\theta$ as $[[s_{1} \cdot t_{1}] +$ $\dots$ $+ [s_{k} \cdot t_{k}]]$$:_{Y}$$\theta\p$.
\end{proof}

\begin{pro}(\textit{Existential Instantiation})
	Let $F(\tvp,\tq)$ be a disjunctive template, $\Gamma \subseteq \Fj$, $\vvarphi$ a sequence of $\D$-formulas,  $\todo x \varphi(x)$ a $\D$-formula, and $a$ a witness variable that does not occur free in $\todo x \varphi(x)$ and in any member of $\vvarphi$. In these conditions, if $\Gamma \cup \Arrowvert \nao F(\vvarphi,\todo x\varphi(x)) \Arrowvert$ is consistent, then $\Gamma \cup \Arrowvert \nao F(\vvarphi,\varphi(a)) \Arrowvert$ is consistent.    
\end{pro}

\begin{proof} 
	Suppose that  $\Gamma \cup \Arrowvert \nao F(\vvarphi,\todo x\varphi(x)) \Arrowvert$ is consistent and $\Gamma \cup \Arrowvert \nao F(\vvarphi,\varphi(a)) \Arrowvert$ is inconsistent.  Then, there are $\psi_{1}, \dots, \psi_{n} \in \Gamma$ and $\nao \phi_{1}(a), \dots, \nao \phi_{k}(a) \in \Arrowvert \nao F(\vvarphi,\varphi(a)) \Arrowvert$ such that
	
	
	
	\begin{center}
		$\teo(\psi_{1} \e \dots \e \psi_{n}) \e (\nao \phi_{1}(a) \e \dots \e \nao \phi_{k}(a)) \impli \bot$
	\end{center} 
Hence,
	
	
	\begin{center}
		$\teo(\psi_{1} \e \dots \e \psi_{n}) \impli (\phi_{1}(a) \ou \dots \ou \phi_{k}(a))$.
	\end{center} 
	
	
\qquad By Proposition 26 there is a $\psi(a) \in \Arrowvert F(\vvarphi,\varphi(a)) \Arrowvert$, such that
	
	\begin{center}
		$\teo (\phi_{1}(a) \ou \dots \ou \phi_{k}(a)) \impli \psi(a)$
	\end{center} 
Hence,
	
	
	\begin{center}
		$\teo (\psi_{1} \e \dots \e \psi_{n}) \impli \psi(a)$
	\end{center}
By generalization (remember, $a$ is a variable in the new language),
	
	\begin{center}
		$\teo \todo a [(\psi_{1} \e \dots \e \psi_{n}) \impli \psi(a)]$.
	\end{center}
	
\qquad Let $y$ be a basic variable that does not occur in $\psi_{1}, \dots,\psi_{n}$, $\todo x \varphi (x)$, $\vvarphi$, $\varphi (a)$ and $\psi(a)$. By classical logic,  
	
	
	
	\begin{center}
		$\teo \todo a [(\psi_{1} \e \dots \e \psi_{n}) \impli \psi(a)] \impli [(\psi_{1} \e \dots \e \psi_{n}) \impli \psi(a)](y/a)$.
	\end{center}
	
	\qquad Since $\Gamma$ is a set of basic formulas, $a$ does not occur in any formula of $\Gamma$; in particular, $a$ does not occur in any $\psi_{i}$. Hence $[(\psi_{1} \e \dots \e \psi_{n}) \impli \psi(a)](y/a)$ is $(\psi_{1} \e \dots \e \psi_{n}) \impli \psi(y)$. So, by modus ponens and generalization, 
	
	\begin{center}
		$\teo \todo y [(\psi_{1} \e \dots \e \psi_{n}) \impli \psi(y)]$.
	\end{center}
	

	\qquad Since $y$ does not occur in any $\psi_{i}$, by classical reasoning,
	
	\begin{center}
		$\teo (\psi_{1} \e \dots \e \psi_{n}) \impli \todo y \psi(y)$
	\end{center}
	
	
	\qquad Since $a$ does not occur free in any formula of $\vvarphi$, it can be easily checked that for every formula $\psi(a)$,
	
	\begin{center}
		If $\psi(a) \in \Arrowvert F(\vvarphi,\varphi(a)) \Arrowvert$, then $\psi(y) \in \Arrowvert F(\vvarphi,\varphi(y)) \Arrowvert$.
	\end{center}
	
	\qquad By this fact, we have that $\psi(y) \in \Arrowvert F(\vvarphi,\varphi(y)) \Arrowvert$. Now since $y$ does not occur in $\vvarphi$, then by Proposition 25 there is a $\theta \in \Arrowvert F(\vvarphi,\todo y \varphi(y)) \Arrowvert$ such that
	\begin{center}
		$\teo \todo y\psi \impli \theta$
	\end{center}
Thus,
	
	\begin{center}
		$\teo  (\psi_{1} \e \dots \e \psi_{n})  \impli \theta$.
	\end{center}
	
	
	\qquad Since $\nao\theta \in \Arrowvert \nao F(\vvarphi,\todo y \varphi(y)) \Arrowvert$, it follows that $\Gamma \cup \Arrowvert \nao F(\vvarphi,\todo y\varphi(y)) \Arrowvert$ is inconsistent. By Corollary 1, $\Gamma \cup \Arrowvert \nao F(\vvarphi,\todo x\varphi(x)) \Arrowvert$ is inconsistent, a contradiction.
\end{proof}


\begin{defn}
	If $\Gamma \subseteq \Fjv$, then let $\Gamma^{\#}$ be the set of all formulas $\todo \vec{y} \varphi$ such that $t$$:_{X}$$\varphi \in \Gamma$, where $t$$:_{X}$$\varphi$ is a closed $\D$-formula with $X$ being the set of witness variables in $\varphi$, and $\vec{y}$ are the free  basic variables of $\varphi$. 
\end{defn}

\begin{pro}\textit{(Up and Down Consistency})
	Let $F(\tvp) = \Box G(\tvp)$ be a template, $\Gamma \subseteq \Fjv$,  and $\vvarphi$ a sequence of $\D$-formulas.
	
	\begin{enumerate}[1)]
		\item Suppose $\Gamma$ is maximal consistent. In these conditions, if $\Gamma^{\#} \cup \Arrowvert \nao G(\vvarphi) \Arrowvert$ is consistent, then  $\Gamma \cup \Arrowvert \nao F(\vvarphi) \Arrowvert$ is consistent.
		\item Suppose $G(\tvp)$ is a disjunctive template. In these conditions,  if $\Gamma \cup \Arrowvert \nao F(\vvarphi) \Arrowvert$ is consistent, then  $\Gamma^{\#} \cup \Arrowvert \nao G(\vvarphi) \Arrowvert$ is consistent.
	\end{enumerate}    
\end{pro}

\begin{proof} 
	
	1) Suppose $\Gamma^{\#} \cup \Arrowvert \nao G(\vvarphi) \Arrowvert$ is consistent and  $\Gamma \cup \Arrowvert \nao F(\vvarphi) \Arrowvert$ is inconsistent. Then, for some $\nao t_{1}$$:_{X_{1}}$$\theta_{1}, \dots,  \nao t_{k}$$:_{X_{k}}$$\theta_{k} \in  \Arrowvert \nao F(\vvarphi) \Arrowvert$ (where $\theta_{1}, \dots, \theta_{k} \in  \Arrowvert  G(\vvarphi) \Arrowvert$)
	
	
	\begin{center}
		$\Gamma \teo (\nao t_{1}$$:_{X_{1}}$$\theta_{1} \e \dots \e  \nao t_{k}$$:_{X_{k}}$$\theta_{k}) \impli \bot $
	\end{center}    
Hence,
	
	\begin{center}
		$\Gamma \teo t_{1}$$:_{X_{1}}$$\theta_{1} \ou \dots \ou   t_{k}$$:_{X_{k}}$$\theta_{k}$.
	\end{center}        
	
	
\qquad Now, since $\Gamma$ is maximal consistent set, for some $i$, $t_{i}$$:_{X_{i}}$$\theta_{i} \in \Gamma$. And since $t_{i}$$:_{X_{i}}$$\theta_{i}$ is a closed $\D$-formula, $\todo \vec{x} \theta_{i} \in \Gamma^{\#}$. By classical logic, 
	
	\begin{center}
		$\teo \nao \theta_{i} \impli \nao \todo \vec{x} \theta_{i}$
	\end{center}        
	
\qquad Since $\nao \theta_{i} \in \Arrowvert  \nao G(\vvarphi) \Arrowvert$, we have that  $\Gamma^{\#} \cup \Arrowvert \nao G(\vvarphi) \Arrowvert$ is inconsistent, a contradiction.\\
	
	
\qquad 2) Suppose $\Gamma \cup \Arrowvert \nao F(\vvarphi) \Arrowvert$ is consistent and $\Gamma^{\#} \cup \Arrowvert \nao G(\vvarphi) \Arrowvert$ is inconsistent. Then, there are $\todo \vec{x}_{1} \psi_{1}, \dots, \todo \vec{x}_{n} \psi_{n} \in \Gamma^{\#}$ (where $t_{1}$$:_{X_{1}}$$\psi_{1}, \dots,  t_{n}$$:_{X_{n}}$$\psi_{n} \in  \Gamma$) and $\nao \theta_{1}, \dots, \nao \theta_{k} \in \Arrowvert  \nao G(\vvarphi) \Arrowvert$ such that
	
	
	\begin{center}
		$\teo (\todo \vec{x}_{1} \psi_{1} \e \dots \e \todo \vec{x}_{n} \psi_{n}) \e  (\nao \theta_{1}\e \dots \e \nao \theta_{k}) \impli \bot $
	\end{center}
So,
	
	\begin{center}
		$\teo (\todo \vec{x}_{1} \psi_{1} \e \dots \e \todo \vec{x}_{n} \psi_{n}) \impli  (\theta_{1} \ou \dots \ou  \theta_{k})$.
	\end{center}
	
	\qquad Since $\theta_{1}, \dots, \theta_{k} \in \Arrowvert G(\vvarphi) \Arrowvert$ and $G(\tvp)$ is a disjunctive template, then by Proposition 26 there is a $\theta \in \Arrowvert G(\vvarphi) \Arrowvert$ such that 
	
	\begin{center}
		$\teo (\theta_{1} \ou \dots \ou  \theta_{k}) \impli \theta$
	\end{center}
By classical logic,
	
	\begin{center}
		$\teo \todo \vec{x}_{1} \psi_{1} \impli \dots \impli \todo \vec{x}_{n} \psi_{n} \impli  \theta$.
	\end{center}
	
	\qquad Now, for each $i$ any member of the sequence $\vec{x}_{i}$ does not occur in the set $X_{i}$ (the set $X_{i}$ is a set of witness variables). So, by repeated use of axiom \textbf{B6} we have that for each $i$
	
	\begin{center}
		$\teo t_{i}$$:_{X_{i}}$$\psi_{i}\impli gen_{\vec{x}_{i}}(t)$$:_{X_{i}}$$\todo \vec{x}\psi_{i}$
	\end{center}
	
	\qquad It should be noted that `$gen_{\vec{x}_{i}}(t)$' is not a justification term, it is just an abbreviation that we use to help readability. Let $X = X_{1}\cup$ $\dots$ $\cup X_{n}$, by axiom \textbf{A3},
	
	\begin{center}
		$\teo t_{i}$$:_{X_{i}}$$\psi_{i}\impli gen_{\vec{x}_{i}}(t)$$:_{X}$$\todo \vec{x}\psi_{i}$
	\end{center}
	
	\qquad By Proposition 21 and axiom \textbf{A3} there is a justification term $s$ of $\Fj$ such that 
	
	\begin{center}
		$\teo s$$:_{X}$$(\todo \vec{x}_{1} \psi_{1} \impli \dots \impli \todo \vec{x}_{n} \psi_{n} \impli  \theta)$
	\end{center}
And by repeated use of axiom \textbf{B2}
	
	
	\begin{center}
		$\teo gen_{\vec{x}_{1}}(t)$$:_{X}$$\todo \vec{x}\psi_{1}\impli \dots \impli gen_{\vec{x}_{n}}(t)$$:_{X}$$\todo \vec{x}\psi_{n}\impli  [s\cdot gen_{\vec{x}_{1}}(t) \cdot$ $\dots$ $\cdot gen_{\vec{x}_{n}}(t) ]$$:_{X}$$\theta$.
	\end{center}
	
	\qquad Let $Y$ be the set of all witness variables of $\theta$; by axioms \textbf{A2} and \textbf{A3}
	
	\begin{center}
		$\teo [s\cdot gen_{\vec{x}_{1}}(t) \cdot$ $\dots$ $\cdot gen_{\vec{x}_{n}}(t) ]$$:_{X}$$\theta \impli  [s\cdot gen_{\vec{x}_{1}}(t) \cdot$ $\dots$ $\cdot gen_{\vec{x}_{n}}(t) ]$$:_{Y}$$\theta$
	\end{center}
By classical reasoning,
	
	\begin{center}
		$\teo (t_{1}$$:_{X_{1}}$$\psi_{1} \e \dots \e t_{n}$$:_{X_{n}}$$\psi_{n}) \impli  [s\cdot gen_{\vec{x}_{1}}(t) \cdot$ $\dots$ $\cdot gen_{\vec{x}_{n}}(t) ]$$:_{Y}$$\theta$.
	\end{center}
	
	
	\qquad Since each $t_{i}$$:_{X_{i}}$$\psi_{i} \in \Gamma$ and $[s\cdot gen_{\vec{x}_{1}}(t) \cdot$ $\dots$ $\cdot gen_{\vec{x}_{n}}(t) ]$$:_{Y}$$\theta \in \Arrowvert F(\vvarphi) \Arrowvert$ we have that $\Gamma \cup \Arrowvert \nao F(\vvarphi) \Arrowvert$ is inconsistent; a contradiction.
\end{proof}



\begin{defn}  
	A set of formulas $\Gamma$ \textit{admits instantiation} provided for each disjunctive template $F(\tvp,\tq)$, for each sequence $\vvarphi$ of $\D$-formulas, and each universally quantified $\D$-formula $\todo x \varphi (x)$, if $\Gamma \cup \Arrowvert \nao F(\vvarphi,\todo x\varphi(x)) \Arrowvert$ is consistent, then for some witness variable $a$, $\Gamma \cup \Arrowvert \nao F(\vvarphi,\varphi(a)) \Arrowvert$ is consistent.\footnote{This is the stronger version of the `$\todo$-property' that we mentioned in subsection 5.4.2.}   
\end{defn}



\begin{pro}
	Suppose $\Gamma$ is maximal consistent and $\Gamma$ admits instantiation. For every $\D$-formula $\todo x \varphi(x)$, if $\nao \todo x \varphi(x) \in \Gamma$, then there is a witness variable $a$ such that $\nao \varphi(a) \in \Gamma$.          
\end{pro}


\begin{proof} 
	
	If $\nao \todo x \varphi (x) \in \Gamma $, then $S\cup \{ \nao \todo x \varphi (x) \}$ is consistent. Let $\tq$ be a propositional letter; $F(\tq) =\tq$ is a disjunctive template. Since $\Arrowvert \nao F(\todo x\varphi(x)) \Arrowvert=  \{ \nao \todo x \varphi (x) \}$, then  $\Gamma \cup \Arrowvert \nao F(\todo x\varphi(x)) \Arrowvert$ is consistent. Since $\Gamma$ admits instantiation, there is a witness variable $a$ such that $\Gamma \cup \Arrowvert \nao F(\varphi(a)) \Arrowvert$ is consistent, i.e., $\Gamma \cup  \{ \nao \varphi (a) \}$ is consistent. By the maximality of $\Gamma$, $ \nao \varphi(a) \in \Gamma$. 
	
\end{proof}


\begin{pro}Let $\Gamma \subseteq \Fjv$. If $\Gamma$ is maximal consistent and admits instantiation, then $\Gamma^{\#}$ also admits instantiation.
\end{pro}

\begin{proof} 
	Suppose $\Gamma$ is maximal consistent, $\Gamma$ admits instantiation, $F(\tvp, \tq)$ is a disjunctive template, $\vvarphi$ is a sequence of $\D$-formulas, $\todo x \varphi(x)$ is a $\D$-formula, and $\Gamma^{\#}\cup \Arrowvert \nao F(\vvarphi,\todo x\varphi(x)) \Arrowvert$ is consistent. By item 1) of Proposition 28, $\Gamma\cup \Arrowvert \nao\Box F(\vvarphi,\todo x\varphi(x)) \Arrowvert$ is consistent.  $\Box F(\tvp, \tq)$ is also a disjunctive template. Then, since $\Gamma$ admits instantiation, for some witness variables $a$,  $\Gamma\cup \Arrowvert \nao\Box F(\vvarphi,\varphi(a)) \Arrowvert$ is consistent. By item 2) of Proposition 28, $\Gamma^{\#}\cup \Arrowvert \nao F(\vvarphi,\varphi(a)) \Arrowvert$ is consistent.       
\end{proof}


\subsection{Using templates for Henkin-like theorems}

\qquad Since the set of all templates is a countable set, the set of all disjunctives templates is also a countable set. By the same set-theoretical considerations, since $\Fjv$ is countable, the set of all sequences $\vvarphi$ of $\D$-formulas is also countable. Hence, the set of all pairs $\bl F(\tvp), \vvarphi \br$ is countable, where $F$ is a disjunctive template, $\tvp$ is $n$-ary sequence of propositional variables and $\vvarphi$ is a $n$-ary sequence of $\D$-formulas.

\qquad For this whole subsection we shall assume that the members of the set of pairs $\bl F(\tvp), \vvarphi \br$ are arranged in a sequence

\begin{center}
	$\bl F_{1}(\tvp_{1}), \vvarphi_{1} \br, \bl F_{2}(\tvp_{2}), \vvarphi_{2} \br, \bl F_{3}(\tvp_{3}), \vvarphi_{3} \br, \dots $
\end{center}  

\qquad From now on we shall refer to this sequence as the `initial sequence'. This sequence of pairs determines a corresponding sequence of instantiation sets:

\begin{center}
	$\Arrowvert F_{1}( \vvarphi_{1}) \Arrowvert, \Arrowvert F_{2}( \vvarphi_{2}) \Arrowvert, \Arrowvert F_{3}( \vvarphi_{3}) \Arrowvert, \dots $
\end{center}  


\qquad It should be noted that for two different pairs $\bl F_{i}(\tvp_{i}), \vvarphi_{i} \br$, $\bl F_{j}(\tvp_{j}), \vvarphi_{j} \br$ the corresponding instantiation sets may be the same. For example, the pairs $\bl \tp_{0}, \bl \todo x \varphi (x)\br\br$, $\bl \tp_{1}, \bl \todo x \varphi (x)\br \br$ determine the same set $\{\todo x \varphi(x)\}$. Hence there are some repetitions in the sequence of instantiation sets, but this will not cause any trouble. 




\begin{pro}(\textit{Basic expansion})
	Let $\C$ be a variant closed and axiomatically appropriate constant specification for the basic language, $\Cv$ its extension and let $\Gamma \subseteq \Fj$ be a $\C$-consistent set. In these conditions, there is a $\Gamma\p \subseteq \Fjv$ such that $\Gamma \subseteq \Gamma\p$, $\Gamma\p$ is $\Cv$-maximal consistent set and $\Gamma\p$ admits instantiation.  
\end{pro}

\begin{proof}
	We define a sequence of sets of $\Fjv$ formulas $\Gamma_{1}, \Gamma_{2}, \Gamma_{3}, \dots $ so that:
	
	\begin{itemize}
		\item $\Gamma_{n}$ is $\Cv$-consistent.
		\item $\Gamma_{n}$ is either $\Gamma$ or $\Gamma \cup \Arrowvert \nao F_{i_{1}}( \vvarphi_{i_{1}}) \Arrowvert \cup$ $\dots$ $\cup\Arrowvert \nao F_{i_{k}}( \vvarphi_{i_{k}}) \Arrowvert $.    
	\end{itemize}    
	
	\qquad First of all, $\Gamma_{1} =\Gamma$. By the remark at the end of subsection 5.4.3, $\Gamma_{1}$ is $\Cv$-consistent.  
	
	\qquad Now, suppose $\Gamma_{n}$ is constructed and it is of the form
	$\Gamma \cup \Arrowvert \nao F_{i_{1}}( \vvarphi_{i_{1}}) \Arrowvert \cup$ $\dots$ $\cup\Arrowvert \nao F_{i_{k}}( \vvarphi_{i_{k}}) \Arrowvert $ (the other case has a similar proof). Let $\bl F_{n}(\tvp_{n}), \vvarphi_{n} \br$ be the $n$\textsuperscript{th} pair of the initial sequence. If the last term of the sequence $\vvarphi_{n}$ is not a universal formula, let $\Gamma_{n+1} = \Gamma_{n}$. Otherwise, consider the following. $\vvarphi_{n}$ is of the form $\vec{\psi}, \todo x \varphi (x)$. And $F_{n}(\tvp_{n})$ is the disjunctive template $G(\vec{\tq},\tr)$ and so  $\Arrowvert\nao  F_{n}( \vvarphi_{n}) \Arrowvert = \Arrowvert \nao G(\vec{\psi}, \todo x \varphi (x)) \Arrowvert$.
	
	\qquad If $\Gamma_{n}\cup \Arrowvert \nao G(\vec{\psi}, \todo x \varphi (x)) \Arrowvert$ is not $\Cv$-consistent, then take $\Gamma_{n+1}$ as $\Gamma_{n}$.
	
	\qquad If $\Gamma_{n}\cup \Arrowvert \nao G(\vec{\psi}, \todo x \varphi (x)) \Arrowvert$ is $\Cv$-consistent, we shall show that for some witness variable $a$, $\Gamma_{n}\cup \Arrowvert \nao G(\vec{\psi}, \varphi (a)) \Arrowvert$ is $\Cv$-consistent.
	
	\qquad First, we can assume that there is no overlap between the propositional variables $\tvp_{i_{1}}, \dots,\tvp_{i_{k}},\vec{\tq},\tr$ because from the point of view of the instantiation sets it does not matter if there is an overlap or not, and we are going to work only with the instantiation sets. Hence, by the definition of template
	
	
	\begin{center}
		$F_{i_{1}}(\tvp_{i_{1}})\ou \dots \ou F_{i_{k}}(\tvp_{i_{k}}) \ou G(\vec{\tq},\tr)$
	\end{center}
is a disjunctive template.
	
	\qquad Second, from the definition of instantiation set and from classical reasoning, it can be easily checked that the sets  
	
	
	\begin{center}
		$\Gamma \cup \Arrowvert \nao F_{i_{1}}( \vvarphi_{i_{1}}) \Arrowvert \cup \dots \cup\Arrowvert \nao F_{i_{k}}( \vvarphi_{i_{k}}) \Arrowvert \cup \Arrowvert \nao G(\vec{\psi}, \todo x\varphi (x)) \Arrowvert$ \\    
		
		$\Gamma \cup \Arrowvert \nao F_{i_{1}}( \vvarphi_{i_{1}}) \e \dots \e \nao F_{i_{k}}( \vvarphi_{i_{k}}) \e \nao G(\vec{\psi}, \todo x\varphi (x)) \Arrowvert$ \\        
		
		$\Gamma \cup \Arrowvert \nao (F_{i_{1}}( \vvarphi_{i_{1}}) \ou \dots \ou  F_{i_{k}}( \vvarphi_{i_{k}}) \ou  G(\vec{\psi}, \todo x\varphi (x))) \Arrowvert$     
	\end{center}
have the same consequences. Thus, $\Gamma \cup \Arrowvert \nao (F_{i_{1}}( \vvarphi_{i_{1}}) \ou \dots \ou  F_{i_{k}}( \vvarphi_{i_{k}}) \ou  G(\vec{\psi}, \todo x\varphi (x))) \Arrowvert$  is $\Cv$-consistent.
	
	
	\qquad Third, let $a$ be the first witness variable that does not occur in $\Gamma, \vvarphi_{i_{1}},\dots, \vvarphi_{i_{k}}, \vec{\psi}$ and $\todo x \varphi (x)$ (remember $\Gamma$ is a set of formulas from the basic language). Then, by Proposition 27, $\Gamma \cup \Arrowvert \nao (F_{i_{1}}( \vvarphi_{i_{1}}) \ou \dots \ou  F_{i_{k}}( \vvarphi_{i_{k}}) \ou  G(\vec{\psi},\varphi (a))) \Arrowvert$  is $\Cv$-consistent. As before, it can be seen that  
	\begin{center}
		$\Gamma \cup \Arrowvert \nao F_{i_{1}}( \vvarphi_{i_{1}}) \Arrowvert \cup \dots \cup\Arrowvert \nao F_{i_{k}}( \vvarphi_{i_{k}}) \Arrowvert \cup \Arrowvert \nao G(\vec{\psi}, \varphi (a)) \Arrowvert$ \\    
		
	\end{center}
is $\Cv$-consistent. That is:    
	
	
	\begin{center}
		$\Gamma_{n} \cup  \Arrowvert \nao G(\vec{\psi}, \varphi (a)) \Arrowvert$ \\        
	\end{center}
is $\Cv$-consistent. So, take $\Gamma_{n+1}$ as $\Gamma_{n} \cup \Arrowvert \nao G(\vec{\psi}, \varphi (a)) \Arrowvert$.
	
	\qquad It can be easily checked that $\bigcup_{n\in \omega} \Gamma_{n}$ is $\Cv$-consistent. So, by Proposition 22 there is a set $\Gamma\p$ such that  $\bigcup_{n\in \omega} \Gamma_{n} \subseteq\Gamma\p$ and $\Gamma\p$ is $\Cv$-maximal consistent. 
	
	\qquad Clearly, $\Gamma\subseteq\bigcup_{n\in \omega} \Gamma_{n} \subseteq\Gamma\p$. Now we show that $\Gamma\p$ admits instantiation.
	
	
	\qquad Let $\vvarphi$ be a sequence of $\D$-formulas, $\todo x \varphi (x)$ a $\D$-formula and $F(\tvp,\tq)$ a disjunctive template. Suppose that $\Gamma\p \cup  \Arrowvert \nao F(\vvarphi, \todo x\varphi (x)) \Arrowvert$ is $\Cv$-consistent. So, for some $k \in \omega$, $\bl F(\tvp,\tq) , \bl\vvarphi,\todo x \varphi (x)  \br\br$ is the $k$\textsuperscript{th} term of the initial sequence. Since $\Gamma_{k}\subseteq\bigcup_{n\in \omega} \Gamma_{n} \subseteq\Gamma\p$, $\Gamma_{k} \cup  \Arrowvert \nao F(\vvarphi, \todo x\varphi (x)) \Arrowvert$ is $\Cv$-consistent. By construction, for some witness variable $a$, $\Gamma_{k+1}=\Gamma_{k} \cup  \Arrowvert \nao F(\vvarphi, \varphi (a)) \Arrowvert$ is $\Cv$-consistent. Thus $\Arrowvert \nao F(\vvarphi, \varphi (a)) \Arrowvert\subseteq \Gamma\p$. Hence, $\Gamma\p \cup \Arrowvert \nao F(\vvarphi, \varphi (a)) \Arrowvert$ is $\Cv$-consistent. 
\end{proof}



\begin{lema}
	Suppose $\Gamma$ is a set of formulas that admits instantiation, $F(\tvp)$ is a disjunctive template, and $\vvarphi$ is a sequence of $\D$-formulas. Then, $\Gamma \cup \Arrowvert \nao F(\vvarphi) \Arrowvert$ also admits instantiation.
\end{lema}

\begin{proof}
	Let $\vec{\psi}$ be a sequence of $\D$-formulas, $\todo x \varphi (x)$ a $\D$-formula and $G(\vec{\tq}, \tr)$ a disjunctive template. Suppose $(\Gamma \cup \Arrowvert \nao F(\vvarphi) \Arrowvert) \cup \Arrowvert \nao G(\vec{\psi}, \todo x \varphi (x)) \Arrowvert$ is $\Cv$-consistent.
	
	\qquad As before, we can assume that $occ(F(\tvp)) \cap occ(G(\vec{\tq}, \tr)) = \vazio$. So $F(\tvp) \ou G(\vec{\tq}, \tr)$ is a disjunctive template. And as before, the sets
	
	\begin{center}
		$(\Gamma \cup \Arrowvert \nao F(\vvarphi) \Arrowvert) \cup \Arrowvert \nao G(\vec{\psi}, \todo x \varphi (x)) \Arrowvert$ \\
		$\Gamma \cup \Arrowvert \nao F(\vvarphi) \e \nao G(\vec{\psi}, \todo x \varphi (x)) \Arrowvert$\\ 
		$\Gamma \cup \Arrowvert  \nao (F(\vvarphi) \ou  G(\vec{\psi}, \todo x \varphi (x))) \Arrowvert$
	\end{center}
have the same consequences. Thus, $\Gamma \cup \Arrowvert  \nao (F(\vvarphi) \ou  G(\vec{\psi}, \todo x \varphi (x))) \Arrowvert$ is $\Cv$-consistent. Since $\Gamma$ admits instantiation, there is a witness variable $a$ such that $\Gamma \cup \Arrowvert  \nao (F(\vvarphi) \ou  G(\vec{\psi}, \varphi (a))) \Arrowvert$ is $\Cv$-consistent. Hence, $(\Gamma \cup \Arrowvert \nao F(\vvarphi) \Arrowvert) \cup \Arrowvert \nao G(\vec{\psi},\varphi (a)) \Arrowvert$ is $\Cv$-consistent.
\end{proof}




\begin{pro}(\textit{Secondary expansion}) Let $\C$ be a variant closed and axiomatically appropriate constant specification for the basic language, $\Cv$ its extension and $\Gamma \subseteq \Fjv$ a $\Cv$-consistent set that admits instantiation. In these conditions, there is a $\Gamma\p \subseteq \Fjv$ such that $\Gamma \subseteq \Gamma\p$, $\Gamma\p$ is $\Cv$-maximal consistent set and $\Gamma\p$ admits instantiation.  
\end{pro}

\begin{proof}
	The proof is very similar to the proof of Proposition 31.
	
	\qquad We define a sequence $\Gamma_{1},\Gamma_{2}, \dots$ of $\Cv$-consistent sets that admit instantiation. First,     $\Gamma_{1} = \Gamma$.
	
	\qquad Now, suppose $\Gamma_{n}$ is already constructed. Let $\bl F_{n}(\tvp_{n}), \vvarphi_{n} \br$ be the $n$\textsuperscript{th} pair of the initial sequence. If the last term of the sequence $\vvarphi_{n}$ is not a universal formula, let $\Gamma_{n+1} = \Gamma_{n}$. Otherwise, consider the following. $\vvarphi_{n}$ is of the form $\vec{\psi}, \todo x \varphi (x)$. And $F_{n}(\tvp_{n})$ is the disjunctive template $G(\vec{\tq},\tr)$ and so  $\Arrowvert \nao F_{n}( \vvarphi_{n}) \Arrowvert = \Arrowvert \nao G(\vec{\psi}, \todo x \varphi (x)) \Arrowvert$. If $\Gamma_{n}\cup \Arrowvert \nao G(\vec{\psi}, \todo x \varphi (x)) \Arrowvert$ is not $\Cv$-consistent, then take $\Gamma_{n+1}$ as $\Gamma_{n}$.
	
	\qquad If $\Gamma_{n}\cup \Arrowvert \nao G(\vec{\psi}, \todo x \varphi (x)) \Arrowvert$ is $\Cv$-consistent, then, since $\Gamma_{n}$ admits instantiation, there is a witness variable $a$ such that $\Gamma_{n}\cup \Arrowvert \nao G(\vec{\psi}, \varphi (a)) \Arrowvert$ is $\Cv$-consistent. By Lemma 4,  $\Gamma_{n}\cup \Arrowvert \nao G(\vec{\psi}, \varphi (a)) \Arrowvert$ admits instantiation. So, take $\Gamma_{n+1}$ as $\Gamma_{n}\cup \Arrowvert \nao G(\vec{\psi}, \varphi (a)) \Arrowvert$.
	
	\qquad As before, it can be checked that $\bigcup_{n\in \omega}\Gamma _{n}$ is a  $\Cv$-consistent set that admits instantiation. By Proposition 22 there is a set $\Gamma\p$ such that  $\bigcup_{n\in \omega} \Gamma_{n} \subseteq\Gamma\p$ and $\Gamma\p$ is $\Cv$-maximal consistent. It is easy to see that  $\Gamma\p$  admits instantiation. 
\end{proof}


\subsection{Completeness}

\begin{defn}
	A \textit{canonical model} $\M = \model$, using constant specification $\C$, is specified as follows.
	
	\begin{itemize}
		\item $\W$ is the set of all $\C(\textbf{V})$-maximally consistent sets that admit instantiation.
		\item Let $\Gamma, \Delta \in \W$. $\Gamma\R\Delta$ iff $\Gamma^{\#} \subseteq  \Delta$.
		\item $\D = \textbf{V}$.
		\item For an $n$-place relation symbol $P$ and for $\Gamma \in \W$, let $\I(P,\Gamma)$ be the set of all $\vec{a}$ where $\vec{a} \in \textbf{V}$ and $P(\vec{a}) \in \Gamma$.
		\item For $\Gamma \in \W$, set $\Gamma \in \E(t,\varphi)$ iff $t$$:_{X}$$\varphi \in \Gamma$, where $t$$:_{X}$$\varphi$ is a closed $\D$-formula and $X$ is the set of witness variables in $\varphi$.
	\end{itemize}
\end{defn}


\qquad First we need to check that $\M$ is indeed a Fitting model meeting $\C$. Since the argument is similar to the one presented in \cite[pp. 13-14]{Fitting14} we are only going to show that $\R$ is an equivalence relation and that the $?$ Condition holds.  \\

\qquad \textit{$\R$ is reflexive}. Let $\Gamma \in \W$, and let $t$$:_{X}$$\varphi = t$$:_{X}$$\varphi(\vec{y})$ be a closed $\D$-formula in $\Gamma$ such that $\vec{y}$ is an $n$-ary sequence of basic variables, say $y_{1}, \dots, y_{n}$ and, of course, $\vec{y} \notin X$. By repeated use of axiom \textbf{B6} and classical reasoning:

\begin{center}
	$\teo_{C(\textbf{V})}t$$:_{X}$$\varphi(\vec{y}) \impli gen_{y_{1}}(gen_{y_{2}} \dots (gen_{y_{n}}(t)))$$:_{X}\todo \vec{y} \varphi(\vec{y})$
\end{center}
By axiom \textbf{B1},

\begin{center}
	$\teo_{C(\textbf{V})} gen_{y_{1}}(gen_{y_{2}} \dots (gen_{y_{n}}(t)))$$:_{X}\todo \vec{y} \varphi(\vec{y}) \impli \todo \vec{y} \varphi(\vec{y})$
\end{center}
hence, by the maximal consistency of $\Gamma$, $\todo \vec{y} \varphi(\vec{y}) \in \Gamma$. Thus $\Gamma^{\#} \subseteq \Gamma$, i.e., $\Gamma\R\Gamma$.\\

\qquad \textit{$\R$ is transitive}. Let $\Gamma, \Delta, \Theta \in \W$ such that $\Gamma\R\Delta$ and $\Delta\R\Theta$; and let $\varphi \in \Gamma^{\#}$, i.e., $\varphi = \todo \vec{y} \psi(\vec{a},\vec{y})$ ($\vec{a}$ is a sequence of witness variables and $\vec{y}$ is a sequence of basic variables) and $t$$:_{\{\vec{a}\}}$$\psi(\vec{a},\vec{y}) \in \Gamma$.

\qquad By the axiom \textbf{B4} and by the maximal consistency of $\Gamma$,  $!t$$:_{\{\vec{a}\}}$$ t$$:_{\{\vec{a}\}}$$\psi(\vec{a},\vec{y}) \in \Gamma$. Since $t$$:_{\{\vec{a}\}}$$\psi(\vec{a},\vec{y})$ has no free basic variables and $\Gamma \R \Delta$, then $t$$:_{\{\vec{a}\}}$$\psi(\vec{a},\vec{y}) \in \Delta$. And since $\Delta\R\Theta$, then $\todo \vec{y} \psi(\vec{a},\vec{y}) \in \Theta$, i.e., $\varphi  \in \Theta$. Thus, $\Gamma^{\#} \subseteq \Theta$, i.e., $\Gamma\R\Theta$.\\





\qquad \textit{$\R$ is symmetric}. Let $\Gamma, \Delta \in \W$. Suppose that $\Gamma \R \Delta$ and suppose it is not the case that $\Delta \R \Gamma$. Then $\Delta^{\#} \nsubseteq \Gamma$. So for some term $t$, some set of witness variables $X$ and some $\D$-formula $\varphi(\vec{y})$,  $t$$:_{X}$$\varphi(\vec{y}) \in \Delta$ and $\todo \vec{y} \varphi(\vec{y}) \notin \Gamma$. By the maximal consistency of $\Gamma$,  $\nao \todo \vec{y} \varphi(\vec{y}) \in \Gamma$. Now, assume that $t$$:_{X}$$\varphi(\vec{y}) \in \Gamma$. Then by repeated use of axiom \textbf{B6}, $gen_{y_{1}}(gen_{y_{2}} \dots (gen_{y_{n}}(t)))$$:_{X}\todo \vec{y} \varphi(\vec{y})\in \Gamma$. By axiom \textbf{B1}, $\todo \vec{y} \varphi(\vec{y})\in \Gamma$, a contradiction. Hence, $t$$:_{X}$$\varphi(\vec{y}) \notin \Gamma$, by the maximal consistency of $\Gamma$, $\nao t$$:_{X}$$\varphi(\vec{y}) \in \Gamma$. By axiom \textbf{B5}, $?t$$:_{X}$$\nao t$$:_{X}$$\varphi(\vec{y}) \in \Gamma$. Since $\Gamma^{\#} \subseteq \Delta$, then $\nao t$$:_{X}$$\varphi(\vec{y}) \in \Delta$, a contradiction. Therefore, if $\Gamma \R \Delta$, then $\Delta \R \Gamma$.\\

\qquad \textit{$?$ Condition}. Suppose $\Gamma \in \W \backslash \E(t,\varphi)$; and let $X$ be the set of all witness variables occurring in $\varphi$. Thus, by the definition of $\E$, $t$$:_{X}$$\varphi \notin \Gamma$. By the maximal consistency of $\Gamma$,  $\nao t$$:_{X}$$\varphi \in \Gamma$. By the axiom \textbf{B5}, $?t$$:_{X}$$\nao t$$:_{X}$$\varphi \in \Gamma$. Hence, $\Gamma \in \E(?t,\nao t$$:_{X}$$\varphi)$.\\


\qquad We have shown that the canonical model is a Fitting model meeting $\C$. Now, to show that the canonical model is a Fitting model for FOJT45, we need to show that $\E$ is a strong evidence function. This is going to be a consequence of the following Lemma:


\begin{lema}
	(\textit{Truth Lemma}). Let $\M=\model$ be a canonical model. For each $\Gamma \in \W$ and for each closed $\D$-formula $\varphi$,
	\begin{center}
		$\M,\Gamma \models \varphi$ iff $\varphi \in \Gamma$
	\end{center}
\end{lema}

\begin{proof}
	Induction on the complexity of $\varphi$. The crucial cases are when $\varphi$ is $t$$:_{X}$$\psi$ and when $\varphi$ is $\todo x \psi(x)$. \\
	
	($\varphi$ is $t$$:_{X}$$\psi$)\\
	
	\qquad ($\Rightarrow$) Suppose $t$$:_{X}$$\psi \notin \Gamma$. Let $X\p \subseteq X$ be a set where $X\p$ contain exactly the witness variables that occur in $\psi$. It is not the case that $t$$:_{X\p}$$\psi \in \Gamma$. Otherwise, by axiom \textbf{A3} and by the maximal consistency of $\Gamma$,  $t$$:_{X}$$\psi \in \Gamma$. So by the definition of $\E$, $\Gamma \notin \E(t,\psi)$, thus $\M,\Gamma \nmodels t$$:_{X}$$\psi$.
	
	\qquad ($\Leftarrow$) First, suppose $t$$:_{X}$$\psi \in \Gamma$. Again, let $X\p \subseteq X$ be as above. So, by the axiom \textbf{A2} and by the maximal consistency of $\Gamma$, $t$$:_{X\p}$$\psi \in \Gamma$. Hence, $\Gamma \in \E(t,\psi)$. Second, let $\Delta \in \W$ such that $\Gamma \R \Delta$. So $\todo \vec{y}\psi \in \Delta$ where $\vec{y}$ are the free basic variables of $\psi$. Thus, by the classical axioms and by the maximal consistency of $\Delta$, for every $\vec{a} \in \textbf{V}$,  $\psi(\vec{a}) \in \Delta$. By the induction hypothesis, for every $\vec{a} \in \textbf{V}$, $\M, \Delta \models \psi(\vec{a})$. Therefore, $\M,\Gamma \models t$$:_{X\p}$$\psi$, and so $M,\Gamma \models t$$:_{X}$$\psi$.\\
	
	
	($\varphi$ is $\todo x \psi(x)$)\\
	
	\qquad ($\Rightarrow$) Suppose $\todo x \psi(x) \notin \Gamma$. By the maximal consistency of $\Gamma$, $\nao \todo x \psi(x) \in \Gamma$. Since $\Gamma$ admits instantiation, then by Proposition 29 there is an $a \in \textbf{V}$ such that $\nao \psi(a) \in \Gamma$. By the consistency of $\Gamma$, $\psi(a) \notin \Gamma$. By the induction hypothesis, $\M, \Gamma \nmodels \psi(a)$, thus $\M, \Gamma \nmodels \todo x \psi(x)$.    
	
	
	\qquad ($\Leftarrow$) Suppose  $\todo x \psi(x) \in \Gamma$. By the classical axioms and by the maximal consistency of $\Gamma$, for every $a \in \textbf{V}$,  $\psi(a) \in \Gamma$. By the induction hypothesis, $\M, \Gamma \models \psi(a)$, for every $a \in \textbf{V}$. Therefore,  $\M, \Gamma \models \todo x \psi(x)$.  
\end{proof}

\qquad By the Truth Lemma, we have the following:


\begin{center}
	$\Gamma \in \E(t,\varphi) \Rightarrow t$$:_{X}$$\varphi \in \Gamma \Rightarrow \M,\Gamma \models t$$:_{X}$$\varphi \Rightarrow \Gamma \in \{w \in \W$ $|$ $ \M,w \models t$$:_{X}$$\varphi\}$
\end{center}

Hence $\E$ is a strong evidence function, and so $\M$ is a Fitting model for FOJT45 meeting $\C$.

\begin{teor}
	(\textit{Completeness}) Let $\C$ be a constant specification. For every closed formula $\varphi \in \Fj$, if $\models_{\C} \varphi$, then $\teo_{\C}\varphi$.
\end{teor}

\begin{proof}
	Suppose $\not\teo_{\C}\varphi$. Then $\{\nao \varphi\}$ is $\C$-consistent. By Proposition 31, there is a $\C(\textbf{V})$-maximal consistent $\Gamma$ such that $\Gamma$ admits instantiation and  $\{\nao \varphi\} \subseteq \Gamma$. By the Truth Lemma, $\M,\Gamma \models \nao \varphi$, so  $\M,\Gamma \nmodels \varphi$. Hence, $\nmodels_{\C} \varphi$.     
\end{proof}




\begin{defn}
	A model $\M = \model$ is \textit{fully explanatory} if the following condition is fulfilled. Let $\varphi$ be a formula with no free individual variables, but with constants from the domain of the model. Let $w \in \W$. If for every $v \in \W$ such that $w\R v$, $\M, v \models \varphi$, then there is a justification term $t$ such that $\M, w \models t$$:_{X}$$\varphi$, where $X$ is the set of
	domain constants appearing in $\varphi$. 
\end{defn}


\begin{teor}
	The canonical model is fully explanatory.
\end{teor}

\begin{proof}
	Let $\M = \model$ be a canonical model, $\Gamma \in \W$, $\varphi$ a closed $\D$-formula and $X$ the set of the witness variables occurring $\varphi$. We shall show that if $\M, \Gamma \nmodels t$$:_{X}$$\varphi$ for every justification term $t$ of $\Fj$, then there is a $\Delta \in \W$ such that $\Gamma \R\Delta$ and $\M, \Delta \nmodels \varphi$.
	
	\qquad If $\M, \Gamma \nmodels t$$:_{X}$$\varphi$ for every justification term $t$ of $\Fj$, then by the Truth Lemma, $\nao t$$:_{X}$$\varphi \in \Gamma$ for every justification term $t$ of $\Fj$. The template $G(\tp) =\tp$ is a disjunctive template. Let $F(\tp) = \Box G(\tp)$. Hence, $\Arrowvert \nao F(\varphi)\Arrowvert \subseteq \Gamma$. And so, $\Gamma \cup \Arrowvert \nao F(\vvarphi)\Arrowvert$ is $\Cv$-consistent. By item 2) of Proposition 28, $\Gamma^{\#} \cup \Arrowvert \nao G(\varphi)\Arrowvert$ is $\Cv$-consistent, i.e.,  $\Gamma^{\#} \cup \{  \nao \varphi \} $ is $\Cv$-consistent. By Proposition 30, $\Gamma^{\#}$ admits instantiation. By Lemma 4,  $\Gamma^{\#} \cup \{  \nao \varphi \} $ admits instantiation. By Proposition 32, there is a $\Cv$-maximal consistent set $\Delta$ such that $\Delta$ admits instantiation and $\Gamma^{\#} \cup \{  \nao \varphi \}\subseteq\Delta$. Since $\Gamma^{\#} \subseteq \Delta$, $\Gamma \R \Delta$. And since $\nao \varphi \in \Delta$, by the Truth Lemma,  $\M, \Delta \nmodels \varphi$.
	
\end{proof}

\chapter{Conclusion and future research}

\section{An axiomatic system for FOS5}

\qquad To prove the results of Chapters 2 and 3 it was convenient to state things in terms of $\nao, \ou, \ex, \Diamond$ and $=$. But to stay connected with the formulations of the last chapter consider now the version of first-order modal logic defined using $\bot$, $\impli$, $\todo$ and $\Box$ (without equality). Let $\Li$ be the same language fixed in Chapter 5. To make things simple, we write $Fml$ instead of $\FLi$.   


\qquad Since we have started working only with semantical notions, we have defined the logic FOS5 as the set of all valid sentences relative to the class of all FOS5-models. Alternatively we can study the logic FOS5 using a simple and elegant axiomatic system composed of the following axiom schemes and inference rules:\\

\textbf{A$\p$1} classical axioms of first-order logic\\

\textbf{A$\p$2} $\Box \varphi \impli \varphi$\\

\textbf{A$\p$3} $\Box \varphi \impli \Box \Box \varphi$\\

\textbf{A$\p$4} $\nao \Box \varphi \impli \Box \nao \Box \varphi$\\

\textbf{A$\p$5} $\Box (\varphi \impli \psi) \impli  (\Box \varphi \impli \Box \psi)$\\
	
\textbf{R$\p$1} (\textit{Modus Ponens}) $\teo \varphi$, $\teo \varphi\impli\psi$ $\Rightarrow$ $\teo \psi$ \\

\textbf{R$\p$2} (\textit{generalization})  $\teo \varphi$ $\Rightarrow$ $\teo \todo x \varphi$ \\

\textbf{R$\p$3} (\textit{necessitation})  $\teo \varphi$ $\Rightarrow$  $\teo \Box \varphi$.\\


\qquad As in the case for FOJT45 we make use of the standard notion of $\Gamma \teo \varphi$. Here the restriction on the generalization rule is the same as stated for FOJT45, and the necessitation rule is allowed only when $\Gamma = \vazio$. We write $FOS5\teo \varphi$ to denote that in this axiomatic system $\vazio\teo \varphi$.

\qquad 
In the seminal paper by Kripke \cite{Kripke59} the Completeness Theorem for this logic was shown, and so the semantical and the syntactical characterization of FOS5 are equivalent. To be more precise, for every sentence $\varphi \in Fml$,

\begin{center}
$FOS5\teo \varphi$ iff  $\vSB \varphi$.
\end{center}





\section{Realization}


\begin{defn}
Let $\varphi$ be a formula of FOS5. We define the \textit{realization} of $\varphi$ in the language of FOJT45, $\varphi^{r}$, as follows:


\begin{itemize}
	\item If $\varphi$ is atomic, then $\varphi^{r}$ is $\varphi$.
	\item If $\varphi$ is $\psi\impli\theta$, then $\varphi^{r}$ is $\psi^{r}\impli\theta^{r}$
	\item If $\varphi$ is $\todo x \psi$, then $\varphi^{r}$ is $\todo x \psi^{r}$
	\item If $\varphi$ is $\Box \psi$ and $fv(\varphi) = \{x_{1}, \dots, x_{n}\}$, then $\varphi^{r}$ is $t$$:_{\{x_{1}, \dots, x_{n}\}}$$ \psi^{r}$
\end{itemize}
\end{defn}

\qquad A realization is normal if all negative occurrences of $\Box$ are assigned justification variables. It can easily be checked that for every $\varphi \in Fml$, $fv(\varphi) = fv(\varphi^{r})$.



\begin{defn}
Let $\varphi$ be a formula of FOJT45. \textit{The forgetful projection} of $\varphi$, $\varphi^{\circ}$, is defined as follows:


\begin{itemize}
	\item If $\varphi$ is atomic, then $\varphi^{\circ}$ is $\varphi$.
	\item If $\varphi$ is $\psi\impli\theta$, then $\varphi^{\circ}$ is $\psi^{\circ}\impli\theta^{\circ}$
	\item If $\varphi$ is $\todo x \psi$, then $\varphi^{\circ}$ is $\todo x \psi^{\circ}$
	\item If $\varphi$ is $t$$:_{X}$$ \psi$, then $\varphi^{\circ}$ is $\Box \todo \vec{y}\psi^{\circ}$\\ where $\vec{y} \in fv(\psi)\backslash X$.
\end{itemize}
\end{defn}

\qquad As before, it can easily be checked that for every $\varphi \in \Fj$,  $fv(\varphi) = fv(\varphi^{\circ})$.

\begin{pro}
For every constant specification $\C$ and for every $\varphi \in \Fj$,
\begin{center}
If FOJT45 $\teo_{\C} \varphi$, then FOS5 $\teo \varphi^{\circ}$.
\end{center}
\end{pro}


\begin{proof}
Induction on the theorems of FOJT45 with $\C$. In this proof only we shall use $\teo$ to denote $FOS5 \teo$. And for simplicity we are going to deal only with a representative special case of each axiom. These special cases are simpler versions of each axiom; the argument can be easily generalized.\\



($\varphi$ is an instance of \textbf{A2})\\

\qquad Suppose $\varphi$ is

\begin{center}
$t$$:_{\{x,y\}}$$\psi(x,z) \impli t$$:_{\{x\}}$$\psi(x,z)$
\end{center}
Since $y \notin fv(\psi(x,z))$,

\begin{center}
$\{z\} = fv(\psi(x,z)) \backslash \{x,y\} = fv(\psi(x,z)) \backslash \{x\}$ 
\end{center}
Thus, $\varphi^{\circ}$ is

\begin{center}
$\Box  \todo z \psi^{\circ}(x,z) \impli \Box  \todo z \psi^{\circ}(x,z)$.
\end{center}

Clearly, $\teo \varphi^{\circ}$.\\
\vspace{5mm}

($\varphi$ is an instance of \textbf{A3})\\

\qquad Suppose $\varphi$ is

\begin{center}
$t$$:_{\{x\}}$$\psi(x,y,z) \impli t$$:_{\{x,y\}}$$\psi(x,y,z)$
\end{center}
Then, $\varphi^{\circ}$ is

\begin{center}
$\Box  \todo y\todo z \psi^{\circ}(x,y,z) \impli \Box  \todo z \psi^{\circ}(x,y,z)$
\end{center}
By classical axioms,

\begin{center}
$\teo \todo y\todo z \psi^{\circ}(x,y,z) \impli \todo z \psi^{\circ}(x,y,z)$
\end{center}
By necessitation and the distributivity of $\Box$ over $\impli$,

\begin{center}
$\teo \Box  \todo y\todo z \psi^{\circ}(x,y,z) \impli \Box  \todo z \psi^{\circ}(x,y,z)$.
\end{center}

\vspace{5mm}

($\varphi$ is an instance of \textbf{B1})\\

\qquad Suppose $\varphi$ is

\begin{center}
$t$$:_{\{x\}}$$\psi(x,y) \impli \psi(x,y)$
\end{center}
Then, $\varphi^{\circ}$ is

\begin{center}
$\Box  \todo y\psi^{\circ}(x,y) \impli \psi^{\circ}(x,y)$
\end{center}
By \textbf{A$\p$2},

\begin{center}
$\teo \Box \todo y\psi^{\circ}(x,y) \impli \todo y\psi^{\circ}(x,y)$
\end{center}
And by classical axioms,

\begin{center}
$\teo \todo y\psi^{\circ}(x,y) \impli \psi^{\circ}(x,y)$
\end{center}
So, 

\begin{center}
$\teo \Box  \todo y\psi^{\circ}(x,y) \impli \psi^{\circ}(x,y)$.
\end{center}
\vspace{5mm}

($\varphi$ is an instance of \textbf{B2})\\

Suppose $\varphi$ is

\begin{center}
$t$$:_{\{x,x\p\}}$$(\psi(x,y) \impli \theta(x\p,z)) \impli$ $(s$$:_{\{x,x\p\}}$$\psi(x,y) \impli$ $[t\cdot s]$$:_{\{x,x\p\}}$$\theta(x\p,z))$
\end{center}
Then, $\varphi^{\circ}$ is

\begin{center}
$\Box  \todo y\todo z (\psi^{\circ}(x,y) \impli \theta^{\circ}(x\p,z)) \impli (\Box \todo y \psi^{\circ}(x,y) \impli \Box \todo z \theta^{\circ}(x\p,z))$
\end{center}
By classical reasoning,


\begin{center}
$\teo \todo y\todo z (\psi^{\circ}(x,y) \impli \theta^{\circ}(x\p,z)) \impli (\todo y\todo z  \psi^{\circ}(x,y) \impli \todo y\todo z  \theta^{\circ}(x\p,z))$
\end{center}
Since $z \notin fv(\psi^{\circ}(x,y))$ and $y \notin fv(\theta^{\circ}(x\p,z))$, we have that


\begin{center}
$\teo \todo y\todo z  \psi^{\circ}(x,y) \see \todo y \psi^{\circ}(x,y)$\\
$\teo \todo y\todo z  \theta^{\circ}(x\p,z) \see \todo z \theta^{\circ}(x\p,z)$
\end{center}
Hence,

\begin{center}
$\teo \todo y\todo z (\psi^{\circ}(x,y) \impli \theta^{\circ}(x\p,z)) \impli (\todo y \psi^{\circ}(x,y) \impli \todo z \theta^{\circ}(x\p,z))$
\end{center}
By necessitation and the distributivity of $\Box$ over $\impli$,

\begin{center}
$\teo \Box  \todo y\todo z (\psi^{\circ}(x,y) \impli \theta^{\circ}(x\p,z)) \impli (\Box \todo y \psi^{\circ}(x,y) \impli \Box \todo z \theta^{\circ}(x\p,z))$.
\end{center}
\vspace{5mm}

($\varphi$ is an instance of \textbf{B3})\\

\qquad If $\varphi$ is $t$$:_{\{x\}}$$\psi(x,y) \impli$ $[t+s]$$:_{\{x\}}$$\psi(x,y)$, then $\varphi^{\circ}$ is $\Box  \todo y \psi^{\circ}(x,y) \impli \Box  \todo y \psi^{\circ}(x,y)$. Clearly, $\teo \varphi^{\circ}$. The same argument holds when $\varphi$ is $s$$:_{\{x\}}$$\psi(x,y) \impli$ $[t+s]$$:_{\{x\}}$$\psi(x,y)$.
\vspace{5mm}

($\varphi$ is an instance of \textbf{B4})\\

\qquad If $\varphi$ is $t$$:_{\{x\}}$$\psi(x,y) \impli$ $!t$$:_{\{x\}}$$t$$:_{\{x\}}$$\psi(x,y)$, then $\varphi^{\circ}$ is $\Box\todo y \psi^{\circ}(x,y) \impli \Box \Box\todo y \psi^{\circ}(x,y)$, which is an instance of axiom \textbf{A$\p$3}; hence $\teo \varphi^{\circ}$.
\vspace{5mm}



($\varphi$ is an instance of \textbf{B5})\\


\qquad If $\varphi$ is $\nao t$$:_{\{x\}}$$\psi(x,y) \impli$ $?t$$:_{\{x\}}$$\nao t$$:_{\{x\}}$$\psi(x,y)$, then $\varphi^{\circ}$ is $\nao \Box\todo y \psi^{\circ}(x,y) \impli \Box \nao \Box\todo y \psi^{\circ}(x,y)$, which is an instance of axiom \textbf{A$\p$4}; hence $\teo \varphi^{\circ}$.

\pagebreak
($\varphi$ is an instance of \textbf{B6})\\

\qquad Suppose $\varphi$ is

\begin{center}
 $t$$:_{\{y\}}$$\psi(x,y,z) \impli gen_{x}(t)$$:_{\{y\}}$$\todo x\psi(x,y,z)$
\end{center}
So $\varphi^{\circ}$ is
\begin{center}
 $\Box \todo x \todo z\psi^{\circ}(x,y,z) \impli \Box \todo z \todo x\psi^{\circ}(x,y,z)$
\end{center}
By classical reasoning,

\begin{center}
 $\teo \todo x \todo z\psi^{\circ}(x,y,z) \impli \todo z \todo x\psi^{\circ}(x,y,z)$
\end{center}
By necessitation and the distributivity of $\Box$ over $\impli$,

\begin{center}
 $\teo \Box \todo x \todo z\psi^{\circ}(x,y,z) \impli \Box \todo z \todo x\psi^{\circ}(x,y,z)$.
\end{center}

\qquad If $\varphi$ is derived by using the rules \textbf{R1} or \textbf{R2} the result easily follows from the induction hypothesis.

\qquad Suppose $\varphi$ is derived using the rule \textbf{R3}. So $\varphi$ is $c$$:$$\psi(x)$ where $\psi(x)$ is an axiom. By the argument above, $\teo \psi^{\circ}(x)$. By generalization and necessitation, $\teo \Box \todo x\psi^{\circ}(x)$, i.e., $\teo \varphi^{\circ}$.
\end{proof}

\qquad As usual in the study of justification logic, the proof of Proposition 33 is a trivial induction on the theorems of the justification logic in question (in this case FOJT45). What is a more significant result is the following:\\


\textbf{(Realization Theorem)} If FOS5 $\teo \varphi$, then FOJT45 $\teo_{\C} \varphi^{r}$ for a constant specification $\C$ and a normal realization $r$.\\



\qquad Right now we believe that the best path to try to prove this theorem is to apply all the notions and results presented in this thesis in order to adapt the proof of the Realization Theorem using semantical tools (as presented in \cite{Fitting13}, \cite{Fitting13R} and \cite{Fitting14R}) for FOJT45. But we also consider different ways. Another strategy is to study the constructive argument using nested sequent calculus (as presented in \cite{Kuz10}) and see how this argument can be used for this case.
We want to consider these two paths in future research.


\section{Justification logic and interpolation}


\qquad When studying justification logic it is natural to investigate the relationship between this logic and modal logic. The Realization Theorem gives us a tool to see this relationship. Although we have left the proof of this theorem for future work, it is worthwhile to see one easy conclusion of the Realization Theorem. To do so we need to state one definition:



\begin{itemize}
\item[] \textit{The Interpolation Theorem} holds for FOJT45 iff for every constant specification $\C$ and sentences $\varphi$ and $\psi$, if $\teo_{\C} \varphi \impli\psi$, then there is a formula $\theta$ such that $\teo_{\C} \varphi \impli \theta$, $\teo_{\C} \theta \impli \psi$ and the non-logical symbols and the justification terms that occur in $\theta$ occur both in $\varphi$ and $\psi$.
\end{itemize}




\begin{pro}
If the Realization Theorem holds between FOS5 and FOJT45, then the Interpolation Theorem fails for FOJT45. 
\end{pro}


\begin{proof}
Suppose that the Interpolation Theorem holds for FOJT45. By Theorem 2 and by the Completeness Theorem for FOS5, let $\varphi$ and $\psi$ be sentences such that FOS5 $\teo \varphi \impli \psi$ and there is no interpolant between them. By the Realization Theorem, there is a normal realization $r$ such that

\begin{center}
FOJT45 $\teo_{\C} \varphi^{r} \impli \psi^{r}$
\end{center}

\qquad By hypothesis, there is a formula $\theta$ such that the non-logical symbols and the justification terms that occur in $\theta$ occur both in $\varphi^{r}$ and $\psi^{r}$. Moreover, we have that 


\begin{center}
FOJT45 $\teo_{\C} \varphi^{r} \impli \theta$\\
FOJT45 $\teo_{\C} \theta \impli \psi^{r}$
\end{center}
by the forgetful projection: 


\begin{center}
FOS5 $\teo (\varphi^{r} \impli \theta)^{\circ}$\\
FOS5 $\teo (\theta \impli \psi^{r})^{\circ}$
\end{center}
i.e.,

\begin{center}
FOS5 $\teo \varphi \impli \theta^{\circ}$\\
FOS5 $\teo \theta^{\circ} \impli \psi$
\end{center}


\qquad Now, since there is no interpolant between $\varphi$ and $\psi$, then there is no relation symbol occurring in $\theta^{\circ}$. Hence, $\theta^{\circ}$ is a formula such that $\bot$ is the only atomic formula that occur in $\theta^{\circ}$. Thus, either $\theta^{\circ}$ is FOS5-valid or $\theta^{\circ}$ is FOS5-unsatisfiable.

\qquad On the one hand, if $\theta^{\circ}$ is FOS5-valid, then, since $\models_{FOS5} \theta^{\circ}\impli \psi$, $\psi$ is FOS5-valid. And so, $\varphi \impli \psi$ has an interpolant, contradicting our hypothesis.

\qquad On the other hand, if $\theta^{\circ}$ is FOS5-unsatisfiable, then, since $\models_{FOS5} \varphi \impli \theta^{\circ}$, $\varphi$ is FOS5-unsatisfiable. And so, $\varphi \impli \psi$ has an interpolant, contradicting our hypothesis.
\end{proof}

\vspace{10mm}

\qquad We hope that the topics presented in this thesis fulfilled two objectives: i) give a brief introduction to first-order S5; ii) clarify the connections between first-order modal logic and first-order justification logic.

\qquad About the last objective it is important to stress that, as Proposition 34 shows, the failure of the Interpolation Theorem for FOJT45 is just a straightforward consequence of the Realization Theorem. And so to prove this theorem for FOJT45 will not be only a subject of interest for the researchers involved in justification logic, but will be a result of interest for the broader modal logic community. 




\chapter{Appendix}

\textbf{Proof of Proposition 17}

\qquad Proof of the explicit version of the converse Barcan Formula:

\begin{enumerate}[1.]
	\item $ \todo y \varphi(y) \impli \varphi(y)$   (\textit{classical axiom}) 
	\item $c_{1}$$:$$(\todo y \varphi(y) \impli \varphi(y))$  (\textit{axiom necessitation})
	\item $c_{1}$$:_{Xy}$$(\todo y \varphi(y) \impli \varphi(y))$  (\textit{\textbf{A3} + Modus Ponens})
	\item $c_{1}$$:_{Xy}$$(\todo y \varphi(y) \impli \varphi(y)) \impli  (t$$:_{Xy}$$\todo y \varphi(y)  \impli   [c_{1} \cdot t]$$:_{Xy}$$ \varphi(y))$  (\textit{Axiom \textbf{B2}})
	\item $t$$:_{Xy}$$\todo y \varphi(y)  \impli   [c_{1} \cdot t]$$:_{Xy}$$ \varphi(y)$  (\textit{Modus Ponens})
	\item $t$$:_{X}$$\todo y \varphi(y)  \impli   t$$:_{Xy}$$\todo y \varphi(y)$  (\textit{Axiom \textbf{A3}})
	\item $t$$:_{X}$$\todo y \varphi(y)  \impli   [c_{1} \cdot t]$$:_{Xy}$$ \varphi(y)$  (\textit{classical reasoning})
	\item $\todo y(t$$:_{X}$$\todo y \varphi(y)  \impli   [c_{1} \cdot t]$$:_{Xy}$$ \varphi(y))$  (\textit{generalization})
	\item $t$$:_{X}$$\todo y \varphi(y)  \impli   \todo y [c_{1} \cdot t]$$:_{Xy}$$ \varphi(y)$  (\textit{$y \notin X$ + classical reasoning})
	
\end{enumerate}

\pagebreak

\qquad Proof of the explicit version of the Barcan Formula:

\begin{enumerate}[1.]
\item $ \todo y t$$:_{Xy}$$\varphi(y) \impli t$$:_{Xy}$$\varphi(y)$   (\textit{classical axiom})

\item $c_{1}$$:$$(\todo y t$$:_{Xy}$$\varphi(y) \impli t$$:_{Xy}$$\varphi(y))$   (\textit{axiom necessitation})

\item $c_{1}$$:_{X}$$(\todo y t$$:_{Xy}$$\varphi(y) \impli t$$:_{Xy}$$\varphi(y))$    (\textit{\textbf{A3} + Modus Ponens})




\item $gen_{y}(c_{1})$$:_{X}$$\todo y (\todo y t$$:_{Xy}$$\varphi(y) \impli t$$:_{Xy}$$\varphi(y))$   (\textit{$y \notin X$ + \textbf{B6} + Modus Ponens})

\item $gen_{y}(c_{1})$$:_{X}$$\todo y (\todo y t$$:_{Xy}$$\varphi(y) \impli t$$:_{Xy}$$\varphi(y)) \impli \todo y [c_{2} \cdot gen_{y}(c_{1})]$$:_{Xy}$$ (\todo y t$$:_{Xy}$$\varphi(y) \impli t$$:_{Xy}$$\varphi(y))$    (\textit{Converse Barcan Formula}\footnote{where $c_{2}$$:$$\todo y (\todo y t$$:_{Xy}$$\varphi(y) \impli t$$:_{Xy}$$\varphi(y)) \impli (\todo y t$$:_{Xy}$$\varphi(y) \impli t$$:_{Xy}$$\varphi(y)) \in \C$.})

\item $\todo y [c_{2} \cdot gen_{y}(c_{1})]$$:_{Xy}$$ (\todo y t$$:_{Xy}$$\varphi(y) \impli t$$:_{Xy}$$\varphi(y))$    (\textit{Modus Ponens})

\item $[c_{2} \cdot gen_{y}(c_{1})]$$:_{Xy}$$ (\todo y t$$:_{Xy}$$\varphi(y) \impli t$$:_{Xy}$$\varphi(y))$    (\textit{classical axiom + Modus Ponens})

\item $c_{3}$$:_{Xy}$$((\todo y t$$:_{Xy}$$\varphi(y) \impli t$$:_{Xy}$$\varphi(y)) \impli (\nao t$$:_{Xy}$$\varphi(y) \impli \nao \todo y t$$:_{Xy}$$\varphi(y)))$ (\textit{tautology + axiom necessitation + \textbf{A3}})

\item $[c_{3} \cdot [c_{2} \cdot gen_{y}(c_{1})]]$$:_{Xy}$$(\nao t$$:_{Xy}$$\varphi(y) \impli \nao \todo y t$$:_{Xy}$$\varphi(y))$ (\textit{\textbf{B2} + 7,8 + Modus Ponens})

\item $[c_{3} \cdot [c_{2} \cdot gen_{y}(c_{1})]]$$:_{Xy}$$(\nao t$$:_{Xy}$$\varphi(y) \impli \nao \todo y t$$:_{Xy}$$\varphi(y)) \impli (?t$$:_{Xy}\nao t$$:_{Xy}$$\varphi(y) \impli [[c_{3} \cdot [c_{2} \cdot gen_{y}(c_{1})]]\cdot ?t]$$:_{Xy}$$\nao \todo y t$$:_{Xy}$$\varphi(y)) $ (\textit{Axiom \textbf{B2}})


\item $?t$$:_{Xy}\nao t$$:_{Xy}$$\varphi(y) \impli [[c_{3} \cdot [c_{2} \cdot gen_{y}(c_{1})]]\cdot ?t]$$:_{Xy}$$\nao \todo y t$$:_{Xy}$$\varphi(y)$ (\textit{Modus Ponens})


\item $\nao[[c_{3} \cdot [c_{2} \cdot gen_{y}(c_{1})]]\cdot ?t]$$:_{Xy}$$\nao \todo y t$$:_{Xy}$$\varphi(y) \impli \nao  ?t$$:_{Xy}\nao t$$:_{Xy}$$\varphi(y)$ (\textit{classical reasoning})

\item $\nao (?t$$:_{Xy}\nao t$$:_{Xy}$$\varphi(y)) \impli \varphi(y)$ (\textit{JT45 theorem})


\item $\nao[[c_{3} \cdot [c_{2} \cdot gen_{y}(c_{1})]]\cdot ?t]$$:_{Xy}$$\nao \todo y t$$:_{Xy}$$\varphi(y)\impli \varphi(y)$ (\textit{classical reasoning})

\item $\nao[[c_{3} \cdot [c_{2} \cdot gen_{y}(c_{1})]]\cdot ?t]$$:_{X}$$\nao \todo y t$$:_{Xy}$$\varphi(y)\impli  \nao[[c_{3} \cdot [c_{2} \cdot gen_{y}(c_{1})]]\cdot ?t]$$:_{Xy}$$\nao \todo y t$$:_{Xy}$$\varphi(y)$ (\textit{\textbf{A2} + classical reasoning})

\item $\nao[[c_{3} \cdot [c_{2} \cdot gen_{y}(c_{1})]]\cdot ?t]$$:_{X}$$\nao \todo y t$$:_{Xy}$$\varphi(y)\impli \varphi(y)$ (\textit{classical reasoning})

\item $\todo y (\nao[[c_{3} \cdot [c_{2} \cdot gen_{y}(c_{1})]]\cdot ?t]$$:_{X}$$\nao \todo y t$$:_{Xy}$$\varphi(y)\impli \varphi(y))$ (\textit{generalization})


\item $\nao[[c_{3} \cdot [c_{2} \cdot gen_{y}(c_{1})]]\cdot ?t]$$:_{X}$$\nao \todo y t$$:_{Xy}$$\varphi(y)\impli \todo y \varphi(y)$ (\textit{$y \notin X$ + classical reasoning})

\item $r$$:(\nao[[c_{3} \cdot [c_{2} \cdot gen_{y}(c_{1})]]\cdot ?t]$$:_{X}$$\nao \todo y t$$:_{Xy}$$\varphi(y)\impli \todo y \varphi(y))$ (\textit{Internalization})

\item $r$$:_{X}$$(\nao[[c_{3} \cdot [c_{2} \cdot gen_{y}(c_{1})]]\cdot ?t]$$:_{X}$$\nao \todo y t$$:_{Xy}$$\varphi(y)\impli \todo y \varphi(y))$ (\textit{\textbf{A3} + Modus Ponens})



\item $?[[c_{3} \cdot [c_{2} \cdot gen_{y}(c_{1})]]\cdot ?t]$$:_{X}$$\nao[[c_{3} \cdot [c_{2} \cdot gen_{y}(c_{1})]]\cdot ?t]$$:_{X}$$\nao \todo y t$$:_{Xy}$$\varphi(y)\impli$\\
$\impli [r \cdot ?[[c_{3} \cdot [c_{2} \cdot gen_{y}(c_{1})]]\cdot ?t]]$$:_{X}$$\todo y \varphi(y)$
 (\textit{\textbf{B2} + 20 + Modus Ponens })




\item $\todo y t$$:_{Xy}$$\varphi(y) \impli ?[[c_{3} \cdot [c_{2} \cdot gen_{y}(c_{1})]]\cdot ?t]$$:_{X}$$\nao[[c_{3} \cdot [c_{2} \cdot gen_{y}(c_{1})]]\cdot ?t]$$:_{X}$$\nao \todo y t$$:_{Xy}$$\varphi(y)$ \\ (\textit{JT45 theorem})

\item$\todo y t$$:_{Xy}$$\varphi(y) \impli [r \cdot ?[[c_{3} \cdot [c_{2} \cdot gen_{y}(c_{1})]]\cdot ?t]]$$:_{X}$$\todo y \varphi(y)$ (\textit{classical reasoning})
\end{enumerate}



\textbf{Proof of Proposition 18}

\begin{enumerate}[1.]
	\item $ \ex y \varphi(y) \impli \ex y \varphi(y)$   (\textit{tautology})
	
	\item $ \todo y (\varphi(y) \impli \ex y \varphi(y))$   (\textit{classical reasoning})
	
	\item $r$$:$$(\todo y (\varphi(y) \impli \ex y \varphi(y)))$   (\textit{Internalization})
	
	\item $r$$:_{X}$$(\todo y (\varphi(y) \impli \ex y \varphi(y)))$  (\textit{\textbf{A3} + Modus Ponens})
	
	\item $\todo y CB(r)$$:_{Xy}$$ (\varphi(y) \impli \ex y \varphi(y))$  (\textit{Converse Barcan formula + Modus Ponens})
	
	\item $CB(r)$$:_{Xy}$$(\varphi(y) \impli \ex y \varphi(y))$  (\textit{classical axioms + Modus Ponens})
		
	\item $t$$:_{Xy}$$\varphi(y) \impli [CB(r) \cdot t]$$:_{Xy}$$ \ex y \varphi(y)$  	 (\textit{\textbf{B2} + 6 + Modus Ponens })
	

	\item $[CB(r) \cdot t]$$:_{Xy}$$ \ex y \varphi(y) \impli [CB(r) \cdot t]$$:_{X}$$ \ex y \varphi(y)$  (\textit{Axiom \textbf{A2}})
	
	\item $t$$:_{Xy}$$\varphi(y) \impli [CB(r) \cdot t]$$:_{X}$$ \ex y \varphi(y)$  	 (\textit{classical reasoning})
	
	\item $\todo y (t$$:_{Xy}$$\varphi(y) \impli [CB(r) \cdot t]$$:_{X}$$ \ex y \varphi(y))$  	 (\textit{generalization})
	
	\item $\ex y t$$:_{Xy}$$\varphi(y) \impli [CB(r) \cdot t]$$:_{X}$$ \ex y \varphi(y)$  	 (\textit{$y \notin X$ + classical reasoning}).
	
	
\end{enumerate}









\pagebreak
\addcontentsline{toc}{chapter}{Bibliography}

\begin{thebibliography}{100} % 100 is a random guess of the total number of
%	%references
\bibitem{Areces01} ARECES, Carlos; BLACKBURN, Patrick; MARX, Maarten. Repairing the interpolation theorem in quantified modal logic.
{\it Annals of Pure and Applied Logic}, v. 124, p. 278-299, 2003.

\bibitem{Artemov08} ARTEMOV, Sergei. Why do we need justification logic? \textit{Technical Report TR-2008014}, CUNY Ph.D. Program in Computer Science, September 2008.

\bibitem{Artemov01} $\rule{1.30cm}{0.15mm}$. Explicit provability and constructive semantics. \textit{The Bulletin of Symbolic Logic}, v. 7, p. 1-36, 2001.

\bibitem{Artemov06} $\rule{1.30cm}{0.15mm}$. Modal logic in mathematics. In: BLACKBURN, Patrick et al. (Ed.) \textit{Handbook of Modal Logic}. New York: Elsevier, 2006. p. 927-970

\bibitem{Artemov11} ARTEMOV, Sergei; YAVORSKAYA (SIDON), Tatiana. First-order logic of proofs. \textit{Technical Report TR-20111005}, CUNY Ph.D. Program in Computer Science, September 2011.

\bibitem{Kuz10} BR\"UNNLER, Kai; GOETSHI, Remo; KUZNETS, Roman. A syntactic realization theorem for justification logics. \textit{Advances in Modal Logic}, v. 8, p. 39-58, 2010.

\bibitem{Fine79} FINE, Kit. Failures of the interpolation lemma in quantified modal logic. \textit{The Journal of Symbolic Logic}, v. 44, p. 201-206, 1979.

\bibitem{Fine78} $\rule{1.30cm}{0.15mm}$. Model theory for modal logic part I. \textit{Journal of Philosophical Logic}, v. 7, p. 125-156, 1978.

\bibitem{Fitting14R} FITTING, Melvin. Justification logics and realization. \textit{Technical Report TR-2014004}, CUNY Ph.D. Program in Computer Science, March 2014.

\bibitem{Fitting13R} $\rule{1.30cm}{0.15mm}$. Realization implemented. \textit{Technical Report TR-2013005}, CUNY Ph.D. Program in Computer Science, May 2013.

\bibitem{Fitting02} $\rule{1.30cm}{0.15mm}$. Interpolation for first-order S5. \textit{The Journal of Symbolic Logic}, v. 67, p. 621-634, 2002.

\bibitem{Fitting13} $\rule{1.30cm}{0.15mm}$. Realization using the model existence theorem. \textit{Journal of Logic and Computation} (online), 2013.

\bibitem{Fitting14}$\rule{1.30cm}{0.15mm}$. Possible world semantics for first order LP. {\it Annals of Pure and Applied Logic}, v. 165, p. 225-240, 2014.

\bibitem{Goedel33}G\"ODEL, Kurt. An interpretation of the intuitionistic propositional calculus. In: FEFERMAN, Solomon et al. (Ed.). \textit{Kurt G\"odel, Collected Works}, v.1. Oxford: Oxford University Press, 1986. p. 296-303.

\bibitem{Goedel38}$\rule{1.30cm}{0.15mm}$. Lecture at Zilsel’s. In: FEFERMAN, Solomon et al. (Ed.). \textit{Kurt G\"odel, Collected Works}, v.3. Oxford: Oxford University Press, 1995. p. 86-113.



\bibitem{Hughes96} HUGHES, George Edward; CRESSWELL, Max. \textit{A New Introduction to Modal Logic}. London: Routledge, 1996.

\bibitem{Kripke59} KRIPKE, Saul. A completeness theorem in modal logic. \textit{The Journal of Symbolic Logic}, v. 24, p. 1-14, 1959.

\bibitem{Kripke63} $\rule{1.30cm}{0.15mm}$. Semantical considerations on modal logic. \textit{Acta Philosophica Fennica}, v. 16, p. 83-94, 1963.

\bibitem{Kripke83} $\rule{1.30cm}{0.15mm}$. Failures of the interpolation lemma in quantified modal logic by Kit Fine. \textit{The Journal of Symbolic Logic}, v. 48, p. 486-488, 1983.

\bibitem{Willi} WILLIAMSON, Timothy. \textit{Modal logic as metaphysics}. Oxford: Oxford University Press, 2013. 
\end{thebibliography}



\end{document}
